\documentclass[11pt,twoside]{article}
\usepackage[authoryear,semicolon]{natbib}
\usepackage[T1]{fontenc}
\usepackage[utf8]{inputenc}
\usepackage{textgreek}
\usepackage{fullpage}
\usepackage[color=yellow]{todonotes}
\setlength{\marginparwidth}{1.25in}
\usepackage{xifthen}
\usepackage{amsmath,amssymb,amsthm,mathtools,stmaryrd}
\usepackage{proof,mathpartir}
\usepackage{colonequals}
\usepackage{code,verbatim}
\usepackage{comment}
\usepackage{textcomp}
\usepackage[us]{optional}
\usepackage{color}
\usepackage{url}
\usepackage{graphics}
\usepackage{import}
\usepackage{stackengine}
\usepackage{scalerel}

\newcommand{\eqdef}{\mathrel{\triangleq}}
\newcommand{\isdef}{\eqdef}

\newcommand{\DFun}[3]{{#2}{:}{#1}\to{#3}}

\allowdisplaybreaks[1]       %mildly permissible to break up displayed equations

\begin{document}

\title{Proving the Symmetry Condition on Type Systems}
\author{Robert Harper}
\date{\today}

\maketitle{}

The candidate type system $\mu$ is constructed using Tarski's Theorem as the least pre-fixed
point of an operator $\textsc{Types}(-)$ on candidate type systems.  The c.t.s. operator
\textsc{Types} is defined by several clauses, one of which is given by the c.t.s. operator
\textsc{Funs}:
\begin{displaymath}
  \textsc{Types}(\tau) \eqdef{} \dots \cup \textsc{Funs}(\tau) \cup \dots,
\end{displaymath}
where
\begin{gather}
  \textsc{Funs}(\tau)(\DFun{A_{1}}{a}{A_{2}},\DFun{A_{1}'}{a}{A_{2}'},\phi) \\
  \textit{iff} \\
  \phi\eqdef{} \DFun{\phi_{1}}{a}{\phi_{2}}\quad\textit{and} \\
  A_{1}\sim A_{1}'\downarrow\phi_{1}\in\tau \quad\textit{and}\\
  \forall M,M'\ \textit{if}\ \phi_{1}(M,M')\ \textit{then}\ A_{2}[M/a]\sim A_{2}'[M'/a]\downarrow\phi_{2}[M,M']\in\tau \
\end{gather}
and
\begin{gather}
  (\DFun{\phi_{1}}{a}{\phi_{2}})(\lambda a.N,\lambda a.N')\\
  \textit{iff}\  \\
  \forall M,M'\ \textit{if}\ \phi_{1}(M,M')\ \textit{then}\ \phi_{2}[M,M](N[M/a],N'[M'/a]).
\end{gather}

\bigskip

A candidate type system $\tau$ is a type system iff it satisfies the following conditions:
\begin{enumerate}
\item Unicity: if $\tau(A_{0},A_{0}',\phi)$ and $\tau(A_{0},A_{0}',\phi')$, then $\phi'=\phi$.
\item Symmetry: if $\tau(A_{0},A_{0}',\phi)$, then $\tau(A_{0}',A_{0},\phi)$;
\item Transitivity: if $\tau(A_{0},A_{0}',\phi)$ and $\tau(A_{0}',A_{0}'',\phi)$, then
  $\tau(A_{0},A_{0}'',\phi)$.
\item PER Valuation: if $\tau(A_{0},A_{0}',\phi)$, then $\phi$ is symmetric and transitive.
\end{enumerate}

The goal is to prove that $\mu$ is a type system.  The unicity condition may be proved
outright by proving that $\textsc{Types}(\Phi)\subseteq\Phi$, where
$\Phi(A_{0},A_{0}',\phi)$ iff $\mu(A_{0},A_{0}',\phi)$ and if $\mu(A_{0},A_{0}',\phi')$, then
$\phi'=\phi$.  For then $\mu\subseteq\Phi$, which means that $\mu$ enjoys unicity.  The proof was sketched in
class, and presents no further difficulties.

It is necessary to prove symmetry, transitivity, and PER valuation of $\mu$ simultaneously.
To do so, define $\Phi(A_{0},A_{0}',\phi)$ iff
\begin{enumerate}
\item $\mu(A_{0},A_{0}',\phi)$;
\item $\mu(A_{0}',A_{0},\phi)$.
\item if $\mu(B_0,A_0,\phi')$, then $\mu(B_{0},A_{0}',\phi')$ and $\phi'=\phi$.
\item if $\mu(A_{0}',B_{0}',\phi')$, then $\mu(A_{0},B_{0}',\phi')$ and $\phi'=\phi$.
\item $\phi$ is symmetric and transitive.
\end{enumerate}
The third and fourth conditions are called \emph{left transitivity} and \emph{right
  transitivity}, respectively.  They may be read as stating that if it is determined that
a type is equal to either of two equal types, then it is equal to the other, and their
membership relations coincide.  If $\textsc{Types}(\Phi)\subseteq\Phi$, then $\mu\subseteq\Phi$, which implies that
$\mu$ is a type system.

\bigskip

\newcommand{\relto}[1]{(\text{rel.}\,#1)}

The proof breaks into cases, one for each operator defining $\textsc{Types}$.  In
particular it is necessary to show that $\textsc{Funs}(\Phi)\subseteq\Phi$.  Suppose that
$\textsc{Funs}(\Phi)(A_{0},A_{0}',\phi)$, Then by definition
$A_{0}=\DFun{A_{1}}{a}{A_{2}}$, $A_{0}'=\DFun{A_{1}'}{a}{A_{2}'}$, and
$\phi=\DFun{\phi_{1}}{a}{\phi_{2}[a]}$, where
\begin{enumerate}
\item $A_{1}\sim A_{1}'\downarrow\phi_{1} \ \relto{\Phi}$;
\item if $M_1\sim M_1'\in\phi_{1}$, then $A_{2}[M_1/a]\sim A_{2}'[M_1'/a]\downarrow\phi_{2}[M_1,M_1']\ \relto{\Phi}$.
\end{enumerate}
These conditions imply that $\phi_{1}$ is a PER, and that
$A'_{2}[M_1'/a]\sim A_{2}[M_1/a]\downarrow \phi_{2}[M_1,M_1']\ \relto{\mu}$, with
$\phi_{2}[M_1,M_1']$ also a PER, whenever $M_1\sim M_1'\in\phi_{1}$.
% and $\phi[M_1,M_1']$ is a PER.

The overall obligation is to show that $\Phi(A_{0},A_{0}',\phi)$, which includes the particular
requirement that $\mu(A_{0}',A_{0},\phi)$.  Because $\textsc{Funs}(\mu)\subseteq\mu$, it suffices for this
case to show these two conditions:
\begin{enumerate}
\item $A_{1}'\sim A_{1}\downarrow\phi_{1}\ \relto{\mu}$, and
\item if $M_1\sim M_1'\in\phi_{1}$, then $A_{2}'[M_1/a]\sim A_{2}[M_1'/a]\downarrow\phi_{2}[M_1,M_1']\ \relto{\mu}$.
\end{enumerate}
The first of these follows directly from the assumption
$A_{1}\sim A_{1}'\downarrow\phi_{1}\ \relto{\Phi}$.  For the second, suppose that
$M_1\sim M_1'\in\phi_{1}$, with the intent to show that
$$A_{2}'[M_1/a]\sim A_{2}[M_1'/a]\downarrow \phi_{2}[M_1,M_1']\ \relto{\mu}.$$  Because $\phi_{1}$ is symmetric,
$M_1'\sim M_1\in\phi_{1}$, and so by the second assumption
$$A_{2}[M_1'/a] \sim A_{2}'[M_1/a]\downarrow\phi_{2}[M_1',M_1]\ \relto{\Phi}.$$  Therefore $\phi_{2}[M_1',M_1]$ is a
PER, and
$$A_{2}'[M_1/a]\sim A_{2}[M_1'/a]\downarrow\phi_{2}[M_1',M_1]\ \relto{\mu}.$$  To complete the proof it is
enough to show that $\phi_{2}[M_1',M_1]=\phi_{2}[M_1,M_1']$.

This equation is proved in two steps by showing that each side is equal to
$\phi_{2}[M_{1}',M_{1}']$.
\begin{enumerate}
\item By transitivity and symmetry of $\phi_{1}$, it follows that $M_{1}'\sim M_{1}'\in\phi_{1}$, and
  hence by the second assumption $$A_{2}[M_{1}'/a]\sim
  A_{2}'[M_{1}'/a]\downarrow\phi_{2}(M_{1}',M_{1}')\ \relto{\Phi},$$ and so by symmetry
  $$A_{2}'[M_{1}'/a]\sim A_{2}[M_{1}'/a]\downarrow\phi_{2}(M_{1}',M_{1}')\ \relto{\mu}.$$
  Moreover, by the second assumption, $$A_{2}[M_{1}'/a] \sim
  A_{2}'[M_{1}/a]\downarrow\phi_{2}[M_{1}',M_{1}]\ \relto{\Phi},$$ and hence $\relto{\mu}$.
  Therefore by right transitivity
  $$A_{2}'[M_{1}'/a]\sim A_{2}'[M_{1}/a]\downarrow\phi_{2}[M_{1}',M_{1}']\ \textit{and}\ \phi_{2}[M_{1}',M_{1}']=\phi_{2}[M_{1}',M_{1}].$$
\item
  As in the previous case $$A_{2}'[M_{1}'/a] \sim A_{2}[M_{1}'/a] \downarrow \phi_{2}[M_{1}',M_{1}']\ \relto{\mu},$$
  and $$A_{2}[M_{1}/a]\sim A_{2}[M_{1}'/a]\downarrow\phi_{2}[M_{1},M_{1}']\ \relto{\Phi}.$$
  By left transitivity
  $$A_{2}[M_{1}/a] \sim A_{2}[M_{1}'/a] \downarrow \phi_{2}[M_{1}',M_{1}']\ \relto{\mu}\ \textit{and}\
  \phi_2[M_1',M_1]=\phi_2[M_1',M_1'].$$
\end{enumerate}
Thus $\phi_{2}[M_{1},M_{1}']=\phi_{2}[M_{1}',M_{1}]$, as desired.

\smallskip

To prove that $\phi=a:\phi_{1}\to\phi_{2}[a]$ is symmetric, suppose that
$\phi(M_{0},M_{0}')$, which is to say that $M_{0}=\lambda a.M_{2}$,
$M_{0}'=\lambda a.M_{2}'$, and if $M_{1}\sim M_{1}'\in\phi_{1}$, then
$M_{2}[M_{1}/a]\sim M_{2}'[Μ_{1}'/α]\in \phi_{2}[M_{1},M_{1}']$.  To show that
$\phi(M_{0}',M_{0})$, suppose that $M_{1}\sim M_{1}'\in\phi_{1}$, with the intent to show that
$$M_{2}'[M_{1}/a]\sim M_{2}[M_{1}'/a]\in\phi_{2}[M_{1},M_{1}'].$$
Because $\phi_{1}$ is symmetric, we have $M_{1}'\sim M_{1} \in\phi_{1}$, and so
$$M_{2}[M_{1}'/a]\sim M_{2}'[M_{1}/a]\in\phi_{2}[M_{1}',M_{1}]$$
and so by symmetry
$$M_{2}'[M_{1}/a]\sim M_{2}[M_{1}'/a]\in\phi_{2}[M_{1}',M_{1}].$$
But by the foregoing argument $\phi_{2}[M_{1}',M_{1}]=\phi_{2}[M_{1},M_{1}']$, completing the
proof.

\bigskip

There is of course much more to the proof: the remaining conditions in $\Phi$ have to be
verified for the \textsc{Funs} case, and all of the other cases defining \textsc{Types}
must be considered as well.  However, the same devices used here apply in all of these
cases as well to complete the proof that $\mu$ is, in fact, a type system.

\bigskip

\noindent
\textbf{Acknowledgement}
The foregoing development is based on \citet{angiuli19}.

\nocite{angiuli19}

\bibliographystyle{plainnat}
\bibliography{notes}



\end{document}

%%% Local Variables:
%%% mode: latex
%%% TeX-master: t
%%% fill-column: 90
%%% auto-fill-mode: t
%%% End:
