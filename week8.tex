\documentclass{article} \usepackage{chtt-notes} \usepackage{stmaryrd}
\usepackage{amssymb}

\scribes{Yue Niu and Jonathan Laurent} \week{8}
% The following command will let you cross-reference labels in the
% files week1.tex, week2.tex, \ldots, week\@week.tex, where if l is a
% label in ``weekN.tex'', then you can access the label using
% \cref{WN:l}.
\doXRs

% General remark: Using \cref{label} will fill in the appropriate
% environment name. For example, ``\begin{lemma}\label{lem:foo}
%   ...\end{lemma} By \cref{lem:foo}'' will produce ``Lemma 15 ... By
% lemma~15''

\newcommand{\lift}[1]{{#1}^{\Downarrow}}
\newcommand{\meet}[1]{\bigwedge\!#1}
\newcommand{\join}[1]{\bigvee\!#1}

\begin{document}
\maketitle
\section{Hypothetico-General Judgments}

Last week, we looked briefly at the semantic judgment of computational dependent type theory with a one
element context. This can be extended hypothetical-generally with the idea functionality: 

Consider a h-g judgment: 

\[
\hasTEC{x_1 : A_1,...,x_n : A_n}{A}{A'}
\]

and the corresponding judgment on terms:

\[
\hasEEC{x_1 : A_1,...,x_n : A_n}{M}{M'}{A}
\]

They are defined mutually recursively by induction on $n$:

\begin{tabular}{l l}
$n = 0$ : & \hasTEC{\cdot}{A}{A'} \text{ iff } \hasTEC{A}{A'} and\\
				& \hasEEC{\cdot}{M}{M'}{A} \text{ iff } \hasEEC{M}{M'}{A}
			 \text{ (equivalent to the categorical judgments) }\\
$n = k + 1$: & \hasTEC{x_1 : A_1,...,x_{k+1} : A_{k+1}}{A}{A'} \text{ iff }\\
& \hasEEC{M_1}{M_1'}{A_1}, \hasEEC{M_2}{M_2'}{[M_1/x_1]A_2},...,
	\hasEEC{M_{k+1}}{M_{k+1}'}{[M_i/x_i]^{k+1}_{i=1}A_{k+1}} \text{ implies } \\
& \hasTEC{[M_i/x_i]^{k+1}_{i=1}A}{[M_i'/x_i]^{k+1}_{i=1}A'} and \\
& \hasEEC{x_1 : A_1,...,x_{k+1} : A_{k+1}}{M}{M'}{A} iff\\
& \hasEEC{M_1}{M_1'}{A_1}, \hasEEC{M_2}{M_2'}{[M_1/x_1]A_2},...,
	\hasEEC{M_{k+1}}{M_{k+1}'}{[M_i/x_i]^{k+1}_{i=1}A_{k+1}} \text{ implies } \\
&	\hasEEC{[M_i/x_i]^{k+1}_{i=1}M}{[M_i'/x_i]^{k+1}_{i=1}M'}{[M_i/x_i]^{k+1}_{i=1}A}\\
\end{tabular}

Note we define the unary versions of the h-g judgments as reflexive cases (e.g. \hasTC{\Gamma}{A} iff 
\hasTEC{\Gamma}{A}{A}). We can check that the h-g judgments are symmetric and transitive, 
and by the ``p.e.r'' trick, we can get reflexivity as well.  
In addition, the h-g jugdments also have the structural properties of entailment: 

\begin{enumerate}
\item \hasEC{\Gamma,x : A,\Gamma'}{x}{A}
\item if \hasTC{\Gamma}{\mathcal{J}} and \hasTC{\Gamma}{A} then \hasTC{\Gamma,x:A}{\mathcal{J}}
\item if \hasTC{\Gamma,x:A,\Gamma'}{\mathcal{J}} and \hasEEC{\Gamma}{M}{M'}{A} then 
	\hasTC{\Gamma[M/x]\Gamma'}{[M/x]\mathcal{J}}
\end{enumerate} 

Where $\mathcal{J}$ any antecedant of a semantic h-g judgment. With these properties, we can embed a logic in 
this framework: 
\begin{enumerate}
\item inhabitantion: \trueF{A} iff there is a term $M$ s.t. \hasEC{M}{A}
\item equality of proofs: $\Eq{A}{M}{N}$
\end{enumerate}

\subsection{Function Equality}

Semantic function equality is extensional, since functionality is built into the definition of the semantic
judgments:

\begin{align*}
\hasEEC{\lam{x}{A}{M}}{\lam{x}{A}{M'}}{\Pitype{x}{A}{B}} &\text{ iff } \hasEEC{x : A}{M}{M'}{B}\\
\end{align*}

where the context is a mapping of variables to \emph{closed} values.

\subsection{Equality Type}

The type $\Eq{A}{M}{N}$ internalizes semantic judgmental equality: 

\begin{align*}
\hasTEC{\Eq{A}{M}{N}}{\Eq{A'}{M'}{N'}} &\text{ iff } \hasTEC{A}{A'}, \hasEEC{M}{M'}{A}, and \hasEEC{N}{N'}{A}\\
\hasEC{\refl{A}{M}}{\Eq{A}{M}{N}} &\text{ iff } \hasEEC{M}{N}{A}\\
\end{align*}

\section{An Explicit Construction of Computational Dependent Types}

In this section, we give a rigorous set-theoretic construction of
computational dependent types (CDTs), as introduced by Allen
\cite{Allen:87}. In this construction, we define the system of
computational dependent types as the least fixed point of a monotone
operator over a lattice of candidate type-systems. The existence of
such a fixed point is guaranteed by the Knaster-Tarski theorem.

\subsection{Background: Knaster-Tarski Fixed Point Theorem}

A \emph{complete lattice} $(L, \leq)$ consists in a set $L$ along with
a pre-order\footnote{A pre-order is a reflexive and transitive
  relation.} $\leq$ such that all subsets of $L$ admit a greatest
lower bound (aka \emph{meet}) and a least upper bound (aka
\emph{join}).  More formally, for every subset $X \subseteq L$, there
are elements $\meet{X}$ and $\join{X}$ in $L$ such that:
\[\begin{array}{ccc}
    \forall x\in X, \ \meet{X} \,\leq\, x  &  \text{and}  &
    \forall y,\, \left( \forall x\in X, \ y \,\leq\, x  \right) \ \Longrightarrow \ 
    y \,\leq\, \meet{X} \\ \\
    \forall x\in X, \ x \,\leq\, \join{X}  & \text{and}  &
    \forall y,\, \left( \forall x\in X, \ x \,\leq\, y  \right) \ \Longrightarrow \ 
    \join{X} \,\leq\, y.
  \end{array}\]
\begin{remark*}
  If $L$ is infinite, this is a stronger requirement than just having
  binary meet and join operators $\wedge$ and $\vee$ on $L$ along with
  smallest and largest elements $\top$ and $\bot$ (in which case
  $(L, \leq)$ is called a \emph{plain lattice}).
\end{remark*}
A function $f : L \to L$ is said to be \emph{monotone} if it preserves
$\leq$ in the sense that
\[ \forall x, y \in L, \ x \leq y \ \Longrightarrow \ f(x) \leq
  f(y).\]

We can now state Knaster-Tarski theorem.

\begin{theorem*}
  Let $(L, \leq)$ a complete lattice and $f : L \to L$ a monotone
  function.  Then, $f$ admits a complete lattice of fixed points. In
  particular, it admits a least fixed point $\mu f$ which is given by
  the meet of its pre-fixed points:
  \[ \mu f = \bigwedge \{ x \in L \ |\ f(x) \leq x\}.\]
\end{theorem*}

In practice, $(L, \leq)$ is often a powerset lattice:
$(L, \leq) = (\mathcal{P}(E), \subseteq)$ where $E$ is a set and
$\subseteq$ refers to set inclusion. In this case, the meet operation
corresponds to set intersection and the join operation corresponds to
set union. Therefore, if $f : \mathcal{P}(E) \to \mathcal{P}(E)$ is
monotone, $\mu f$ is the intersection of every subset of $E$ that is
\emph{closed under} $f$:
\[ \mu f = \bigcap \ \{ X \subseteq E \ | \ f(X) \subseteq X \}. \]


\subsection{Construction of Computational Dependent Types}

\subsubsection{Candidate type systems}

A \emph{candidate type-system} is a ternary relation between values,
values, and binary relations over values.  Intuitively, if
$\tau(A, B, \phi)$ then $A$ and $B$ are closed values corresponding to
equal types and $\phi$ is a binary relation over closed values that
corresponds to equality of values within type $A$ (or $B$).  For
example, a type system for booleans would be
\[\tau_{\bool} = \{(\bool, \bool, \beta)\}, \qquad
  \beta = \{(\true, \true), (\false, \false)\}.\]

\subsubsection{Real type systems}
A \emph{real type-system} is defined as a candidate type system for
which the following properties hold (every free variable is
universally quantified):
\begin{enumerate}
\item \textbf{Functionality:} $\tau(A, B, \phi)$ and
  $\tau(A, B, \phi')$ imply $\phi = \phi'$
\item \textbf{\textsc{P.E.R} valued:} if $\tau(A, B, \phi)$, then
  $\phi$ is a partial equivalence relation\footnote{A partial
    equivalence relation is a symmetric and transitive relation.}
\item \textbf{Symmetry:} $\tau(A, B, \phi)$ implies $\tau(B, A, \phi)$
\item \textbf{Transitivity:} $\tau(A, B, \phi)$ and $\tau(B, C, \phi)$
  imply $\tau(A, C, \phi)$.
\end{enumerate}

\medskip

\begin{remark*}
  Another way to formulate transitivity would be as follows:
  \begin{center}
    $\tau(A, B, \phi)$ and $\tau(B, C, \phi')$ imply
    $\tau(A, C, \phi)$ and $\phi = \phi'$.
  \end{center}
  This is equivalent to our first formulation.  Indeed, suppose
  $\tau(A, B, \phi)$ and $\tau(B, C, \phi')$. Using symmetry and
  transitivity, we have $\tau(B, B, \phi)$ and $\tau(B, B, \phi')$
  (p.e.r trick).  Using functionality, we get $\phi = \phi'$.
\end{remark*}

The set of all candidate type-systems forms a complete lattice for set
inclusion. This is \textbf{not} true of the set of all real
type-systems, which is why we need the notion of candidate type-system
(see section \ref{subsub:tau0}).


\subsubsection{Truth judgments relative to a real type system}

We define the following truth judgments relative to a real type-system
$\tau$:
\begin{itemize}
\item $\tau \models \hasTEC{A}{B}$ if and only if
  $\lift{\tau}(A, B, \phi)$ for some $\phi$
\item $\tau \models \hasEEC{M}{N}{A}$ if and only if
  $\lift{\tau}(A, B, \phi)$ and $\lift{\phi}(M, N)$ for some $\phi$
\end{itemize}
where the lifted relations $\lift{\tau}$ and $\lift{\phi}$ are defined
as follows:
\begin{itemize}
\item $\lift{\phi}(M, N)$ iff there exists values $V, W$ such that
  $\eval{M}{V}$, $\eval{N}{W}$ and $\phi(V, W)$
\item $\lift{\tau}(A, B, \phi)$ iff.  there exists values $V, W$ such
  that $\eval{A}{V}$, $\eval{B}{W}$ and $\tau(V, W, \phi)$.
\end{itemize}

\subsubsection{Defining $\tau_0$ as a least fixed
  point}\label{subsub:tau0}

We can define the type system $\tau_0$ of computational dependent
types as the least fixed-point of a monotone operator
\[ \Phi(\tau) \triangleq \textsc{bool} \cup \textsc{nat} \cup
  \textsc{eq}(\tau) \cup \textsc{pi}(\tau) \cup
  \textsc{sigma}(\tau).  \] After we define $\textsc{bool}$,
$\textsc{nat}$, $\textsc{eq}$, $\textsc{pi}$ and $\textsc{tau}$, we
must prove that the resulting operator $\Phi$ is monotone and that
$\tau_0 \triangleq \mu \Phi$ is a real type-system.

\paragraph{Booleans}
As seen earlier,
\[ \textsc{bool} = \{(\bool, \bool, \beta)\} \text{ where } \beta =
  \{(\true, \true), (\false, \false)\}. \]

\paragraph{Natural numbers}
We define $ \textsc{nat} = \{(\nat, \nat, \mu N)\} $ where $\mu N$ is
the least fixed-point of the following monotone operator over the
complete lattice of relations over values:
\[ N(\alpha) = \{(\natz, \natz)\} \cup \{ (\nats{n}, \nats{m}) \ | \
  \alpha(m,n) \}. \]

\paragraph{Equality}
We have $\textsc{eq}(\tau)(A_0, A'_0, \epsilon)$ if and only if
\begin{itemize}
\item $A_0 = \textsf{Eq}_A(M, N)$ and
  $A_0' = \textsf{Eq}_{A'}(M', N')$ for some $A, A', M, M', N, N'$
\item $\lift{\tau}(A, A', \phi)$, $\lift{\phi}(M, M')$ and
  $\lift{\phi}(N, N')$ for some $\phi$
\item $\epsilon = \{(\textsf{refl}, \textsf{refl})\}$ if $\phi(M,N)$
  and $\{\}$ otherwise.
\end{itemize}

\paragraph{Dependent function}
We have $\textsc{pi}(\tau)(A_0, A'_0, \rho)$ if and only if
\begin{itemize}
\item $A_0 = (x:A \to B)$ and $A_0' = (x:A' \to B')$ for some
  $A,A',B,B'$
\item $\tau(A, A', \alpha)$ for some $\alpha$
\item there exists a function $\psi$ mapping pairs of terms to
  relations over values such that
  \begin{center}
    $\lift{\alpha}(M, M')$ implies
    $\lift{\tau}\left(\subst{M}{x}{B}, \, \subst{M'}{x}{B'}, \,
      \psi(M,M')\right)$ for all terms $M, M'$
  \end{center}
\item $\rho(M, M')$ if and only if
  \begin{itemize}
  \item $M = \lambda x. N$ for some $N$
  \item $M' = \lambda x. N'$ for some $N'$
  \item $\lift{\alpha}(P, P')$ implies
    $\lift{\psi(P, P')}(\subst{P}{x}{N}, \, \subst{P'}{x}{N'})$ for
    all terms $P, P'$.
  \end{itemize}
\end{itemize}

\paragraph{Dependent products}
We have $\textsc{sigma}(\tau)(A_0, A'_0, \sigma)$ if and only if
\begin{itemize}
\item $A_0 = (x:A \times B)$ and $A_0' = (x:A' \times B')$ for some
  $A,A',B,B'$
\item $\tau(A, A', \alpha)$ for some $\alpha$
\item there exists a function $\psi$ mapping pairs of terms to
  relations over values such that
  \begin{center}
    $\lift{\alpha}(M, M')$ implies
    $\lift{\tau}\left(\subst{M}{x}{B}, \, \subst{M'}{x}{B'}, \,
      \psi(M,M')\right)$ for all terms $M, M'$
  \end{center}
\item $\sigma(P, P')$ if and only if
  \begin{itemize}
  \item $P = \textsf{pair}_{A, x.B}(M,N)$ for some $M, N$
  \item $P' = \textsf{pair}_{A', x.B'}(M',N')$ for some $M', N'$
  \item $\lift{\alpha}(M, M')$ and $\lift{\psi(M, M')}(N, N')$.
  \end{itemize}
\end{itemize}

\subsubsection{Proving the typing rules for $\tau_0$}

Finally, we must show that all the typical typing rules are true with
respect to $\tau_0$. For example, we must show that if
$\tau_0 \models \hasEF{M}{(x:A \to B)}$ and
$\tau_0 \models \hasEF{M}{A}$ then
$\tau_0 \models \hasEF{MN}{\subst{N}{x}{B}}$.


\subsection{Discussion}

In order to show that $\Phi$ (as defined in \ref{subsub:tau0}) is
monotone, it is sufficient to show that $\textsc{eq}$, $\textsc{pi}$
and $\textsc{sigma}$ are monotone. A full proof would be tedious but
the core idea that makes it work is that $\tau$ only appears in
covariant positions in the definitions of $\textsc{eq}(\tau)$,
$\textsc{pi}(\tau)$ and $\textsc{sigma}(\tau)$.  In the case of
$\textsc{pi}$ for example, this reflects the fact that the terms $A$
and $\subst{M}{x}{B}$ for all $M$ can be considered as defined prior
to $x:A \to B$.

This would not hold for the type constructor $\forall X. A$, which
cannot regard $\subst{B}{X}{A}$ as prior to $\forall X. A$ for all
$B$, as $B$ could be $\forall X. A$. More concretely, here is an
attempt of adding universal quantification over types in our language
by extending $\Phi$ with a $\textsc{forall}$ operator:

\paragraph{Type quantification} 
We have $\textsc{forall}(\tau)(A_0, A'_0, F)$ if and only if
\begin{itemize}
\item $A_0 = \forall X. A$ and $A_0' = \forall X. A'$
\item $\tau(A, A', \alpha)$ for some $\alpha$
\item There is a mapping $\psi$ from pairs of terms to relations over
  values such that
  \begin{center}
    $\color{red}{\lift{\tau}(T, T')}$ implies
    $\lift{\tau}(\subst{T}{x}{A}, \subst{T'}{x'}{A'}, \psi(T, T'))$
  \end{center}
\item $F(M, M')$ if and only if
  \begin{itemize}
  \item $M = \Lambda X. N$ for some term $N$
  \item $M' = \Lambda X. N'$ for some term $N'$
  \item $\color{red}{\lift{\tau}(T, T')}$ implies
    $\lift{\psi(T, T')} (\subst{T}{x}{N}, \subst{T'}{x}{N'})$ for all
    terms $T, T'$
  \end{itemize}
\end{itemize}

Such an operator is \textbf{not} monotone though, as can be seen by
the contravariant occurences of $\tau$ in the definition of
$\textsc{forall}(\tau)$ (highlighted in red).

\bigskip

Similarly, in the definition of $\textsc{nat}$, it is crucial that the
$N$ relation is monotone so that $\mu N$ is well-defined.  This
monotonicity constraint prevents us from giving a fixed point
definition for a type like $D \simeq D \to D$. Indeed, it is tempting
to define a type-system for $D$ as follows:
$\tau_D \triangleq \{(D, D, \mu X)\}$ where $\mu X$ is the least-fixed
point of $X$:
\[ X(\alpha) \triangleq \{ (\textsf{Fun}(f), \textsf{Fun}(g)) \ | \
  \forall V, W. \ {\color{red}{\alpha(V, W)}} \implies
  \lift{\alpha}(f(V), f(W)) \} \] Such an operator is \textbf{not}
monotone though, $\alpha$ appearing in a contravariant position in the
definition of $X(\alpha)$. Therefore, $\mu X$ does not necessarily
exist. This makes sense because were a type like $D$ allowed, we would
have non-termination, which we do not want at this stage.


\bibliographystyle{plainnat} \bibliography{ctt}

\end{document}

%%% Local Variables:
%%% mode: latex
%%% TeX-master: t
%%% End:
