\documentclass[11pt,twoside]{article}
\usepackage[authoryear,semicolon]{natbib}
\usepackage[T1]{fontenc}
\usepackage[utf8]{inputenc}
\usepackage{textgreek}
\usepackage{fullpage}
\usepackage[color=yellow]{todonotes}
\setlength{\marginparwidth}{1.25in}
\usepackage{xifthen}
\usepackage{amsmath,amssymb,amsthm,mathtools,stmaryrd}
\usepackage{proof,mathpartir}
\usepackage{colonequals}
\usepackage{code,verbatim}
\usepackage{comment}
\usepackage{textcomp}
\usepackage[us]{optional}
\usepackage{color}
\usepackage{url}
\usepackage{graphics}
\usepackage{import}
\usepackage{stackengine}
\usepackage{scalerel}

\newtheorem{theorem}{Theorem}
\newtheorem{lemma}[theorem]{Lemma}

\newcommand{\subst}[3]{[{#1}/{#2}]{#3}}

\newcommand{\eqdef}{\mathrel{\triangleq}}
\newcommand{\isdef}{\eqdef}

\newcommand{\parens}[1]{(#1)}

\newcommand{\const}[1]{\text{#1}}

\newcommand{\LF}[1][]{\ensuremath{\mathsf{LF}_{#1}}}
\newcommand{\bnfdef}{\mathrel{{:}{:}{=}}}
\newcommand{\bnfalt}{\mathrel{\mid}}

\newcommand{\sortclass}{\ensuremath{\textbf{sort}}}
\newcommand{\eqclass}[3]{{#2}=_{#1}{#3}}
\newcommand{\piclass}[3]{({#2}\mathbin{:}{#1})\,{#3}}
\newcommand{\arrclass}[2]{{#1}\to{#2}}
\newcommand{\lamobj}[3]{[{#2}\mathbin{:}{#1}]\,{#3}}
\newcommand{\appobj}[2]{{#1}\,{#2}}
\newcommand{\reflobj}{\star}
\newcommand{\empctx}{\varepsilon}
\newcommand{\snocctx}[3]{{#1}\mathbin{,}{#2}{:}{#3}}
\newcommand{\appctx}[2]{{#1}\,{#2}}

\newcommand{\issig}[1]{{#1}\;\mathsf{sig}}
\newcommand{\isctx}[2][]{{#2}\;\mathsf{ctx}_{#1}}
\newcommand{\eqctx}[3][]{{#2}={#3}\;\mathsf{ctx}_{#1}}
\newcommand{\iscls}[3][]{{#2}\vdash_{#1}{#3}\;\mathsf{cls}}
\newcommand{\eqcls}[4][]{{#2}\vdash_{#1}{#3}={#4}\;\mathsf{cls}}
\newcommand{\isobj}[4][]{{#2}\vdash_{#1}{#3}:{#4}}
\newcommand{\eqobj}[5][]{{#2}\vdash_{#1}{#3}={#4}:{#5}}

\allowdisplaybreaks[1]       %mildly permissible to break up displayed equations

\begin{document}

\title{A Semantic Logical Framework}
\author{Robert Harper}
\date{\today}

\maketitle{}

\section{Introduction}

A \emph{logical framework} is language for defining logical systems, in particular type
theories.  The definition of a logical system consists of a collection of
\emph{generators} that populate a collection of classifying \emph{sorts}, and a collection
of \emph{equations} that govern the objects of those sorts.  The generators specify the
sorts of the objects that constitute the logical system---say, the sort of types and, for
each type, the sort of its elements---and in addition specify the objects of each of these
sorts.  These objects comprise the syntactic entities of the logical system---the types and
their elements---and are specified using \emph{higher-order abstract syntax} to express the
binding and scopes of variables.  The native equality of the framework is a congruence---an
equivalence relation compatible with the generators---and defines substitution of objects
for variables in another object.  The defining equations of a logical system enrich the
native equality to specify the behavior of the represented objects---such as the inversion
and unicity properties of type constructors or connectives.

The integration of the defining equations with the native equations of the framework
constrains the \emph{meaning} of the defined system within the framework in the sense that
any interpretation must obey the specified laws.  There being no limitations on the nature
of these equations, the enriched equality judgment of the framework may or may not be
(feasibly or infeasibly) decidable.  A \emph{syntactic logical
  framework}~\cite{harper-etal:lf} is one that presents a logical system using only
generators, and no relations, so that the induced equational theory is the native one,
which is decidable.  A \emph{semantic logical theory}~\cite{smith1990programming} admits
the specification of an equational theory that may not be (feasibly) decidable.  Each
logical system is a problem of its own, and it is in general difficult to transfer results
from one case to another.

This note defines a semantic logical framework suitable for defining a broad---but by no
means comprehensive---class of logical systems, including full-scale dependent type
theories.  It is a dependently typed language with a single Russellian universe of sorts,
an extensional equality types governing the objects of a sort, and closed under the
formation of dependent function types.  The definition of a logical system is a form of
context, called a \emph{signature}, that specifies generators that populate the sorts and
the equality types that govern them.  The \emph{adequacy} of a signature expresses the
intended correspondence between the components of the represented logical system and their
counterpart objects in the logical framework.

\section{A Logical Framework}

The syntax of the logical framework \LF{} is given in Figure~\ref{fig:lf-syntax}.  It is a
dependently typed $\lambda$-calculus with the structure specified in the introduction.  The
types of \LF{} are called \emph{classes}, or \emph{kinds}, $K$ or $S$, and their elements
are called \emph{objects}, $O$ or $S$.\footnote{The double role of the meta-variable $S$
  will be explained shortly.}  The notation is inspired by \textsf{AUTOMATH}, using square
brackets for $\lambda$-abstraction, round brackets for $\Pi$-types, and juxtaposition for
application.  The binding and scopes of identifiers are understood; all classes and
objects are identified up to renaming of bound variables.  Substitution of an object for a
variable within a class or kind is defined in the usual way up to such renamings.

\begin{figure}[tp]
  
  \begin{displaymath}
    \begin{array}{lrcl}
      \textit{Variables} & X,x \\
      \textit{Kinds} & K, S & \bnfdef{} & S \bnfalt \sortclass \bnfalt \piclass{K_{1}}{X}{K_{2}}
                                          \bnfalt \eqclass{S}{O_{1}}{O_{2}} \\
      \textit{Objects} & O, S & \bnfdef & X \bnfalt \lamobj{K_{1}}{X}{K_{2}} \bnfalt
                                          \appobj{O_{1}}{O_{2}} \bnfalt \reflobj{} \\
      \textit{Contexts} & \Gamma & \bnfdef & \empctx{} \bnfalt \snocctx{\Gamma}{X}{K}
    \end{array}
  \end{displaymath}

  \caption{Abstract Syntax of \LF{}}
  \label{fig:lf-syntax}
\end{figure}

\LF{} uses dependent function classes to define the \emph{hypothetico-general judgment
  form} that is central to the definition of many logical systems.  It uses (extensional)
equality classes as a convenient way to present equational theories, and it uses the
objects of class \sortclass{} for the syntactic categories of a logical system.  It is
itself defined in the conventional manner in terms of the hypothetico-general judgment
forms given in Figure~\ref{fig:lf-judgments}.

\begin{figure}[tp]
  \centering
  \begin{tabular}{l@{\qquad}l}
    $\isctx{\Gamma}$ & $\Gamma$ is a context \\[1ex]
    $\iscls{\Gamma}{K}$ & $K$ is a class in context $\Gamma$ \\
    $\isobj{\Gamma}{O}{K}$ & $O$ is an object of class $K$ in context $\Gamma$ \\[1ex]
    $\eqcls{\Gamma}{K}{K'}$ & $K$ and $K'$ are equal classes in context $\Gamma$ \\
    $\eqobj{\Gamma}{O}{O'}{K}$ & $O$ and $O'$ are equal objects of class $K$ in context $\Gamma$
  \end{tabular}
  \caption{Judgment Forms of \LF{}}
  \label{fig:lf-judgments}
\end{figure}

A \emph{signature} $\Sigma$ is a context.  The variables declared in a signature are written
$C$ and $c$ to suggest their role as constants, or generators.  The specialization of the
judgment forms of \LF{} to a particular signature $\Sigma$ are defined in
Figure~\ref{fig:lf-specialized} using concatenation of contexts defined in the evident
manner.

\begin{figure}[tp]
  \centering
  \begin{tabular}{l@{\qquad\qquad}l}
    $\issig{\Sigma}$             & $\isctx{\Sigma}$ \\[1ex]

    $\isctx[\Sigma]{\Gamma}$           & $\isctx{\Sigma\,\Gamma}$  \\[1ex]
    
    $\iscls[\Sigma]{\Gamma}{K}$           & $\iscls{\Sigma\,\Gamma}{K}$ \\
    $\isobj[\Sigma]{\Gamma}{O}{K}$     & $\isobj{\Sigma\,\Gamma}{O}{K}$ \\[1ex]

    $\eqcls[\Sigma]{\Gamma}{K}{K'}$   & $\eqcls{\Sigma\,\Gamma}{K}{K'}$ \\
    $\eqobj[\Sigma]{\Gamma}{O}{O'}{K}$ & $\eqobj{\Sigma\,\Gamma}{O}{O'}{K}$
  \end{tabular}
  \caption{Specialized Judgment Forms of \LF{}}
  \label{fig:lf-specialized}
\end{figure}

The rules defining the \LF{} judgment forms are given in Figures~\ref{fig:lf-is},
\ref{fig:lf-str}, and~\ref{fig:lf-eq} which are to be understood as constituting one
simultaneous inductive definition.  Rule~\textsc{incl} specifies that every sort is itself
a class; the class $\sortclass$ is thus formulated as a Russellian universe of ``small''
classes.  Rule~\textsc{pi}, and associated rules~\textsc{lam} and~\textsf{app}, could be
restricted without loss to require that $K_{1}$ be a small class.  Rule~\textsc{eq}
defines the class of equations between objects of a sort; rule~\textsc{self} specifies
that every object is equal to itself.  Rules~\textsc{obj-cls} and~\textsc{obj-eq-cls}
specifies that equal classes classify the same objects.  Rules~\textsc{app-lam}
and~\textsc{lam-app} specify the inversion principles governing abstraction and
application.  Rule~\textsc{reflection} specifies that the equality class internalizes
equality, and rule~\textsc{unicity} specifies that an equation is ``at most true,'' there
being no distinction between two objects witnessing its truth.

\begin{lemma}[Presuppositions]
  \label{lemma:presup}
  \begin{enumerate}
  \item If\/ $\iscls{\Gamma}{K}$, then $\isctx{\Gamma}$, and if\/ $\eqcls{\Gamma}{K}{K'}$, then $\iscls{\Gamma}{K}$
    and $\iscls{\Gamma}{K'}$.
  \item If\/ $\isobj{\Gamma}{O}{K}$, then $\iscls{\Gamma}{K}$, and if\/ $\eqobj{\Gamma}{O}{O'}{K}$, then
    $\isobj{\Gamma}{O}{K}$ and $\isobj{\Gamma}{O}{K'}$.
  \end{enumerate}
\end{lemma}

\begin{lemma}[Weakening]
  \label{lemma:weak}
  Suppose that $\isctx[\Gamma]{\Gamma'}$.
  \begin{enumerate}
  \item If\/ $\iscls{\Gamma}{K}$, then $\iscls{\Gamma\,\Gamma'}{K}$, and if\/ $\eqcls{\Gamma}{K}{K'}$, then $\eqcls{\Gamma\,\Gamma'}{K}{K'}$.
  \item If\/ $\isobj{\Gamma}{O}{K}$, then $\isobj{\Gamma\,\Gamma'}{O}{K}$, and if\/
    $\eqobj{\Gamma}{O}{O'}{K}$, then $\eqobj{\Gamma\,\Gamma'}{O}{O'}{K}$.
  \end{enumerate}
\end{lemma}

\begin{lemma}[Substitution]
  \label{lemma:subst}
  Suppose that $\isctx{\Gamma_{1}\,X{:}K_{1}\,\Gamma_{2}}$ and $\isobj{\Gamma}{O_{1}}{K_{1}}$.
  \begin{enumerate}
%  \item $\isctx{\Gamma_{1}\,\subst{O_{1}}{X}{\Gamma_{2}}}$.
  \item If\/ $\iscls{\Gamma_{1}\,X{:}K_{1}\,\Gamma_{2}}{K_{2}}$, then
    $\iscls{\Gamma_{1}\,\subst{O_{1}}{X}{\Gamma_{2}}}{\subst{O_{1}}{X}{K_{2}}}$, and similarly for class equality.
  \item If\/ $\isobj{\Gamma_{1}\,X{:}K_{1}\,\Gamma_{2}}{O_{2}}{K_{2}}$, then
    $\isobj{\Gamma_{1}\,\subst{O_{1}}{X}{\Gamma_{2}}}{\subst{O_{1}}{X}{O_{2}}}{\subst{O_{1}}{X}{K_{2}}}$,
    and similarly for object equality.
  \end{enumerate}
\end{lemma}

\begin{lemma}[Functionality]
  \label{lemma:func}
  Suppose that $\isctx{\Gamma_{1}\,X{:}K_{1}\,\Gamma_{2}}$ and $\eqobj{\Gamma}{O_{1}}{O_{1}'}{K_{1}}$.
  \begin{enumerate}
  \item If\/ $\iscls{\Gamma_{1}\,X{:}K_{1}\,\Gamma_{2}}{K_{2}}$, then
    $\eqcls{\Gamma_{1}\,\subst{O_{1}}{X}{\Gamma_{2}}}{\subst{O_{1}}{X}{K_{2}}}{\subst{O_{1}'}{X}{K_{2}}}$.
  \item If\/ $\isobj{\Gamma_{1}\,X{:}K_{1}\,\Gamma_{2}}{O_{2}}{K_{2}}$, then
    $\eqobj{\Gamma_{1}\,\subst{O_{1}}{X}{\Gamma_{2}}}{\subst{O_{1}}{X}{O_{2}}}{\subst{O_{1}'}{X}{O_{2}}}{\subst{O_{1}}{X}{K_{2}}}$,
  \end{enumerate}
\end{lemma}

\begin{figure}[tp]
  
  \begin{mathpar}
    
    \inferrule[sort]
    {\isctx{\Gamma}}
    {\iscls{\Gamma}{\sortclass}}

    \inferrule[incl]
    {\isobj{\Gamma}{S}{\sortclass}}
    {\iscls{\Gamma}{S}}

    \inferrule[pi]
    {\iscls{\Gamma}{K_{1}} \\ \iscls{\snocctx{\Gamma}{X}{K_{1}}}{K_{2}}}
    {\iscls{\Gamma}{\piclass{K_{1}}{X}{K_{2}}}}

    \inferrule[eq]
    {\isobj{\Gamma}{S}{\sortclass} \\ \isobj{\Gamma}{O_{1}}{S} \\ \isobj{\Gamma}{O_{2}}{S}}
    {\iscls{\Gamma}{\eqclass{S}{O_{1}}{O_{2}}}}

  \end{mathpar}

  \begin{mathpar}

    \inferrule[lam]
    {\iscls{\Gamma}{K_{1}} \\ \isobj{\snocctx{\Gamma}{X}{K_{1}}}{O_{2}}{K_{2}}}
    {\isobj{\Gamma}{\lamobj{K_{1}}{X}{O_{2}}}{\piclass{K_{1}}{X}{K_{2}}}}

    \inferrule[app]
    {\isobj{\Gamma}{O}{\piclass{K_{1}}{X}{K_{2}}} \\ \isobj{\Gamma}{O_{1}}{K_{1}}}
    {\isobj{\Gamma}{\appobj{O}{O_{1}}}{\subst{O_{1}}{X}{K_{2}}}}

    \inferrule[self]
    {\isobj{\Gamma}{O}{S}}
    {\isobj{\Gamma}{\reflobj}{\eqclass{S}{O}{O}}}

  \end{mathpar}

  \caption{Formation Judgments}
  \label{fig:lf-is}
\end{figure}

\begin{figure}[tp]
  
  \begin{mathpar}
    
    \inferrule[emp]
    {\strut}
    {\isctx{\empctx}}

    \inferrule[snoc]
    {\iscls{\Gamma}{K}}
    {\isctx{\snocctx{\Gamma}{X}{K}}}

    \inferrule[var]
    {\isctx{\Gamma_{1}\,X{:}K\,\Gamma_{2}}}
    {\isobj{\Gamma_{1}\,X{:}K\,\Gamma_{2}}{X}{K}}

  \end{mathpar}

  \begin{mathpar}
    
    \inferrule[cls-rfl]
    {\iscls{\Gamma}{K}}
    {\eqcls{\Gamma}{K}{K}}

    \inferrule[cls-st]
    {\eqcls{\Gamma}{K}{K'} \\ \eqcls{\Gamma}{K''}{K'}}
    {\eqcls{\Gamma}{K}{K''}}

  \end{mathpar}

  \begin{mathpar}

    \inferrule[obj-rfl]
    {\isobj{\Gamma}{O}{K}}
    {\eqobj{\Gamma}{O}{O}{K}}

    \inferrule[obj-st]
    {\eqobj{\Gamma}{O}{O'}{K} \\ \eqobj{\Gamma}{O''}{O'}{K}}
    {\eqobj{\Gamma}{O}{O''}{K}}

  \end{mathpar}

  \begin{mathpar}
    
    \inferrule[obj-cls]
    {\isobj{\Gamma}{O}{K} \\ \eqcls{\Gamma}{K}{K'}}
    {\isobj{\Gamma}{O}{K'}}

    \inferrule[obj-eq-cls]
    {\eqobj{\Gamma}{O}{O'}{K} \\ \eqcls{\Gamma}{K}{K'}}
    {\eqobj{\Gamma}{O}{O'}{K'}}

  \end{mathpar}

  \caption{Structural Judgments}
  \label{fig:lf-str}
\end{figure}

\begin{figure}[tp]
  
  \begin{mathpar}
    
    \inferrule[incl-eq]
    {\eqobj{\Gamma}{S}{S'}{\sortclass}}
    {\eqcls{\Gamma}{S}{S'}}

    \inferrule[pi-eq]
    {\eqcls{\Gamma}{K_{1}}{K_{1}'} \\ \eqcls{\snocctx{\Gamma}{X}{K_{1}}}{K_{2}}{K_{2}'}}
    {\eqcls{\Gamma}{\piclass{K_{1}}{X}{K_{2}}}{\piclass{K_{1}'}{X}{K_{2}'}}}

    \inferrule[eq-eq]
    {\eqobj{\Gamma}{S}{S'}{\sortclass} \\ \eqobj{\Gamma}{O_{1}}{O_{1}'}{S} \\ \eqobj{\Gamma}{O_{2}}{O_{2}'}{S}}
    {\eqcls{\Gamma}{\eqclass{S}{O_{1}}{O_{2}}}{\eqclass{S'}{O_{1}'}{O_{2}'}}}

  \end{mathpar}

  \begin{mathpar}
    
    \inferrule[lam-eq]
    {\eqcls{\Gamma}{K_{1}}{K_{1}'} \\ \eqobj{\snocctx{\Gamma}{X}{K_{1}}}{O_{2}}{O_{2}'}{K_{2}}}
    {\eqobj{\Gamma}{\lamobj{K_{1}}{X}{O_{2}}}{\lamobj{K_{1}'}{X}{O_{2}'}}{\piclass{K_{1}}{X}{K_{2}}}}

    \inferrule[app-eq]
    {\eqobj{\Gamma}{O}{O'}{\piclass{K_{1}}{X}{K_{2}}} \\ \eqobj{\Gamma}{O_{1}}{O_{1}'}{K_{1}}}
    {\eqobj{\Gamma}{\appobj{O}{O_{1}}}{\appobj{O'}{O_{1}'}}{\subst{O_{1}}{X}{K_{2}}}}

  \end{mathpar}

  \begin{mathpar}
    
    \inferrule[app-lam]
    {\isobj{\snocctx{\Gamma}{X}{K_{1}}}{O_{2}}{K_{2}} \\ \isobj{\Gamma}{O_{1}}{K_{1}}}
    {\eqobj{\Gamma}{\appobj{(\lamobj{K_{1}}{X}{O_{2}})}{O_{1}}}{\subst{O_{1}}{X}{O_{2}}}{\subst{O_{1}}{X}{K_{2}}}}

    \inferrule[lam-app]
    {\isobj{\Gamma}{O}{\piclass{K_{1}}{X}{K_{2}}}}
    {\eqobj{\Gamma}{O}{\lamobj{K_{1}}{X}{(\appobj{O}{X})}}{\piclass{K_{1}}{X}{K_{2}}}}

  \end{mathpar}

  \begin{mathpar}
    
    \inferrule[reflection]
    {\isobj{\Gamma}{O}{\eqclass{S}{O_{1}}{O_{2}}}}
    {\eqobj{\Gamma}{O_{1}}{O_{2}}{S}}

    \inferrule[unicity]
    {\isobj{\Gamma}{O}{\eqclass{S}{O_{1}}{O_{2}}} \\ \isobj{\Gamma}{O'}{\eqclass{S}{O_{1}}{O_{2}}}}
    {\eqobj{\Gamma}{O}{O'}{\eqclass{S}{O_{1}}{O_{2}}}}

  \end{mathpar}

  \caption{Equality Judgments}
  \label{fig:lf-eq}
\end{figure}

\section{Two Type Theories}

The benefit of a logical framework is that it permits the concise specification of type
theories, or other logical systems, as a signature.

\subsection{G\"odel's T}

\newcommand{\tpsort}{\const{tp}}
\newcommand{\nattp}{\const{nat}}
\newcommand{\arrtp}{\const{arr}}
\newcommand{\arrof}[2]{\appobj{\appobj{\arrtp}{#1}}{#2}}

\newcommand{\elfam}{\const{el}}
\newcommand{\elof}[1]{\appobj{\elfam}{#1}}

\newcommand{\zerocon}{\const{zero}}
\newcommand{\succcon}{\const{succ}}
\newcommand{\succof}[1]{\appobj{\succcon}{#1}}
\newcommand{\reccon}{\const{rec}}
\newcommand{\recof}[4]{\appobj{\appobj{\appobj{\appobj{\reccon}{#1}}{#2}}{#3}}{#4}}

\newcommand{\lamcon}{\const{lam}}
\newcommand{\lamof}[3]{\appobj{\appobj{\appobj{\lamcon}{#1}}{#2}}{#3}}
\newcommand{\appcon}{\const{app}}
\newcommand{\appof}[4]{\appobj{\appobj{\appobj{\appobj{\appcon}{#1}}{#2}}{#3}}{#4}}

\newcommand{\reczcon}{\const{nat-$\beta$-z}}
\newcommand{\recscon}{\const{nat-$\beta$-s}}
\newcommand{\betacon}{\const{arr-$\beta$}}
\newcommand{\etacon}{\const{arr-$\eta$}}

The signature $\Sigma_{T}$ defining G\"{o}del's System T is given in Figure~\ref{fig:t-sig}.
The specifications are verbose, but an implementation would eliminate much of the
redundancy.  In particular the outermost quantifiers on the classes of constants can be
safely omitted when their class is evident.  For example, it would be sensible to drop the
quantification over objects of sort $\tpsort$ in Figure~\ref{fig:t-sig}.  For example, the
declaration of $\lamcon$ could be abbreviated to
\begin{align*}
  \lamcon & : \arrclass{\parens{\arrclass{\elof{A_{1}}}{\elof{A_{2}}}}}{\elof{\parens{\arrof{A_{1}}{A_{2}}}}},
\end{align*}
with the outermost quantification over $A_{1}$ and $A_{2}$ of sort $\tpsort$ being
understood.  Correspondinly, it would be sensible to write $\lamof{\_{}}{\_{}}{F}$ when
the class of $F$ is given as $\arrclass{\elof{A_{1}}}{\elof{A_{2}}}$, the omitted
arguments being inferrable from context.  An abbreviated form of $\Sigma_{T}$ is given in
Figure~\ref{fig:t-sig-abbr} for comparison.

\begin{figure}
  
  \begin{align*}
    \tpsort
    & : \sortclass \\
    \nattp
    & : \tpsort \\
    \arrtp
    & : \arrclass{\tpsort}{\arrclass{\tpsort}{\tpsort}} \\[1ex]
    \elfam
%
    & : \arrclass{\tpsort}{\sortclass} \\
    \zerocon
    & : \elof{\nattp} \\
    \succcon
    & : \arrclass {\elof{\nattp}}{\elof{\nattp}} \\
    \reccon
    &  :
      \piclass{\tpsort}{A}{
      \arrclass{\elof{A}}{
      \arrclass{\parens{\arrclass{\elof{\nattp}}{\arrclass{\elof{A}}{\elof{A}}}}}{
      \elof{A}}}} \\[1ex] 
%
    \lamcon
    & :
      \piclass{\tpsort}{A_{1}}{
      \piclass{\tpsort}{A_{2}}{
      \arrclass{\parens{\arrclass{\elof{A_{1}}}{\elof{A_{2}}}}}{
      \elfam{\parens{\arrof{A_{1}}{A_{2}}}}}}} \\
    \appcon
    & :
      \piclass{\tpsort}{A_{1}}{
      \piclass{\tpsort}{A_{2}}{
      \arrclass{\elof{\parens{\arrof{A_{1}}{A_{2}}}}}{
      \arrclass{\elof{A_{1}}}{\elof{A_{2}}}}}} \\[1ex]
%
    \reczcon
    & :
      \piclass{\tpsort}{A}{
      \piclass{\elof{A}}{b}{
      \piclass{\arrclass{\elof{\nattp}}{\arrclass{\elof{A}}{\elof{A}}}}{s}{
      \eqclass{\elof{A}}{\recof{A}{b}{s}{\zerocon}}{b}}}} \\
    \recscon
    & :
      \piclass{\tpsort}{A}{
      \piclass{\elof{A}}{b}{
      \piclass{\arrclass{\elof{\nattp}}{\arrclass{\elof{A}}{\elof{A}}}}{s}{
      \piclass{\elof{\nattp}}{n}{
      \eqclass{\elof{A}}
      {\recof{A}{b}{s}{\parens{\succof{n}}}}{\appobj{\appobj{s}{n}}{\parens{\recof{A}{b}{s}{n}}}}}}}} \\[1ex]
%
    \betacon
    & :
      \piclass{\tpsort}{A_{1}}{
      \piclass{\tpsort}{A_{2}}{
      \piclass{\arrclass{\elof{A_{1}}}{\elof{A_{2}}}}{F}{
      \piclass{\elof{A_{1}}}{M}{
      \eqclass{\elof{A_{2}}}
      {\appof{A_{1}}{A_{2}}{\parens{\lamof{A_{1}}{A_{2}}{F}}}{M}}
      {\appobj{F}{M}}}}}} \\
    \etacon
    & :
      \piclass{\tpsort}{A_{1}}{
      \piclass{\tpsort}{A_{2}}{
      \piclass{\elof{\parens{\arrof{A_{1}}{A_{2}}}}}{M}{
      \eqclass{\elof{\parens{\arrof{A_{1}}{A_{2}}}}}
      {M}
      {\lamof{A_{1}}{A_{2}}{\parens{\lamobj{\elof{A_{1}}}{x}{\appof{A_{1}}{A_{2}}{M}{x}}}}}}}}
  \end{align*}

  \caption{Signature of G\"odel's T}
  \label{fig:t-sig}
\end{figure}

\begin{figure}
  
  \begin{align*}
    \tpsort
    & : \sortclass \\
    \nattp
    & : \tpsort \\
    \arrtp
    & : \arrclass{\tpsort}{\arrclass{\tpsort}{\tpsort}} \\[1ex]
    \elfam
%
    & : \arrclass{\tpsort}{\sortclass} \\
    \zerocon
    & : \elof{\nattp} \\
    \succcon
    & : \arrclass {\elof{\nattp}}{\elof{\nattp}} \\
    \reccon
    &  :
      \arrclass{\elof{A}}{
      \arrclass{\parens{\arrclass{\elof{\nattp}}{\arrclass{\elof{A}}{\elof{A}}}}}{
      \elof{A}}} \\[1ex] 
%
    \lamcon
    & :
      \arrclass{\parens{\arrclass{\elof{A_{1}}}{\elof{A_{2}}}}}{
      \elfam{\parens{\arrof{A_{1}}{A_{2}}}}} \\
    \appcon
    & :
      \arrclass{\elof{\parens{\arrof{A_{1}}{A_{2}}}}}{
      \arrclass{\elof{A_{1}}}{\elof{A_{2}}}} \\[1ex]
%
    \reczcon
    & :
      \piclass{\elof{A}}{b}{
      \piclass{\arrclass{\elof{\nattp}}{\arrclass{\elof{A}}{\elof{A}}}}{s}{
      \eqclass{\elof{A}}{\recof{\_{}}{b}{s}{\zerocon}}{b}}} \\
    \recscon
    & :
      \piclass{\elof{A}}{b}{
      \piclass{\arrclass{\elof{\nattp}}{\arrclass{\elof{A}}{\elof{A}}}}{s}{
      \piclass{\elof{\nattp}}{n}{
      \eqclass{\elof{A}}
      {\recof{\_{}}{b}{s}{\parens{\succof{n}}}}{\appobj{\appobj{s}{n}}{\parens{\recof{\_{}}{b}{s}{n}}}}}}} \\[1ex]
%
    \betacon
    & :
      \piclass{\arrclass{\elof{A_{1}}}{\elof{A_{2}}}}{F}{
      \piclass{\elof{A_{1}}}{M}{
      \eqclass{\elof{A_{2}}}
      {\appof{\_{}}{\_{}}{\parens{\lamof{\_{}}{\_{}}{F}}}{M}}
      {\appobj{F}{M}}}} \\
    \etacon
    & :
      \piclass{\elof{\parens{\arrof{A_{1}}{A_{2}}}}}{M}{
      \eqclass{\elof{\parens{\arrof{A_{1}}{A_{2}}}}}
      {M}
      {\lamof{\_{}}{\_{}}{\parens{\lamobj{\elof{A_{1}}}{x}{\appof{\_{}}{\_{}}{M}{x}}}}}}
  \end{align*}

  \caption{Signature of G\"odel's T, Abbreviated}
  \label{fig:t-sig-abbr}
\end{figure}

\subsection{Dependent Types}

\bibliographystyle{plainnat}
\bibliography{notes}

\end{document}

%%% Local Variables:
%%% mode: latex
%%% TeX-master: t
%%% fill-column: 90
%%% auto-fill-mode: t
%%% End:
