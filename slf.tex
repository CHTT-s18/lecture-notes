\documentclass[11pt,twoside]{article}
\usepackage[authoryear,semicolon]{natbib}
\usepackage[T1]{fontenc}
\usepackage[utf8]{inputenc}
\usepackage{textgreek}
\usepackage{fullpage}
\usepackage[color=yellow]{todonotes}
\setlength{\marginparwidth}{1.25in}
\usepackage{xifthen}
\usepackage{amsmath,amssymb,amsthm,mathtools,stmaryrd}
\usepackage{proof,mathpartir}
\usepackage{colonequals}
\usepackage{code,verbatim}
\usepackage{comment}
\usepackage{textcomp}
\usepackage[us]{optional}
\usepackage{color}
\usepackage{url}
\usepackage{graphics}
\usepackage{import}
\usepackage{stackengine}
\usepackage{scalerel}

\newcommand{\eqdef}{\mathrel{\triangleq}}
\newcommand{\isdef}{\eqdef}

\allowdisplaybreaks[1]       %mildly permissible to break up displayed equations

\begin{document}

\title{A Semantic Logical Framework}
\author{Robert Harper}
\date{\today}

\maketitle{}

\section{Introduction}

A \emph{logical framework} is language for defining logical systems, in particular type
theories.  The definition of a logical system consists of a collection of
\emph{generators} that populate a collection of classifying \emph{sorts}, and a collection
of \emph{equations} that govern the objects of those sorts.  The generators specify the
sorts of the objects that constitute the logical system---say, the sort of types and, for
each type, the sort of its elements---and in addition specify the objects of each of these
sorts.  These objects comprise the syntactic entities of the logical system---the types and
their elements---and are specified using \emph{higher-order abstract syntax} to express the
binding and scopes of variables.  The native equality of the framework is a congruence---an
equivalence relation compatible with the generators---and defines substitution of objects
for variables in another object.  The defining equations of a logical system enrich the
native equality to specify the behavior of the represented objects---such as the inversion
and unicity properties of type constructors or connectives.

The integration of the defining equations with the native equations of the framework
constrains the \emph{meaning} of the defined system within the framework in the sense that
any interpretation must obey the specified laws.  There being no limitations on the nature
of these equations, the enriched equality judgment of the framework may or may not be
(feasibly or infeasibly) decidable.  A \emph{syntactic logical
  framework}~\cite{harper-etal:lf} is one that presents a logical system using only
generators, and no relations, so that the induced equational theory is the native one,
which is decidable.  A \emph{semantic logical theory}~\cite{smith1990programming} admits
the specification of equations, with the consequence that the decidability of a defined
logical system presents a question that must be resolved for each case separately.

This note defines a semantic logical framework suitable for defining a broad---but by no
means comprehensive---class of logical systems, including full-scale dependent type
theories.  It is a dependently typed language with a single Russellian universe of sorts,
and extensional equality types governing the objects of a sort.  The definition of a
logical system is a form of context, called a \emph{signature}, that specifies generators
that populate the sorts and the equality types that govern them.  The \emph{adequacy} of a
signature expresses the intended correspondence between the components of the represented
logical system and their counterpart objects in the logical framework.

\section{The Meat}

\section{Conclusion}

\bibliographystyle{plainnat}
\bibliography{notes}

\end{document}

%%% Local Variables:
%%% mode: latex
%%% TeX-master: t
%%% fill-column: 90
%%% auto-fill-mode: t
%%% End:
