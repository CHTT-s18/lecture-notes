\documentclass{article}

\usepackage{amsthm}
\usepackage{amsmath}
\usepackage{amssymb}
\usepackage{natbib}
\usepackage{xcolor}
\usepackage{mathpartir}
\usepackage{url}

% Theorems
\newtheorem{thm}{Theorem}
\newtheorem{cor}[thm]{Corollary}
\newtheorem{lem}[thm]{Lemma}
\newtheorem{remark}[thm]{Remark}
\newtheorem{claim}[thm]{Attempt}
\newtheorem{defn}[thm]{Definition}
\newtheorem*{lemm}{Lemma}


\newenvironment{proofattempt}{\paragraph{\emph{Proof Attempt}.}}{\hfill\color{red}{X}\par}

\newcommand{\redprl}{{\color{red}\textbf{Red}}\textbf{PRL}}

\newcommand{\eeq}{\ensuremath{\doteq}}

\newcommand{\hasTF}[2]{\ensuremath{#1 \vdash #2\ \mathsf{type}}}
\newcommand{\hasEF}[3]{\ensuremath{#1 \vdash #2 : #3}}
\newcommand{\hasTEF}[3]{\ensuremath{#1 \vdash #2 \equiv #3\ \mathsf{type}}}
\newcommand{\hasEEF}[4]{\ensuremath{#1 \vdash #2 \equiv #3 : #4}}

\newcommand{\hasTC}[2]{\ensuremath{#1 \gg #2\ \mathsf{type}}}
\newcommand{\hasEC}[3]{\ensuremath{#1 \gg #2 \in #3}}
\newcommand{\hasTEC}[3]{\ensuremath{#1 \gg #2 \eeq #3\ \mathsf{type}}}
\newcommand{\hasEEC}[4]{\ensuremath{#1 \gg #2 \eeq #3 : #4}}

\newcommand{\hterm}[2]{\ensuremath{\mathsf{HT}_{#1}(#2)}}
\newcommand{\htermp}[2]{\ensuremath{\mathsf{HT}^+_{#1}(#2)}}
\newcommand{\termb}[1]{\ensuremath{{#1} \; \mathsf{term}_{\beta}}}
\newcommand{\hcbn}[2]{\ensuremath{\mathsf{HT}^{\mathsf{cbn}}_{#1}(#2)}}
\newcommand{\hcbv}[2]{\ensuremath{\mathsf{HT}^{\mathsf{cbv}}_{#1}(#2)}}

\newcommand{\contrb}[2]{#1 \mathrel{\text{contr}_\beta} #2}
\newcommand{\bnf}[1]{#1 \mathrel{\text{nf}_\beta}}
\newcommand{\bnorm}[1]{#1 \mathrel{\text{norm}_\beta}}
\newcommand{\stepb}[2]{\ensuremath{#1 \mapsto_{\beta} #2}}
\newcommand{\stepbs}[2]{\ensuremath{#1 \mapsto_{\beta}^* #2}}

\newcommand{\hnorm}[3]{\ensuremath{\mathsf{HN}^{#1}_{#2}(#3)}}
\newcommand{\id}[1]{\ensuremath{\mathsf{id}(#1)}}

\newcommand{\AND}{\mathrel{\wedge}}
\newcommand{\OR}{\mathrel{\vee}}

\newcommand{\step}[2]{\ensuremath{#1 \mapsto #2}}
\newcommand{\steps}[2]{\ensuremath{#1 \mapsto^* #2}}
\newcommand{\eval}[2]{\ensuremath{#1 \mathrel{\Downarrow} #2}}
\newcommand{\valueJ}[1]{\ensuremath{#1\ \mathsf{value}}}
\newcommand{\fillin}[2]{\ensuremath{#1\{#2\}}}

\newcommand{\fn}[2]{\ensuremath{#1 \to #2}}
\newcommand{\nat}{\ensuremath{\mathsf{nat}}}
\newcommand{\bool}{\ensuremath{\mathsf{bool}}}
\newcommand{\tp}{\ensuremath{\mathsf{T}}}
\newcommand{\ap}[2]{\ensuremath{#1\ #2}}
\newcommand{\lam}[3]{\ensuremath{\lambda #1 {:} #2.\, #3}}

% Page layout
\usepackage[paper=letterpaper,top=1in,bottom=1in,left=0.8in,right=0.8in,]{geometry}
\setlength{\parindent}{0mm}
\setlength{\parskip}{1.5mm}

\usepackage{fancyhdr}
\fancyhead[RO,LE]{\scshape 15-819: Computational Higher Type Theory}
\fancyhead[LO,RE]{\scshape Lecture Notes Week 1}
\pagestyle{fancy}

\usepackage{hyperref}

% titles
\usepackage{titlesec}
\titleformat*{\section}{\large\bfseries}
\titleformat*{\subsection}{\normalsize\bfseries}
\titleformat*{\subsubsection}{\bfseries}
\titleformat*{\paragraph}{\bfseries}

% package to customize three basic list environments: enumerate, itemize and description.
\usepackage{enumitem}
\setitemize{noitemsep, topsep=0pt, leftmargin=*}
\setenumerate{noitemsep, topsep=0pt, leftmargin=*}
\setdescription{noitemsep, topsep=0pt, leftmargin=*}

\begin{document}

\begin{center}
\Large{\scshape Computational Higher Type Theory (CHTT)}\\[2pt]
\large{\scshape Robert Harper}\\[4pt]
\large\bfseries{Lecture Notes of Week 2 by Stephanie Balzer and Yue Niu}
\end{center}

\bigskip

\section{Notation \& Historical Notes}
The following proof method is attributed to William Tait, who first proved normalization for the STLC. Below are some 
notations you might find in the literature:

\begin{enumerate}
\item hereditarily \dots\\
\item logical relations (Statman)\\
\item Tait's Method (W. Tait)\\
\item Girard's Method (J-Y Girard)\\
\item Computability (Tait's Method)\\
\item Reducibility Method (Girard)
\end{enumerate}

\section{Fundamental Theorem Cont'd}

\subsubsection{Head expansion}

To complete the proof from last time (that well-typed terms are hereditarily terminating under substitution by 
hereditarily terminating maps), we need to prove a lemma known as ``head expansion'' or ``reverse execution'':

\begin{lem}[Head expansion]
  If $\hterm{A}{M'}$ and $\step{M}{M'}$ then $\hterm{A}{M}$.
\end{lem}
\begin{proof}
  Induction on $A$.
  \begin{itemize}
  \setlength\itemsep{1em}
  \item $A = b$\\
    Suppose $\hterm{b}{M'}$ and $\step{M}{M'}$. We need to show $\hterm{b}{M}$. By definition, it suffices to show that $M \Downarrow$. 
    This is clear since $M' \Downarrow$ and $\step{M}{M'}$.
  \item $A = \fn{A_1}{A_2}$\\
    Suppose $\hterm{\fn{A_1}{A_2}}{M'}$ and $\step{M}{M'}$. We need to show $\hterm{\fn{A_1}{A_2}}{M}$.
    Now suppose $\hterm{A_1}{N}$, and we need to show $\hterm{A_2}{\ap{M}{N}}$. 
    By the definition of HT and our assumption, we have $\hterm{A_2}{\ap{M'}{N}}$. 
    Note that $\step{\ap{M}{N}}{\ap{M'}{N}}$ by our assumption and the compuation rule.
    Now applying the IH, we have $\hterm{A_2}{\ap{M}{N}}$.
    \footnote{Terms in the hereditary termination predicate get bigger, but types get smaller. Since the induction 
    is on $A$, we are okay.}
    \qedhere
  \end{itemize}
\end{proof}

\subsubsection{Termination of STLC}

Now we need to show that hereditary termination implies termination of terms in the STLC:
\begin{thm}
  If $\hterm{A}{M}$ then $\termb{M}$.
\end{thm}
\begin{proof}
  Induction on $A$.
   \begin{itemize}
  \setlength\itemsep{1em}
  \item $A = b$\\
    Suppose $\hterm{b}{M}$. Done by definition of hereditary termination.  
  \item $A = \fn{A_1}{A_2}$\\
    Suppose $\hterm{\fn{A_1}{A_2}}{M}$. We need to show $\termb{M}$. We are stuck here; and there are a couple of
    options to consider.
  \end{itemize}
\end{proof}

\begin{enumerate}
\item Weaken the theorem so that we can only conclude termination for observables (terms in the base type).
\item Add $\termb{M}$ to the definition of $\hterm{A}{M}$; while this comes off as a hack, everything so far will 
still work out.
\end{enumerate}

\subsection{Another Hereditary Termination (Positive Formulation)}

Notice in the definition of hereditary termination above, we take a ``negative'' approach with respect to the 
function type, in that we consider everything that can happen in the elimination form for functions. Dually, we can 
also formulate hereditary termination ``positively''. 

\begin{align*}
  \htermp{b}{M} &\triangleq \steps{M}{c}\\
  \htermp{\fn{A}{B}}{M} &\triangleq \steps{M}{\lam{x}{A}{M_2}} \; \text{and}\;
  \forall M_1.\ \htermp{A}{M_1} \implies \htermp{A_2}{[M_1/x]M_2}
\end{align*}

This is also known as the ``method of canonical forms'', and we can define hereditary termination in terms of 
``hereditary values'', whose definition will depend on whether we are working ``call by value'' or ``call by name''. 

\subsection{C.B.V vs. C.B.N}

In both C.B.V and C.B.N, a hereditary value at base type is simply a value. 
At the function type for C.B.N, we have
\[
\hcbn{\fn{A_1}{A_2}}{V} \triangleq V = \lam{x}{A}{M_2} \; \text{and} \; \forall M_1. \hterm{A_1}{M_1} \implies
\hterm{A_2}{[M_1/x]M_2}
\]
For C.B.V:
\[
\hcbv{\fn{A_1}{A_2}}{V} \triangleq V = \lam{x}{A}{M_2} \; \text{and} \; \forall V_1. \hcbv{A_1}{V_1} \implies
\hterm{A_2}{[V_1/x]M_2}
\]

TODO: fill in explanation for distinction. 

\subsection{Summary}

We we have shown so far can be summarized as a relationship between syntax and semantics: well-typed terms 
terminate according to the semantics (the computation rules). We can make this more apparent by introducing 
some new notation: 

\begin{align}
&\hasTC{\Gamma}{A} && \\
&M \in A && \hterm{A}{M}\\
&\hasEC{\Gamma}{M}{A} && \gamma \in \Gamma \implies \hat\gamma(M) \in A &\\
\end{align}

Notice that the ``membership'' relation has a computation flavor; it is a behavorial condition on $M$: which says 
that $M$ satisfy the specification $A$. \\

Now, we can state our theorem as follows: 

\begin{thm}
If $\hasEF{\Gamma}{M}{A}$, then $\hasEC{\Gamma}{M}{A}$.
\end{thm}

This is in some sense a soundness theorem (in the language of formal logics), which means that the formally derivable
terms are actually true (according to the computational specification). Therefore, we can think of formal systems as a 
way of \emph{accessing} the \emph{truth}. Note that we make no claims that $\hasEC{\Gamma}{M}{A}$ be decidable, 
as it is fruitless to expect the truth to be decidable in general. \footnote{see dreyer's Milner lectures for 
further discussion.}

\section{Normalization}

We can also interpret variables in a different sense, where they stand for open terms - they are indeterminates. 
Then the judgment $\hasEF{\Gamma}{M}{A}$ could be interpreted as a mapping $M$ from open terms to the type $A$.\\

Recall the beta-contraction relation, $\contrb{P}{P'}$:
\[
\contrb{\ap{(\lam{x}{A}{M})}{N}}{[N/x]M}
\]

With this, we formulate beta-reduction: 

\begin{mathpar}
\inferrule{
  \contrb{M}{N}
}{
  \stepb{M}{N}
} \and
\inferrule{
  \stepb{M}{M'}
}{
  \stepb{\ap{M}{N}}{\ap{M'}{N}}
} \and
\inferrule{
  \stepb{N}{N'}
}{
  \stepb{\ap{M}{N}}{\ap{M}{N'}}
} \and
\inferrule{
  \stepb{M}{M'}
}{
  \stepb{\lam{x}{A}{M}}{\lam{x}{A}{M'}}
} 
\end{mathpar}

Further, we define beta-normal form, $\bnf{N}$:
\[
\bnf{N} \triangleq N \not\mapsto_{\beta}
\]

Where $N \not\mapsto_{\beta}$ means that no beta-reduction rule applies to $N$. This informal notation can be formalized.

Finally, we can define when a term is beta-normalizing, $\bnorm{M}$: 
\[
\bnorm{M} \triangleq \exists N. \stepbs{M}{N} \;\text{and}\; \bnf{N}
\]

Now we can state the fundamental theorem for normalization: 
\begin{thm}
If $\hasEF{\Gamma}{M}{A}$ then $\bnorm{M}$.
\end{thm}

Following the proof for termination, we introduce a stronger notion for normalization, \emph{hereditary normalization}:

\begin{align*}
  \hnorm{\Delta}{b}{M} &\triangleq \bnorm{M}\\
  \hnorm{\Delta}{\fn{A}{B}}{M} &\triangleq
  \forall N.\ \hnorm{\Delta}{A}{N} \implies \hnorm{\Delta}{B}{\ap{M}{N}}
\end{align*}
As before, we define $\hnorm{\Delta}{\Gamma}{\gamma}$ to be $\forall x : A \in \Gamma.\ \hnorm{\Delta}{A}{\gamma(x)}$

Some notes about this definition:
\begin{itemize}
\setlength\itemsep{1em}
\item This predicate is defined on terms $M$ s.t. $\hasEF{\Delta}{M}{A}$. The extra argument to the hereditary normalization (compared to termination) $\Delta$ reflects this fact, since normalization is defined on open terms. \\
\item We formulate hereditary normalization in the negative way; turns out this is the only way\\
\item What can we say about positive types (sums)?
\end{itemize}

The proof can be divided into three lemmas: 
\begin{lem}\label{l1}
If $\hasEF{\Gamma}{M}{A}$ and $\hnorm{\Delta}{\Gamma}{\gamma}$, then $\hnorm{\Delta}{A}{\hat\gamma(M)}$.
\end{lem}

\begin{lem}\label{l2}
If $\hnorm{\Delta}{A}{M}$ then $\bnorm{M}$.
\end{lem}

\begin{lem}\label{l3}
$\hnorm{\Gamma}{\Gamma}{\id{\Gamma}}$.
\end{lem}

\subsection{Lemma \ref{l1}}

\begin{lemm}
If $\hasEF{\Gamma}{M}{A}$ and $\hnorm{\Delta}{\Gamma}{\gamma}$, then $\hnorm{\Delta}{A}{\hat\gamma(M)}$.
\end{lemm}

\begin{proof}
Induction on typing. 
\begin{itemize}
  \setlength\itemsep{1em}
  \item $\hasEF{\Gamma', x : A}{x}{A}$\\
    Suppose $\hnorm{\Delta}{\Gamma}{\gamma}$. 
    We need to  show $\hnorm{\Delta}{A}{\hat{\gamma}(x)}$, which follows directly by our assumption.
  \item $\hasEF{\Gamma}{c}{b}$\\
    Suppose $\hnorm{\Delta}{\Gamma}{\gamma}$. We need to show that $\hnorm{\Delta}{b}{c}$, which is to 
    show that $\bnorm{c}$. Since $\bnf{c}$, we have $c$ normalizing in 0 steps. 
  \item $\inferrule{
      \hasEF{\Gamma}{M}{\fn{A}{B}}\\
      \hasEF{\Gamma}{N}{A}
    }{\hasEF{\Gamma}{\ap{M}{N}}{B}}$\\
    Applying the IH, we have $\hnorm{\Delta}{\fn{A}{B}}{\hat{\gamma}(M)}$
    and $\hnorm{\Delta}{A}{\hat{\gamma}(N)}$. By definition of hereditary normalization, we have
    $\hnorm{\Delta}{B}{\ap{\hat{\gamma}(M)}{\hat{\gamma}(N)}}$. Since $\hat{\gamma}(\ap{M}{N}) = 
    \ap{\hat{\gamma}(M)}{\hat{\gamma}(N)}$, we have
    $\hnorm{\Delta}{B}{\hat{\gamma}(\ap{M}{N})}$.
  \item
    $\inferrule{
      \hasEF{\Gamma, x : A}{M}{B}\\
    }{\hasEF{\Gamma}{\lam{x}{A}{M}}{\fn{A}{B}}}$\\
    Suppose $\hnorm{\Delta}{\Gamma}{\gamma}$. We need to show that 
    $\hnorm{\Delta}{\fn{A}{B}}{\hat{\gamma}(\lam{x}{A}{M})}$. 
    Thus, suppose $\hnorm{\Delta}{A}{N}$. It suffices to show that 
    $\hnorm{\Delta}{B}{\ap{\hat{\gamma}(\lam{x}{A}{M})}{N}}$. Notice that with $N$, we can extend $\gamma$ to be
    $\gamma' = \gamma[x \mapsto N]$, and since $N$ is hereditarily normalizing in $A$, it follows that 
    $\hnorm{\Delta}{\Gamma,x : A}{\gamma'}$. Applying the IH, we have that $\hnorm{\Delta}{B}{\hat{\gamma'}(M)}$. 
    Since substitution is commutative, $\hat{\gamma'}(M) = [N/x]\hat{\gamma}(M)$, and 
    $\ap{\hat{\gamma}(\lam{x}{A}{M})}{N} = \ap{\lam{x}{A}{\hat{\gamma}(M)}}{N}$. Further, 
    $\step{\ap{\lam{x}{A}{\hat{\gamma}(M)}}{N}}{[N/x]\hat{\gamma}(M)}$ by the computation rule. The result then follows
    from head expansion. 
   \qedhere
\end{itemize}
\end{proof}

The proof for the next 2 lemmas will be mutual induction. However, this is not enough. We will need to further
strengthen Lemma \ref{l3}, but let's see where we fail.

\subsection{Lemma \ref{l2}}

\begin{lemm}
If $\hnorm{\Delta}{A}{M}$ then $\bnorm{M}$.
\end{lemm}

\begin{proofattempt}
Induction on $A$.
\begin{itemize}
  \setlength\itemsep{1em}
  \item $A = b$\\
  Immediate from definition of hereditary normalization.
  \item $A = \fn{A_1}{A_2}$\\
  Suppose $\hnorm{\Delta}{A}{M}$. We need to show $\bnorm{M}$. Now we are in a bind. If we can somehow use 
  Lemma \ref{l3} to obtain a hereditarily normalizing input of type $A_1$, then we can proceed with induction 
  to obtain the result. For now, we will assume (*) that gives us a $\Gamma'$ such that $x : A_1 \in \Gamma'$.
  By Lemma (\ref{l3}) we have $\hnorm{\Gamma'}{\Gamma'}{\id{\Gamma'}}$, in particular, $\hnorm{\Gamma'}{A_1}{x}$. By 
  the definition of hereditary normalization, we further have that $\hnorm{\Gamma'}{A_2}{\ap{M}{x}}$. Now applying the 
  IH, we see that $\bnorm{\ap{M}{x}}$, and by Lemma \ref{l5} we have $\bnorm{M}$. 
  \qedhere
\end{itemize}
\end{proofattempt}

Postponing the issues with this proof, we will move on to proving Lemma \ref{l3}, and resolve everything after
presenting the two proofs.

\subsection{Lemma \ref{l3}}

\begin{lemm}
$\hnorm{\Gamma}{\Gamma}{\id{\Gamma}}$.
\end{lemm}

\begin{proofattempt}
Let $x : A \in \Gamma$ be arbitrary. Proceed with induction on $A$.
\begin{itemize}
  \setlength\itemsep{1em}
  \item $A = b$\\
  We need to show $\hnorm{\Gamma}{b}{x}$, which holds since $\bnf{x}$.
  \item $A = \fn{A_1}{A_2}$\\
  We need to show $\hnorm{\Gamma}{\fn{A_1}{A_2}}{x}$. Suppose $\hnorm{\Gamma}{A_1}{M_1}$, and it suffices to show
  that $\hnorm{\Gamma}{A_2}{\ap{x}{M_1}}$. Now it would be nice to apply the IH to complete the proof, but which
  is not strong enough.
\end{itemize}
\end{proofattempt}

Notice that for every variable of function type, we want them to be hereditarily normalizing when applied to 
beta-normal terms. We will add this to the lemma:

\begin{lem}\label{l4}
For all $k$, if $x : A_1 \to \dots \to A_k \to A \in \Gamma$ and for each $i \le k$, $\hasEF{\Gamma}{M_i}{A_i}$ and $\bnorm{M_i}$, then $\hnorm{\Gamma}{A}{\ap{\ap{x}{M_1}}{\dots M_k}}$.
\end{lem}

Note that we need each argument to be \emph{normalizing} instead of merely \emph{hereditarily normalizing}. 
Before proving \ref{l4}, we shall see how to fix \ref{l2}. The crucial fact is the (*) that gave us a context $\Gamma'$
out of thin air, which allowed us to supply the argument to the function term. To do this, we will make a change to 
the definition of hereditary normalization, which will involve the notion of Kripke semantics.

\subsection{Kripke semantics}

Kripke semantics, or presheafs, are related to/gives the solution to the following:
\begin{itemize}
\item allocating a ``fresh variable'' in a context
\item why should hereditary normalization be stable under context extensions? (In other words, why should context
extensions be admissible for hereditary normalization?)
\end{itemize}

If we think of the typing context as a model of the ``world'', then Kripke semantics suggests that we can stipulate
hereditary normalization to hold in all context extensions, or ``future words''. Viewed logically, this would be the
admissibility of weakening with respect to hereditary normalization.\\

Thus for our definition, we change the higher order terms to expect ``world extensions'':

\[
\hnorm{\Delta}{\fn{A_1}{A_2}}{M} \triangleq \;\text{if}\; \Delta' \ge \Delta \;\text{and}\; \hnorm{\Delta'}{A_1}{M_1}
\;\text{then}\; \hnorm{\Delta'}{A_2}{\ap{M}{M_1}}
\]

Note that $\ge$ is a pre-order on contexts $\Delta$, and is reflexive and transitive. Now we can fix both lemmas. 

\subsection{Fix Lemmas}

\begin{lemm}
If $\hnorm{\Delta}{A}{M}$ then $\bnorm{M}$.
\end{lemm}

\begin{proof}
Induction on $A$.
\begin{itemize}
  \setlength\itemsep{1em}
  \item $A = b$\\
  Immediate from definition of hereditary normalization.
  \item $A = \fn{A_1}{A_2}$\\
  Suppose $\hnorm{\Delta}{A}{M}$. We need to show $\bnorm{M}$. Let $\Delta' = \Delta, x : A_1$. By IH on Lemma \ref{l4} 
  (with $k = 0$), we have $\hnorm{\Delta'}{A_1}{x}$. By the definition of hereditary termination, we have 
  $\hnorm{\Delta'}{A_2}{\ap{M}{x}}$. Applying the IH, we have $\bnorm{\ap{M}{x}}$, and by Lemma \ref{l5} we have 
  $\bnorm{M}$. 
  \qedhere
\end{itemize}
\end{proof}

\begin{lem}\label{l4}
For all $k$, if $x : A_1 \to \dots \to A_k \to A \in \Gamma$ and for each $i \le k$, $\hasEF{\Gamma}{M_i}{A_i}$ and $\bnorm{M_i}$, then $\hnorm{\Gamma}{A}{\ap{\ap{x}{M_1}}{\dots M_k}}$.
\end{lem}

\begin{proof}
Proceed with induction on $A$.
\begin{itemize}
  \setlength\itemsep{1em}
  \item $A = b$\\
  We need to show $\bnorm{\ap{\ap{x}{M_1}}{\dots M_k}}$, which holds since each $M_i$ normalizing, and thus the 
  term does not beta-reduce.
  \item $A = \fn{A_i}{A_j}$\\
  We need to show $\hnorm{\Gamma}{A}{\ap{\ap{x}{M_1}}{\dots M_k}}$. Let $\Gamma' \ge \Gamma$ and 
  $\hnorm{\Gamma'}{A_i}{M_i}$. It suffices to show that $\hnorm{\Gamma'}{A_j}{\ap{\ap{\ap{x}{M_1}}{\dots M_k}}{M_i}}$.
  By IH on Lemma \ref{l2}, we have that $\bnorm{M_i}$, and we obtain the result by applying the IH with $k + 1$.
\end{itemize}
\end{proof}

\section{Strong Normalization}
\bibliographystyle{plainnat}
\bibliography{ctt}

\end{document}
