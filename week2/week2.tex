\documentclass{article}

\usepackage{amsthm}
\usepackage{amsmath}
\usepackage{amssymb}
\usepackage{natbib}
\usepackage{xcolor}
\usepackage{mathpartir}
\usepackage{url}

% Theorems
\newtheorem{thm}{Theorem}
\newtheorem{cor}[thm]{Corollary}
\newtheorem{lem}[thm]{Lemma}
\newtheorem{remark}[thm]{Remark}
\newtheorem{claim}[thm]{Attempt}
\newtheorem{defn}[thm]{Definition}
\newtheorem*{lemm}{Lemma}


\newenvironment{proofattempt}{\paragraph{\emph{Proof Attempt}.}}{\hfill\color{red}{X}\par}

\newcommand{\redprl}{{\color{red}\textbf{Red}}\textbf{PRL}}

\newcommand{\eeq}{\ensuremath{\doteq}}

\newcommand{\hasTF}[2]{\ensuremath{#1 \vdash #2\ \mathsf{type}}}
\newcommand{\hasEF}[3]{\ensuremath{#1 \vdash #2 : #3}}
\newcommand{\hasTEF}[3]{\ensuremath{#1 \vdash #2 \equiv #3\ \mathsf{type}}}
\newcommand{\hasEEF}[4]{\ensuremath{#1 \vdash #2 \equiv #3 : #4}}

\newcommand{\hasTC}[2]{\ensuremath{#1 \gg #2\ \mathsf{type}}}
\newcommand{\hasEC}[3]{\ensuremath{#1 \gg #2 \in #3}}
\newcommand{\hasTEC}[3]{\ensuremath{#1 \gg #2 \eeq #3\ \mathsf{type}}}
\newcommand{\hasEEC}[4]{\ensuremath{#1 \gg #2 \eeq #3 : #4}}

\newcommand{\hterm}[2]{\ensuremath{\mathsf{HT}_{#1}(#2)}}
\newcommand{\htermp}[2]{\ensuremath{\mathsf{HT}^+_{#1}(#2)}}
\newcommand{\termb}[1]{\ensuremath{{#1} \; \mathsf{term}_{\beta}}}
\newcommand{\hv}[2]{\ensuremath{\mathsf{HV}_{#1}(#2)}}
\newcommand{\hvcbn}[2]{\ensuremath{\mathsf{HV}^{\mathsf{cbn}}_{#1}(#2)}}
\newcommand{\hvcbv}[2]{\ensuremath{\mathsf{HV}^{\mathsf{cbv}}_{#1}(#2)}}

\newcommand{\contrb}[2]{#1 \mathrel{\text{contr}_\beta} #2}
\newcommand{\bnf}[1]{#1 \mathrel{\text{nf}_\beta}}
\newcommand{\bnorm}[1]{#1 \mathrel{\text{norm}_\beta}}
\newcommand{\stepb}[2]{\ensuremath{#1 \mapsto_{\beta} #2}}
\newcommand{\stepbs}[2]{\ensuremath{#1 \mapsto_{\beta}^* #2}}

\newcommand{\hnorm}[3]{\ensuremath{\mathsf{HN}^{#1}_{#2}(#3)}}
\newcommand{\id}[1]{\ensuremath{\mathsf{id}(#1)}}

\newcommand{\AND}{\mathrel{\wedge}}
\newcommand{\OR}{\mathrel{\vee}}

\newcommand{\step}[2]{\ensuremath{#1 \mapsto #2}}
\newcommand{\steps}[2]{\ensuremath{#1 \mapsto^* #2}}
\newcommand{\eval}[2]{\ensuremath{#1 \mathrel{\Downarrow} #2}}
\newcommand{\valueJ}[1]{\ensuremath{#1\ \mathsf{value}}}
\newcommand{\fillin}[2]{\ensuremath{#1\{#2\}}}

\newcommand{\fn}[2]{\ensuremath{#1 \to #2}}
\newcommand{\nat}{\ensuremath{\mathsf{nat}}}
\newcommand{\bool}{\ensuremath{\mathsf{bool}}}
\newcommand{\tp}{\ensuremath{\mathsf{T}}}
\newcommand{\ap}[2]{\ensuremath{#1\ #2}}
\newcommand{\lam}[3]{\ensuremath{\lambda #1 {:} #2.\, #3}}

% Page layout
\usepackage[paper=letterpaper,top=1in,bottom=1in,left=0.8in,right=0.8in,]{geometry}
\setlength{\parindent}{0mm}
\setlength{\parskip}{1.5mm}

\usepackage{fancyhdr}
\fancyhead[RO,LE]{\scshape 15-819: Computational Higher Type Theory}
\fancyhead[LO,RE]{\scshape Lecture Notes Week 2}
\pagestyle{fancy}

\usepackage{hyperref}

% titles
\usepackage{titlesec}
\titleformat*{\section}{\large\bfseries}
\titleformat*{\subsection}{\normalsize\bfseries}
\titleformat*{\subsubsection}{\bfseries}
\titleformat*{\paragraph}{\bfseries}

% package to customize three basic list environments: enumerate, itemize and description.
\usepackage{enumitem}
\setitemize{noitemsep, topsep=0pt, leftmargin=*}
\setenumerate{noitemsep, topsep=0pt, leftmargin=*}
\setdescription{noitemsep, topsep=0pt, leftmargin=*}

\begin{document}

\begin{center}
\Large{\scshape Computational Higher Type Theory (CHTT)}\\[2pt]
\large{\scshape Robert Harper}\\[4pt]
\large\bfseries{Lecture Notes of Week 2 by Stephanie Balzer and Yue Niu}
\end{center}

\bigskip

\section{Setting the Scene}

Last week, we explored \emph{closed term computation} for the simply typed lambda calculus
(STLC) with the goal to prove that all well-typed STLC programs terminate.  Our exploration
lead us to redescover Tait's~\citeyear{Tait:67} \emph{hereditary termination} $\hterm{A}{M}$ of
a term $M$ at type $A$ and \emph{hereditary terminating substitution} $\hterm{\Gamma}{\gamma}$
of closing substitution mapping $\gamma$ at context $\Gamma$ to strengthen our inductive
hypothesis.  The predicates $\hterm{A}{M}$ and $\hterm{\Gamma}{\gamma}$ amount to
\emph{behavioral} invariants, indexed by a type and typing context, respectively.  Depending on
the source, different terms are used in the literature to refer to methods based on such
behavioral invariants.  Most commonly used is the term \emph{hereditary}.  Tait himself used
the term \emph{computability}, whereas some people refer to such methods as \emph{Tait's
  Method}.  Statman~\citeyear{Statman:85} introduced the term \emph{logical relation}, which
underlines the idea of a behavioral invariant.  Moreover, the term \emph{Girard's method} can
be found in the literature as well as the term \emph{Reducibility Method}, which is most likely
due to Girard himself.

In this week's lectures, we are going to generalize the results of the previous week to
\emph{open term computation}.  Before doing so, however, we first have to finish our proof of
hereditary termination.

\section{Closed Term Computation Continued}

Our main result of last week is the proof or Theorem~\ref{thm:hterm}, which states that well-typed terms are hereditarily terminating under substitution by 
hereditarily terminating maps:

\begin{thm}[Hereditary termination]\label{thm:hterm}
  If $\hasEF{\Gamma}{M}{A}$ and $\gamma : \cdot \to \Gamma$ so that
  $\hterm{\Gamma}{\gamma}$ then $\hterm{A}{\hat{\gamma}(M)}$.
\end{thm}

Given this result, we should move on to proving that hereditary termination actually implies
termination.  Before doing so, however, we first have to finish our proof
Theorem~\ref{thm:hterm}, which relies on a lemma known as \emph{head expansion} or
\emph{reverse execution}, which we are going to prove next:

\begin{lem}[Head expansion]\label{lem:hexp}
  If $\hterm{A}{M'}$ and $\step{M}{M'}$ then $\hterm{A}{M}$.
\end{lem}

\begin{proof}
  By induction on the structure of $A$.
  \begin{itemize}
  \setlength\itemsep{1em}
  \item $A = b$\\
    Suppose $\hterm{b}{M'}$ and $\step{M}{M'}$. We need to show $\hterm{b}{M}$. By the
    definition of hereditary termination, it suffices to show that $M \Downarrow$. 
    This is clear since $M' \Downarrow$ and $\step{M}{M'}$.
  \item $A = \fn{A_1}{A_2}$\\
    Suppose $\hterm{\fn{A_1}{A_2}}{M'}$ and $\step{M}{M'}$. We need to show $\hterm{\fn{A_1}{A_2}}{M}$.
    Now suppose $\hterm{A_1}{N}$, and we need to show $\hterm{A_2}{\ap{M}{N}}$. 
    By the definition of hereditary termination and by our assumption, we have $\hterm{A_2}{\ap{M'}{N}}$. 
    Note that $\step{\ap{M}{N}}{\ap{M'}{N}}$ by our assumption and the compuation rule.
    Now applying the IH\footnote{Note that the IH can be applied because the induction is on
      the structure of $A$, not on terms.  Whereas the terms in the hereditary termination
      predicate get bigger, the types get smaller.}, we have $\hterm{A_2}{\ap{M}{N}}$.
    \qedhere
  \end{itemize}
\end{proof}

Let's now return to proving our ultimate goal, namely that hereditary termination implies termination of terms in the STLC:

\begin{thm}[Termination]\label{thm:term}
  If $\hterm{A}{M}$ then $\termb{M}$.
\end{thm}

\begin{proofattempt}
  Induction on $A$.
   \begin{itemize}
  \setlength\itemsep{1em}
  \item $A = b$\\
    Suppose $\hterm{b}{M}$. Done by definition of hereditary termination.  
  \item $A = \fn{A_1}{A_2}$\\
    Suppose $\hterm{\fn{A_1}{A_2}}{M}$. We need to show $\termb{M}$. We are stuck here.
  \end{itemize}
\end{proofattempt}

The question is what we should do at this point.  One possibility is to basically ``give up''
and weaken the theorem to expect termination to hold only for observables, i.e., for terms of
the base type.  Alternatively, we could strengthen the induction hypothesis even further by
changing the defintion of $\hterm{A}{M}$ to also insist on $\termb{M}$.

Another option is to change the definition of hereditary termination entirely and use a
\emph{positive} formulation rather than a \emph{negative} one.  Whereas our current negative
formulation is phrased in terms of an implication, the positive formulation requires the term
to be an actual lambda.  \emph{Positive hereditary termination} $\htermp{A}{M}$ of a term $M$ at type
$A$ is defined as follows:

% Notice in the definition of hereditary termination above, we take a ``negative'' approach with respect to the 
% function type, in that we consider everything that can happen in the elimination form for functions. Dually, we can 
% also formulate hereditary termination ``positively''. 

\begin{align*}
  \htermp{b}{M} &\triangleq \steps{M}{c} \quad (\exists c : b)\\
  \htermp{\fn{A}{B}}{M} &\triangleq \steps{M}{\lam{x}{A}{M_2}} \; \text{and}\;
  \forall M_1.\ \htermp{A}{M_1} \implies \htermp{A_2}{[M_1/x]M_2}
\end{align*}

The positive formulation is also known as the \emph{method of canonical forms}.  Using a
positive formulation of hereditary termination, the proof of the introductory rule is
straightforward, whereas the elimination rule takes some work, relying on head expansion, in
particular.

We can rephrase the above definition of $\htermp{A}{M}$ by introducing the notion of a
\emph{hereditary value} $\hv{A}{V}$ of a value $V$ at type $A$

\begin{align*}
  \htermp{b}{M} &\triangleq \steps{M}{\valueJ{V}} \;\text{such that}\; \hv{A}{V}
\end{align*}

\noindent and then give a definition of a hereditary value that differs depending on whether a \emph{call-by-value}
($\mathsf{cbv}$) or \emph{call-by-name} ($\mathsf{cbn}$) semantics is used:

\begin{align*}
\hv{b}{V} &\triangleq V = c : b \\
\hvcbv{\fn{A_1}{A_2}}{V} &\triangleq V = \lam{x}{A}{M_2} \; \text{and} \; \forall V_1. \hvcbv{A_1}{V_1} \implies
\hterm{A_2}{[V_1/x]M_2} \\
\hvcbn{\fn{A_1}{A_2}}{V} &\triangleq V = \lam{x}{A}{M_2} \; \text{and} \; \forall M_1. \hterm{A_1}{M_1} \implies
\hterm{A_2}{[M_1/x]M_2}
\end{align*}

In both interpretations a hereditary value at a base type is simply a value, whereas at a
function type the insistance on a value as an argument is only made for a call-by value
semantics, but not for a call-by-name semantics.  From this perspective, the negative
formulation of hereditary termination only really makes sense in the context of a call-by-name
semantics.  Only then, function arguments can be unevaluated computations that permit the weakened
termination theorem discussed earlier that insists on termination for base types only.  For a
call-by-value semantics, on the other hand, a positive formulation of hereditary termination is
required for termination, as this formulation is phrased in terms of values.

In this case study of closed term computation we have assumed a call-by-name evaluation
semantics.  For a call-by-value evaluation semantics, our main fundamental
Theorem~\ref{thm:hterm} needs to be phrased in terms of hereditary values, with
$\hvcbv{\Gamma}{\gamma}$ correspondingly defined:

\begin{thm}[Call-by-value hereditary termination]\label{thm:hcbvterm}
  If $\hasEF{\Gamma}{M}{A}$ and $\gamma : \cdot \to \Gamma$ so that
  $\hvcbv{\Gamma}{\gamma}$ then $\htermp{A}{\hat{\gamma}(M)}$.
\end{thm}

\subsection{Summary}

The introduction of behavioral invariants has allowed us to express the \emph{semantic}
property of \emph{syntax}.  Specifically, we have shown that well-type terms terminate.  To
talk about the semantic properties of terms, we introduce the following notation:

\begin{align*}
&A \; \mathsf{type} &&\text{iff}  \quad \quad A, B ::= b \mid \fn{A}{B}&\\
&\gamma \in \Gamma &&\text{iff} \quad \quad \hterm{\Gamma}{\gamma}&\\
&M \in A &&\text{iff} \quad \quad \hterm{A}{M}&\\
&\hasEC{\Gamma}{M}{A} &&\text{iff} \quad \quad \gamma \in \Gamma \implies \hat\gamma(M) \in A&
\end{align*}

Note that the ``membership'' relation has a computational flavor; it is a behavorial condition
on $M$ (or $\gamma$), which says that $M$ (or $\gamma$) satisfies the specification $A$ (or
$\Gamma$).

Now, we can state our main fundamental Theorem~\ref{thm:hterm} as follows:

\begin{thm}
If $\hasEF{\Gamma}{M}{A}$, then $\hasEC{\Gamma}{M}{A}$.
\end{thm}

This is in some sense a soundness theorem (in the language of formal logics), which means that
the formally derivable terms are actually true (according to the computational
specification). Therefore, we can think of formal systems as a way of \emph{accessing} the
\emph{truth}. Note that we make no claims that $\hasEC{\Gamma}{M}{A}$ be decidable, as it is
fruitless to expect the truth to be decidable in general. \footnote{See Derek Dreyer's Milner
  Award Lecture for further discussion.}

\section{Open Term Computation}

We can also interpret variables in a different sense, where they stand for open terms - they are indeterminates. 
Then the judgment $\hasEF{\Gamma}{M}{A}$ could be interpreted as a mapping $M$ from open terms to the type $A$.\\

Recall the beta-contraction relation, $\contrb{P}{P'}$:
\[
\contrb{\ap{(\lam{x}{A}{M})}{N}}{[N/x]M}
\]

With this, we formulate beta-reduction: 

\begin{mathpar}
\inferrule{
  \contrb{M}{N}
}{
  \stepb{M}{N}
} \and
\inferrule{
  \stepb{M}{M'}
}{
  \stepb{\ap{M}{N}}{\ap{M'}{N}}
} \and
\inferrule{
  \stepb{N}{N'}
}{
  \stepb{\ap{M}{N}}{\ap{M}{N'}}
} \and
\inferrule{
  \stepb{M}{M'}
}{
  \stepb{\lam{x}{A}{M}}{\lam{x}{A}{M'}}
} 
\end{mathpar}

Further, we define beta-normal form, $\bnf{N}$:
\[
\bnf{N} \triangleq N \not\mapsto_{\beta}
\]

Where $N \not\mapsto_{\beta}$ means that no beta-reduction rule applies to $N$. This informal notation can be formalized.

Finally, we can define when a term is beta-normalizing, $\bnorm{M}$: 
\[
\bnorm{M} \triangleq \exists N. \stepbs{M}{N} \;\text{and}\; \bnf{N}
\]

Now we can state the fundamental theorem for normalization: 
\begin{thm}
If $\hasEF{\Gamma}{M}{A}$ then $\bnorm{M}$.
\end{thm}

Following the proof for termination, we introduce a stronger notion for normalization, \emph{hereditary normalization}:

\begin{align*}
  \hnorm{\Delta}{b}{M} &\triangleq \bnorm{M}\\
  \hnorm{\Delta}{\fn{A}{B}}{M} &\triangleq
  \forall N.\ \hnorm{\Delta}{A}{N} \implies \hnorm{\Delta}{B}{\ap{M}{N}}
\end{align*}
As before, we define $\hnorm{\Delta}{\Gamma}{\gamma}$ to be $\forall x : A \in \Gamma.\ \hnorm{\Delta}{A}{\gamma(x)}$

Some notes about this definition:
\begin{itemize}
\setlength\itemsep{1em}
\item This predicate is defined on terms $M$ s.t. $\hasEF{\Delta}{M}{A}$. The extra argument to the hereditary normalization (compared to termination) $\Delta$ reflects this fact, since normalization is defined on open terms. \\
\item We formulate hereditary normalization in the negative way; turns out this is the only way\\
\item What can we say about positive types (sums)?
\end{itemize}

The proof can be divided into three lemmas: 
\begin{lem}\label{l1}
If $\hasEF{\Gamma}{M}{A}$ and $\hnorm{\Delta}{\Gamma}{\gamma}$, then $\hnorm{\Delta}{A}{\hat\gamma(M)}$.
\end{lem}

\begin{lem}\label{l2}
If $\hnorm{\Delta}{A}{M}$ then $\bnorm{M}$.
\end{lem}

\begin{lem}\label{l3}
$\hnorm{\Gamma}{\Gamma}{\id{\Gamma}}$.
\end{lem}

\subsection{Lemma \ref{l1}}

\begin{lemm}
If $\hasEF{\Gamma}{M}{A}$ and $\hnorm{\Delta}{\Gamma}{\gamma}$, then $\hnorm{\Delta}{A}{\hat\gamma(M)}$.
\end{lemm}

\begin{proof}
Induction on typing. 
\begin{itemize}
  \setlength\itemsep{1em}
  \item $\hasEF{\Gamma', x : A}{x}{A}$\\
    Suppose $\hnorm{\Delta}{\Gamma}{\gamma}$. 
    We need to  show $\hnorm{\Delta}{A}{\hat{\gamma}(x)}$, which follows directly by our assumption.
  \item $\hasEF{\Gamma}{c}{b}$\\
    Suppose $\hnorm{\Delta}{\Gamma}{\gamma}$. We need to show that $\hnorm{\Delta}{b}{c}$, which is to 
    show that $\bnorm{c}$. Since $\bnf{c}$, we have $c$ normalizing in 0 steps. 
  \item $\inferrule{
      \hasEF{\Gamma}{M}{\fn{A}{B}}\\
      \hasEF{\Gamma}{N}{A}
    }{\hasEF{\Gamma}{\ap{M}{N}}{B}}$\\
    Applying the IH, we have $\hnorm{\Delta}{\fn{A}{B}}{\hat{\gamma}(M)}$
    and $\hnorm{\Delta}{A}{\hat{\gamma}(N)}$. By definition of hereditary normalization, we have
    $\hnorm{\Delta}{B}{\ap{\hat{\gamma}(M)}{\hat{\gamma}(N)}}$. Since $\hat{\gamma}(\ap{M}{N}) = 
    \ap{\hat{\gamma}(M)}{\hat{\gamma}(N)}$, we have
    $\hnorm{\Delta}{B}{\hat{\gamma}(\ap{M}{N})}$.
  \item
    $\inferrule{
      \hasEF{\Gamma, x : A}{M}{B}\\
    }{\hasEF{\Gamma}{\lam{x}{A}{M}}{\fn{A}{B}}}$\\
    Suppose $\hnorm{\Delta}{\Gamma}{\gamma}$. We need to show that 
    $\hnorm{\Delta}{\fn{A}{B}}{\hat{\gamma}(\lam{x}{A}{M})}$. 
    Thus, suppose $\hnorm{\Delta}{A}{N}$. It suffices to show that 
    $\hnorm{\Delta}{B}{\ap{\hat{\gamma}(\lam{x}{A}{M})}{N}}$. Notice that with $N$, we can extend $\gamma$ to be
    $\gamma' = \gamma[x \mapsto N]$, and since $N$ is hereditarily normalizing in $A$, it follows that 
    $\hnorm{\Delta}{\Gamma,x : A}{\gamma'}$. Applying the IH, we have that $\hnorm{\Delta}{B}{\hat{\gamma'}(M)}$. 
    Since substitution is commutative, $\hat{\gamma'}(M) = [N/x]\hat{\gamma}(M)$, and 
    $\ap{\hat{\gamma}(\lam{x}{A}{M})}{N} = \ap{\lam{x}{A}{\hat{\gamma}(M)}}{N}$. Further, 
    $\step{\ap{\lam{x}{A}{\hat{\gamma}(M)}}{N}}{[N/x]\hat{\gamma}(M)}$ by the computation rule. The result then follows
    from head expansion. 
   \qedhere
\end{itemize}
\end{proof}

The proof for the next 2 lemmas will be mutual induction. However, this is not enough. We will need to further
strengthen Lemma \ref{l3}, but let's see where we fail.

\subsection{Lemma \ref{l2}}

\begin{lemm}
If $\hnorm{\Delta}{A}{M}$ then $\bnorm{M}$.
\end{lemm}

\begin{proofattempt}
Induction on $A$.
\begin{itemize}
  \setlength\itemsep{1em}
  \item $A = b$\\
  Immediate from definition of hereditary normalization.
  \item $A = \fn{A_1}{A_2}$\\
  Suppose $\hnorm{\Delta}{A}{M}$. We need to show $\bnorm{M}$. Now we are in a bind. If we can somehow use 
  Lemma \ref{l3} to obtain a hereditarily normalizing input of type $A_1$, then we can proceed with induction 
  to obtain the result. For now, we will assume (*) that gives us a $\Gamma'$ such that $x : A_1 \in \Gamma'$.
  By Lemma (\ref{l3}) we have $\hnorm{\Gamma'}{\Gamma'}{\id{\Gamma'}}$, in particular, $\hnorm{\Gamma'}{A_1}{x}$. By 
  the definition of hereditary normalization, we further have that $\hnorm{\Gamma'}{A_2}{\ap{M}{x}}$. Now applying the 
  IH, we see that $\bnorm{\ap{M}{x}}$, and by Lemma \ref{l5} we have $\bnorm{M}$. 
  \qedhere
\end{itemize}
\end{proofattempt}

Postponing the issues with this proof, we will move on to proving Lemma \ref{l3}, and resolve everything after
presenting the two proofs.

\subsection{Lemma \ref{l3}}

\begin{lemm}
$\hnorm{\Gamma}{\Gamma}{\id{\Gamma}}$.
\end{lemm}

\begin{proofattempt}
Let $x : A \in \Gamma$ be arbitrary. Proceed with induction on $A$.
\begin{itemize}
  \setlength\itemsep{1em}
  \item $A = b$\\
  We need to show $\hnorm{\Gamma}{b}{x}$, which holds since $\bnf{x}$.
  \item $A = \fn{A_1}{A_2}$\\
  We need to show $\hnorm{\Gamma}{\fn{A_1}{A_2}}{x}$. Suppose $\hnorm{\Gamma}{A_1}{M_1}$, and it suffices to show
  that $\hnorm{\Gamma}{A_2}{\ap{x}{M_1}}$. Now it would be nice to apply the IH to complete the proof, but which
  is not strong enough.
\end{itemize}
\end{proofattempt}

Notice that for every variable of function type, we want them to be hereditarily normalizing when applied to 
beta-normal terms. We will add this to the lemma:

\begin{lem}\label{l4}
For all $k$, if $x : A_1 \to \dots \to A_k \to A \in \Gamma$ and for each $i \le k$, $\hasEF{\Gamma}{M_i}{A_i}$ and $\bnorm{M_i}$, then $\hnorm{\Gamma}{A}{\ap{\ap{x}{M_1}}{\dots M_k}}$.
\end{lem}

Note that we need each argument to be \emph{normalizing} instead of merely \emph{hereditarily normalizing}. 
Before proving \ref{l4}, we shall see how to fix \ref{l2}. The crucial fact is the (*) that gave us a context $\Gamma'$
out of thin air, which allowed us to supply the argument to the function term. To do this, we will make a change to 
the definition of hereditary normalization, which will involve the notion of Kripke semantics.

\subsection{Kripke semantics}

Kripke semantics, or presheafs, are related to/gives the solution to the following:
\begin{itemize}
\item allocating a ``fresh variable'' in a context
\item why should hereditary normalization be stable under context extensions? (In other words, why should context
extensions be admissible for hereditary normalization?)
\end{itemize}

If we think of the typing context as a model of the ``world'', then Kripke semantics suggests that we can stipulate
hereditary normalization to hold in all context extensions, or ``future words''. Viewed logically, this would be the
admissibility of weakening with respect to hereditary normalization.\\

Thus for our definition, we change the higher order terms to expect ``world extensions'':

\[
\hnorm{\Delta}{\fn{A_1}{A_2}}{M} \triangleq \;\text{if}\; \Delta' \ge \Delta \;\text{and}\; \hnorm{\Delta'}{A_1}{M_1}
\;\text{then}\; \hnorm{\Delta'}{A_2}{\ap{M}{M_1}}
\]

Note that $\ge$ is a pre-order on contexts $\Delta$, and is reflexive and transitive. Now we can fix both lemmas. 

\subsection{Fix Lemmas}

\begin{lemm}
If $\hnorm{\Delta}{A}{M}$ then $\bnorm{M}$.
\end{lemm}

\begin{proof}
Induction on $A$.
\begin{itemize}
  \setlength\itemsep{1em}
  \item $A = b$\\
  Immediate from definition of hereditary normalization.
  \item $A = \fn{A_1}{A_2}$\\
  Suppose $\hnorm{\Delta}{A}{M}$. We need to show $\bnorm{M}$. Let $\Delta' = \Delta, x : A_1$. By IH on Lemma \ref{l4} 
  (with $k = 0$), we have $\hnorm{\Delta'}{A_1}{x}$. By the definition of hereditary termination, we have 
  $\hnorm{\Delta'}{A_2}{\ap{M}{x}}$. Applying the IH, we have $\bnorm{\ap{M}{x}}$, and by Lemma \ref{l5} we have 
  $\bnorm{M}$. 
  \qedhere
\end{itemize}
\end{proof}

\begin{lem}\label{l4}
For all $k$, if $x : A_1 \to \dots \to A_k \to A \in \Gamma$ and for each $i \le k$, $\hasEF{\Gamma}{M_i}{A_i}$ and $\bnorm{M_i}$, then $\hnorm{\Gamma}{A}{\ap{\ap{x}{M_1}}{\dots M_k}}$.
\end{lem}

\begin{proof}
Proceed with induction on $A$.
\begin{itemize}
  \setlength\itemsep{1em}
  \item $A = b$\\
  We need to show $\bnorm{\ap{\ap{x}{M_1}}{\dots M_k}}$, which holds since each $M_i$ normalizing, and thus the 
  term does not beta-reduce.
  \item $A = \fn{A_i}{A_j}$\\
  We need to show $\hnorm{\Gamma}{A}{\ap{\ap{x}{M_1}}{\dots M_k}}$. Let $\Gamma' \ge \Gamma$ and 
  $\hnorm{\Gamma'}{A_i}{M_i}$. It suffices to show that $\hnorm{\Gamma'}{A_j}{\ap{\ap{\ap{x}{M_1}}{\dots M_k}}{M_i}}$.
  By IH on Lemma \ref{l2}, we have that $\bnorm{M_i}$, and we obtain the result by applying the IH with $k + 1$.
\end{itemize}
\end{proof}

\section{Strong Normalization}
\bibliographystyle{plainnat}
\bibliography{ctt}

\end{document}
