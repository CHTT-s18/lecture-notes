\documentclass{article}
\usepackage{chtt-notes}
\scribes{Daniel Gratzer}
\week{1}
\doXRs

%%%% Don't break the formatting
\setitemize{noitemsep, topsep=0pt, leftmargin=*}
\setenumerate{noitemsep, topsep=0pt, leftmargin=*}
\setdescription{noitemsep, topsep=0pt, leftmargin=*}

\begin{document}

\maketitle

\section{Administrative Materials}

\begin{itemize}
\item Participation in this course is measured by presenting papers to
  the seminar and taking notes on a weekly basis. There
  will be no credit assigned for the course.
\item The course website is \url{http://cs.cmu.edu/~rwh/courses/chtt}.
\end{itemize}

\section{Course Overview}

This is a course dedicated to computational type theory.  The term \emph{computational type
  theory} (CTT) is due to Constable in conjunction with the
Nuprl project~\citep{Constable:86}, based on Martin-L\"{o}f's meaning
explanation~\citep{MartinLof:79}.  Possible topics of this course
include, but are not limited to, the following:
\begin{itemize}
\item \emph{Fundamentals} of computational type theory in the zero
  dimensional setting (as opposed to the higher dimensional setting).
\item Explore the \redprl{}~\citep{redprl} proof assistant, a usable
  implementation of computational higher type theory.
\item \emph{Guarded computational type theory} (Sterling and Harper). Rough
  idea: talk about causality/time in type theory to discuss the
  consumption of infinite data through time.
\item Higher type theory, i.e., \emph{cubical type theory}, in the computational
  form~\citep{Angiuli:chtt1:16,Angiuli:chtt2:16,Angiuli:chtt3:17,Cavallo:18}.  Of particular
  concern in this setting are \emph{univalence} and \emph{inductive types}.  Univalence is
  concerned with the identification of ``equivalent'' types.  A computational interpretation of
  equivalence is roughly based on the idea of a line between types, allowing to map back and forth
  between them.
\item Computational type theory as a \emph{specification language} for program
  \emph{verification}. Specifically the idea of using computational type theory to handle general
  references (Gratzer and Crary) and more generally that types
  classify computation, not merely logical propositions!  This idea adopts the
  Brouwer-Heyting-Kolmogorov operational interpretation of intuitionistic logic that views type
  theory as a theory of \emph{truth}, whereas the formal correspondence discovered by Curry and
  Howard views type theory as a theory of \emph{proof}.
\item A comparison of formal \emph{higher type theories}, such as
  \citet{Cohen:16} (Kleene algebra form) and \citet{Angiuli:cart:17} (Cartesian form).
\end{itemize}

This week's lecture will be a more modest discussion of what defines
the difference between \emph{computational} type theory versus
\emph{formal} type theory. This discussion will be supplemented by
investigating the practical differences between the formal perspective
and the computational perspective on a simple type theory. This
exercises will have two further benefits for our future study:
\begin{enumerate}
\item Provide a historical context for the development of critical
  ideas. In particular, the development of a relational interpretation
  of types.
\item Foreshadow some of the necessary technical tools for developing
  higher type theory~\citep{Angiuli:chtt:17}. Higher type theory can roughly
  be understood as a \emph{master theory of coercion} for the purpose of investigating the
  meaning of type equality.  Both formal and computational higher type theory rely on
  two notions of a variable.
\end{enumerate}

\section{Computational Type Theory versus Formal Type Theory}
We now position \emph{formal type theory} (FTT) with \emph{computational type theory} (CTT).
In the beginning we focus on the zero dimensional setting and expand to the higher dimensional
setting in later parts of this course.  Formal type theory is also known as \emph{structural},
\emph{prescriptive}, or \emph{axiomatic} type theory, whereas computational type theory is
referred to as \emph{behavioral}, \emph{descriptive}, or \emph{semantic}.

\subsection{Formal Type Theory (FTT)}
\emph{Formal type theory} is a \emph{formal logic}. It defines a
\emph{recursively-enumerable} set inductively by a collection of rules of
judgments. The key judgments for formal type theory are:
\begin{itemize}
\item $\hasTF{\Gamma}{A}$
\item $\hasEF{\Gamma}{M}{A}$
\end{itemize}
These judgments are relative to the typing context $\Gamma$, which provides the typing of the
free variables.  A type theory is a collection of rules of such judgments, defining the
precise meanings of ``type'' and ``term''. As a type theory becomes more complex, also
necessary is a form of equality. Traditionally called \emph{definitional equality} it is also
defined by a pair of judgments:
\begin{itemize}
\item $\hasTEF{\Gamma}{A}{B}$
\item $\hasEEF{\Gamma}{M}{N}{A}$
\end{itemize}
These judgments are also used to classify \emph{calculation}. All of
the rules are expected to be stable under $\equiv$. A common way that
this is ensured (at least for types) is the inclusion of the following
rule:
\[
  \inferrule{
    \hasEF{\Gamma}{M}{A}\\
    \hasTEF{\Gamma}{A}{B}
  }{\hasEF{\Gamma}{M}{B}}
\]
These judgments link the type theory to a formal logic. A type, as
defined by $\hasTF{\Gamma}{A}$, can be treated as a proposition. Then
a term can be understood as a proof. A proposition is true when there
is a term that occupies the corresponding type, giving rise to the following formal correspondence:
\[
\hasTF{\Gamma}{A} \quad \sim \quad \ensuremath{\Gamma \vdash A\ \mathsf{prop}}\]
\[\hasEF{\Gamma}{M}{A} \quad \sim \quad \ensuremath{\Gamma \vdash A\ \mathsf{true}}
\]
There is a slight
mismatch in that there are types in type theory which seem utterly
trivial from the perspective of logic. For instance, the natural
numbers are just $\top$ under this interpretation, but we intuitively
believe that the natural numbers have a more interesting structure
than say, unit!

We should also expect that a standard formal type theory will define a
\emph{constructive} logic. The meaning of constructive is slightly ambiguous
in its modern usage. There are at least two senses in which this can be
interpreted.
\begin{enumerate}
\item A \emph{computational} meaning: proofs are programs that run.\\
  This is a property not enjoyed by an arbitrary collection of rules
  and it also fails for interesting collections of sound rules. For
  instance, ``book HoTT'' is not constructive in this sense.

  There is a certain intrinsic motivation for this for a computer
  scientist; types classify terms which are thought of as programs so
  we expect them to run. Beyond aesthetic considerations, there are
  practical advantages to this. Upon calculating a particular term of
  type \nat{} with some desired property, in a constructive type
  theory this term can be run to an actual numeral.

\item The \emph{absence} of classical principles.\\
  That is, no proofs of the law of excluded middle, double negation
  elimination, or other similar propositions. This view has been
  termed ``not not constructive''. The absence of these principles
  means the theory admits more models and correspondingly more logical
  principles, giving rise to axiomatic freedom. Another interesting
  observation is the fact that univalence is simply inconsistent in
  the presence of the law of excluded middle so it's only sensible to
  consider in a constructive theory.
\end{enumerate}
Formal type theory also is recursively enumerable by definition. Many
people also want the judgments to be \emph{decidable}, as opposed to merely
semidecidable. If the four judgments are decidable then, since type
checking is proof checking, it's possible to validate whether a proof
is correct. If you're using type theory for a foundation of
mathematics centered around mechanization then perhaps this is a
desirable property. Even in the most decidable theory the complexity of
deciding judgments is ``horrendous''. This presents a slight practical
issue for the above story.

Finally, the \emph{computational} meaning of formal type theory is \emph{extrinsically}
imposed.  If the type theory is constructive in
the first sense then, after-the-fact, we can build such an
interpretation.

\subsection{Computational Type Theory (CTT)}
In \emph{computational type theory}, rather than working with computation
after-the-fact, computation comes first. Moreover, rather than a collection
of inference rules defining the type theory and bolting on models of
these rules, the type theory stands alone.  In particular, it does not depend on any other
mathematical structures (e.g., sets) for its interpretation.  Computational type theory
can be seen as a foundational \emph{theory of truth}, as opposed to a theory of
proof.  Computational type theory thus adopts the Brouwerian principle, in which truth is based on
computation because the notion of computation (or of an algorithm) is a fundamental human faculty.

The start of computational type theory is then a system of
computation, a programming language.  The resulting deterministic transition system relies on two
judgments
\begin{itemize}
\item $\step{M}{M'}$, which signifies that the program $M$ steps to the
  program $M'$,
\item $\valueJ{M}$,  which signifies that $M$ no longer steps and is a
  done computing.
\end{itemize}
The types are defined by a meaning explanation~\citep{MartinLof:79} in
terms of computation. All of the objects of this meaning explanation
are drawn from this programming language so that they have a notion of
computation off the bat. This also means that types are just certain
programs which avoids a sticking point in formal type theory, the
precise relationship between terms and types.

In computational type theory, a type provides a specification of what values belong to it and
is as such a program that runs to a value.  % In computational type theory, a type is simply a program that runs to
% a value for which we have a specification of what values belong to
% it.
A specification may look like
\begin{quote}
  $\nat$ specifies programs which evaluate to either $0$ or
  $\mathrm{suc}(M)$ where $M$ is a term of type $\nat$.
\end{quote}
This specification gives an obvious definition of what elements of a
type are. An element of a type is simply a program which satisfies the
specification of the type. We bundle up these ideas into two
judgments:
\begin{itemize}
\item $\hasTC{\Gamma}{A}$, for when $A$ evaluates to a value
  which names a specification,
\item $\hasEC{\Gamma}{M}{A}$, for when $M$ evaluates to a value that satisfies the
  specification of $A$.
% \item $\hasEC{\Gamma}{M}{A}$, for when an $M$ satisfies the
%   specification of $A$.
\end{itemize}
These judgments are not enumerable or checkable in any sense. They are
arbitrarily complex and are richer than the recursively enumerable set
we encountered with formal type theory.

Types will also come equipped with a notion of \emph{exact equality}, $\eeq$,
specified at each type. This notion of equality can be bundled up into
judgments:
\[
  \hasTEC{\Gamma}{A}{B} \qquad \qquad
  \hasEEC{\Gamma}{M}{N}{A}
\]
Since these judgments are wildly undecidable there is a question of
convenience. We can no longer rely on computers to aid us in the
construction or checking of these judgments. In order to work with
this, computational type theory can be paired with some proof theory
(constructed after-the-fact) which provides a convenient way to
establish these judgments. They cannot capture the full scope of the
original judgments but by carving out useful subsets we can work with
them. Since these proof theories are secondary and cannot possibly be
canonical, it is supremely unimportant if they satisfy many classic
proof theory properties. For instance, it's common to add explicit
rules for cut or even that are not closed under computational
equivalence.

\section{Case Study: The Simply Typed Lambda Calculus with Base Types}

We now turn to trying to make the above discussion more concrete by
applying it to a simple type theory. The language consists of the types
\[
  A, B ::= b \mid \fn{A}{B}
\]
and the terms
\[
  M, N ::= x \mid c \mid \lam{x}{A}{M} \mid \ap{M}{N}
\]
The grammar for contexts is the completely standard version for
non-dependent type theories.
\[
  \Gamma ::= \cdot \mid \Gamma, x : A\ (\text{distinct variables})
\]
To express this language as a formal type theory we now define the
usual judgments:
\begin{mathpar}
  \inferrule{ }{\hasEF{\Gamma, x : A}{x}{A}}\and
  \inferrule{
    \hasEF{\Gamma}{M}{\fn{A}{B}}\\
    \hasEF{\Gamma}{N}{A}
  }{\hasEF{\Gamma}{\ap{M}{N}}{B}} \and
  \inferrule{ }{\hasEF{\Gamma}{c_0}{b}} \and
  \inferrule{ }{\hasEF{\Gamma}{c_1}{b}}
\end{mathpar}
Here we have assumed the existence of two distinct constants $c_0$ and $c_1$ of type
$b$. We want these constants to provide some way to distinguish
terms and they can be thought of as the two different answers a
program might provide. This does not, however, make the base type equivalent to
$\bool$. Firstly, we have not ensured that the base type includes no
other constants besides these two distinguished points, so the base
type may include arbitrarily many distinct members. Even if we did
limit ourselves to just two such constants we have not internalized
branching on the difference in our program. In other words, we have
not included anything like an {\tt if}. This is deliberate, we wish the
base type to only serve as the answer that a program provides, and
including {\tt if} would lead to subtle complications for the upcoming
proof of normalization.

Of course to work with this theory we need a notion of substitution:
$[M/x]M'$. More generally, it will be necessary to have a notion of
substitution between contexts to allow for simultaneous substitution:
$\gamma : \Delta \to \Gamma$. This means that for each
$x : A \in \Gamma$, we then have $\hasEF{\Delta}{\gamma(x)}{A}$. The
notion of applying a substitution can be extended to a full term:
$\hat{\gamma}(M)$. The details of this are elided because they're a
bit low level.

There remains the question of the judgment $\equiv$. We would expect
to have some notion of computation included in $\equiv$. More than
this, however, we would also expect it to be the case that $\equiv$ is
a reflexive, symmetric, transitive, compatible (congruent)
relation. The first three qualifiers ensure that $\equiv$ behaves like
some notion of equality and the last would allow us to replace equals
by equals in terms. Two examples rules are:
\begin{mathpar}
  \inferrule{
    \hasEEF{\Gamma, x : A}{M}{N}{B}
  }{\hasEEF{\Gamma}{\lam{x}{A}{M}}{\lam{x}{A}{N}}{\fn{A}{B}}}\and
  \inferrule{
    \hasEF{\Gamma, x : A}{M}{B}\\
    \hasEF{\Gamma}{N}{A}\\
  }{\hasEEF{\Gamma}{\ap{(\lam{x}{A}{M})}{N}}{[N/x]M}{B}}\and
\end{mathpar}
This concludes the essentials of a formal system here. We now turn to
the computational understanding of this system. There are two ways to
do this and we will briefly explore both.
\begin{itemize}
\item We study closed terms only\footnote{Perhaps we can restrict
  to even just closed terms of base type.} with a notion of closed
  evaluation.
\item We can study open terms of any type paired with a notion of
  \emph{open reduction}.
\end{itemize}
Of particular importance is the differences in the equalities that these two
approaches produce. Foreshadowing: in the closed term model there are
a number of equalities that hold on open terms because we know that
variables will be replaced with closed terms of the right type. This
permits reasoning principles that are simply invalid in an open
reductions setting. For instance, in the closed setting only should
$\hasEEF{x: \nat}{x + 0}{0 + x}{\nat}$.

\subsection{Closed Term Computation}

In order to get started with the first computational interpretation we
need to define the notion of closed evaluation that we're working
with. This comes with two judgments,
\[
  \valueJ{M} \qquad
  \step{M}{M'}
\]
On top of these judgments we can of course specify the reflexive
transitive closure of $(\step{-}{-})$: $(\steps{-}{-})$. We will also
use the specialized version of $(\steps{-}{-})$: $(\eval{-}{-})$. The
rule is that $\eval{M}{N}$ if $\steps{M}{N}$ and $\valueJ{N}$. The
rules for stepping and value judgments are given below.
\begin{mathpar}
  \label{head-reduction}
  \inferrule{ }{\valueJ{c}}\and
  \inferrule{ }{\valueJ{\lam{x}{A}{M}}}\and
  \inferrule{
    \step{M}{M'}
  }{\step{\ap{M}{N}}{\ap{M'}{N}}}\and
  \inferrule{ }{\step{\ap{(\lam{x}{A}{M})}{N}}{[N/x]M}}
\end{mathpar}
This could also have been phrased in terms of evaluation contexts. In
this set up we define
\[
  E ::= \cdot \mid \ap{E}{M}
\]
Then we can define $\fillin{E}{M}$ as filling the hole with $M$. The
main appeal of this approach is that our operational semantics
collapse to the rule
\[
  \step{M}{M'} \iff M = \fillin{E}{P} \AND M' = \fillin{E}{P'} \AND P \mathrel{\text{contr}_\beta} P'
\]
Where $P \mathrel{\text{contr}_\beta} P'$ holds precisely when
$P = \ap{(\lam{x}{A}{M})}{N}$ and $P' = [N/x]M$ for some $M$ and $N$.

With this machinery in hand, we are in a position to state the goal
that we are after for the closed notion of computation.
\begin{quote}
  We want to show that all well-typed programs terminate. That is, if
  $\hasEF{\cdot}{M}{A}$ then there is a $V$ so that $\eval{M}{V}$
  (shortened to $M \Downarrow$).
\end{quote}
There is an analogous statement for open terms, which is that open
terms ``normalize''. To formally state this, however, we need a notion
of open reduction which we do not currently possess.

\begin{claim}
  If $\hasEF{\cdot}{M}{A}$ then $M \Downarrow$.
\end{claim}
\begin{proofattempt}
  We proceed by induction on $\hasEF{\cdot}{M}{A}$ which is
  inductively defined after all.
  \begin{itemize}
  \setlength\itemsep{1em}
  \item $\hasEF{\Gamma', x : A}{x}{A}$\\
    This is vacuous since we assumed $\Gamma = \cdot$.
  \item $\inferrule{
      \hasEF{\cdot}{M}{\fn{A}{B}}\\
      \hasEF{\cdot}{N}{A}
    }{\hasEF{\cdot}{\ap{M}{N}}{B}}$\\
    In this case our induction hypothesis tells us that $M \Downarrow$
    and $N \Downarrow$. So we can get that $\ap{M}{N}$ runs to two
    values applied to each other.

    At this point, we're very stuck however. There's no intrinsic
    reason that values applied to each other should always
    terminate. In fact, this is only true if the application is
    well-typed. As an explicit instance of this, if the program was
    not required to be typed we would have to show that the following
    terminates.
    \[
      \ap{(\lam{x}{?}{\ap{x}{x}})}{(\lam{x}{?}{\ap{x}{x}})}
    \]
  \end{itemize}
  And thus our proof fails.
\end{proofattempt}

\subsubsection{Tait's Groundbreaking Development}

This leads us to the conclusion that we should \emph{strengthen} our
induction hypothesis somehow so that we have leverage when we get to
that case. In particular, we need something with better closure
conditions. This motivates the switch from termination to
\emph{hereditary termination}. In order for this switch to be a good
move, we need two facts to hold.
\begin{enumerate}
\item It needs to imply termination.
\item It needs to be strong enough to get the induction to go through.
\end{enumerate}
Hereditary termination, in the style first proposed by
Tait~\citep{Tait:67}, written $\hterm{A}{-}$ will be defined by
induction on the type, $A$.
\begin{align*}
  \hterm{b}{M} &\triangleq M \Downarrow\\
  \hterm{\fn{A}{B}}{M} &\triangleq
  \forall N.\ \hterm{A}{N} \implies \hterm{B}{\ap{M}{N}}
\end{align*}
The key insight is that it's defined by induction on the structure of
$A$. As a side note, this is the negative formulation of the clause
for hereditary termination at function type since it is phrased in terms of an implication. This could also be
formulated to ensure that functions are terms which run to lambdas and
also satisfy the above condition. This positive formulation is
slightly more robust in the case of empty types but will be equivalent
for our system.

Let's reattempt our proof, but with a strengthened induction hypothesis.
\begin{claim}
  If $\hasEF{\cdot}{M}{A}$ then $\hterm{A}{M}$.
\end{claim}
\begin{proofattempt}
  We proceed by induction on $\hasEF{\cdot}{M}{A}$ which is
  inductively defined after all.
  \begin{itemize}
  \setlength\itemsep{1em}
  \item $\hasEF{\Gamma', x : A}{x}{A}$\\
    This is vacuous since we assumed $\Gamma = \cdot$.
  \item $\hasEF{\cdot}{c}{b}$\\
    This is immediate since we must show that $c \Downarrow$ but
    $\steps{c}{c}$ and $\valueJ{c}$ holds.
  \item $\inferrule{
      \hasEF{\cdot}{M}{\fn{A}{B}}\\
      \hasEF{\cdot}{N}{A}
    }{\hasEF{\cdot}{\ap{M}{N}}{B}}$\\
    We have by assumption that $\hterm{\fn{A}{B}}{M}$ and
    $\hterm{A}{N}$ and therefore, unfolding definitions we have
    $\hterm{B}{\ap{M}{N}}$.
  \item
    $\inferrule{
      \hasEF{x : A}{M}{B}\\
    }{\hasEF{\cdot}{\lam{x}{A}{M}}{\fn{A}{B}}}$\\
    Now we must show that $\hterm{\fn{A}{B}}{\lam{x}{A}{M}}$. In order to
    show this, suppose that we have some $N$ so that $\hterm{A}{N}$
    holds. We must now show that the following holds.
    \[
      \hterm{B}{\ap{(\lam{x}{A}{M})}{N}}
    \]
    And now we are stuck. We have no assumption about $M$ since it's
    open. On the other hand, we do know that
    $\step{\ap{(\lam{x}{A}{M})}{N}}{[N/x]M}$.
  \end{itemize}
  And thus our proof fails.
\end{proofattempt}

We've failed, but we failed a step further which is progress. Our
issue seems to be that we need a bit of information about open terms
because we need to know about the body of the function. The definition
of hereditary termination is going to make us apply it after all. The
key move here to switch from closed terms to working with open terms
and proving something about all well-behaved closing substitutions
applied to these open terms. In order to write down our strengthened
claim let us define an extension of $\hterm{-}{-}$ to substitutions:
\[
  \hterm{\Gamma}{\gamma} \triangleq \forall x : A \in \Gamma.\ \hterm{A}{\gamma(x)}
\]
\begin{theorem}
  If $\hasEF{\Gamma}{M}{A}$ and $\gamma : \cdot \to \Gamma$ so that
  $\hterm{\Gamma}{\gamma}$ then $\hterm{A}{\hat{\gamma}(M)}$.
\end{theorem}
\begin{proof}
  We proceed by induction on $\hasEF{\cdot}{M}{A}$ which is
  inductively defined after all.
  \begin{itemize}
  \setlength\itemsep{1em}
  \item $\hasEF{\Gamma', x : A}{x}{A}$\\
    In this case, we must show that $\hterm{A}{\hat{\gamma}(x)}$ but
    this is immediate since $x : A$ is in the context and
    $\hterm{\Gamma}{\gamma}$.
  \item $\hasEF{\Gamma}{c}{b}$\\
    This is immediate since we must show that
    $\hat{\gamma}(c) = c \Downarrow$ but $\steps{c}{c}$ and
    $\valueJ{c}$ holds.
  \item $\inferrule{
      \hasEF{\Gamma}{M}{\fn{A}{B}}\\
      \hasEF{\Gamma}{N}{A}
    }{\hasEF{\Gamma}{\ap{M}{N}}{B}}$\\
    We have by assumption that $\hterm{\fn{A}{B}}{\hat{\gamma}(M)}$
    and $\hterm{A}{\hat{\gamma}(N)}$ and therefore, unfolding
    definitions we have $\hterm{B}{\hat{\gamma}(\ap{M}{N})}$.
  \item
    $\inferrule{
      \hasEF{\Gamma, x : A}{M}{B}\\
    }{\hasEF{\Gamma}{\lam{x}{A}{M}}{\fn{A}{B}}}$\\

    Now we must show that
    $\hterm{\fn{A}{B}}{\hat{\gamma}(\lam{x}{A}{M})}$. In order to show
    this, suppose that we have some $N$ so that $\hterm{A}{N}$
    holds. We must now show that the following holds.
    \[
      \hterm{B}{\ap{\hat{\gamma}(\lam{x}{A}{M})}{N}}
    \]
    It is obvious that it suffices to show that
    $\hterm{B}{[N/x]\hat{\gamma}(M)}$. We can then define the
    substitution $\gamma' = \gamma[x \mapsto N]$. Then, by the
    assumptions that $\hterm{\Gamma}{\gamma}$ and $\hterm{A}{N}$ we
    have that $\hterm{\Gamma, x : A}{\gamma'}$. Our goal, though, can
    be rewritten:
    \[
      \hterm{B}{[N/x]\hat{\gamma}(M)} \iff
      \hterm{B}{\hat{\gamma}'(M)}
    \]
    What was our induction hypothesis for $M$ again? It states
    precisely that since $\hterm{\Gamma, x : A}{\gamma'}$ then
    $\hterm{B}{\hat{\gamma}'(M)}$ holds so we're done. \qedhere
  \end{itemize}
\end{proof}
In the above, I have claimed that it was \emph{obvious} that if a term
steps to another one and this new term is hereditarily terminating
the first term is terminating. This can be formally stated:
\begin{lemma}
  If $\hterm{A}{M'}$ and $\step{M}{M'}$ then $\hterm{A}{M}$.
\end{lemma}
This is a result of the determinacy of $\mapsto$ and the fact that
hereditary termination is purely behavioral: it relies only on the
evaluation behavior of $M$.

\bibliographystyle{plainnat}
\bibliography{ctt}

\end{document}
