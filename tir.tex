\documentclass[11pt,twoside]{article}
\usepackage[authoryear,semicolon]{natbib}
\usepackage[T1]{fontenc}
\usepackage[utf8]{inputenc}
\usepackage{textgreek}
\usepackage{fullpage}
\usepackage[color=yellow]{todonotes}
\setlength{\marginparwidth}{1.25in}
\usepackage{xifthen}
\usepackage{amsmath,amssymb,amsthm,mathtools,stmaryrd}
\usepackage{proof,mathpartir}
\usepackage{colonequals}
\usepackage{code,verbatim}
\usepackage{comment}
\usepackage{textcomp}
\usepackage[us]{optional}
\usepackage{color}
\usepackage{url}
\usepackage{graphics}
\usepackage{import}
\usepackage{stackengine}
\usepackage{scalerel}

\newcommand{\IsNat}[1]{{#1}\,\mathsf{nat}}
\newcommand{\zeronat}{\mathtt{zero}}
\newcommand{\succnat}[1]{\mathtt{succ}(#1)}

\newcommand{\IsEven}[1]{{#1}\,\mathsf{even}}
\newcommand{\IsOdd}[1]{{#1}\,\mathsf{odd}}

\allowdisplaybreaks[1]       %mildly permissible to break up displayed equations

\begin{document}

\title{Strong Normalization as Transfinite Induction on Reduction}
\author{Robert Harper}
\date{\today}

\maketitle{}

\section{Introduction}

A formal type system is inductively defined by a collection of rules for deriving
judgments of the form $\Gamma\vdash M:A$ and $\Gamma\vdash M\equiv M':A$ expressing, respectively, that a term
$M$ is structurally well-typed at type $A$, relative to typing assumptions for variables
given by the context $\Gamma$, and that well-typed terms $M$ and $M'$ of type $A$ are
\emph{convertible}, or \emph{definitionally equivalent}, at type $A$, relative to context
$\Gamma$.

It is common to consider whether or not conversion between terms is decidable. A number of
methods have been developed to address such questions. Of interest in this note are
methods that characterize conversion by a \emph{reduction} relation on terms with the
property that two well-typed terms are convertible iff they have the same fully reduced
form, known as their common \emph{normal form}. The correctness of this procedure relies
on showing that a well-typed term has a \emph{unique normal form}. This requirement may be
broken into two parts, a \emph{normalization theorem} stating that every well-typed term
has a normal form, and a \emph{confluence theorem} ensuring that the normal form is
unique.

The specifics of such proofs depend heavily on the type system under consideration, though
in some cases confluence can be proved independently of typing.


\section{Conclusion}

\bibliographystyle{plainnat}
\bibliography{notes}

\end{document}

%%% Local Variables:
%%% mode: latex
%%% TeX-master: t
%%% fill-column: 90
%%% auto-fill-mode: t
%%% End:
