\documentclass[11pt,twoside]{article}
\usepackage[authoryear,semicolon]{natbib}
\usepackage[T1]{fontenc}
\usepackage[utf8]{inputenc}
\usepackage{textgreek}
\usepackage{fullpage}
\usepackage[color=yellow]{todonotes}
\setlength{\marginparwidth}{1.25in}
\usepackage{xifthen}
\usepackage{amsmath,amssymb,amsthm,mathtools,stmaryrd}
\usepackage{proof,mathpartir}
\usepackage{colonequals}
\usepackage{code,verbatim}
\usepackage{comment}
\usepackage{textcomp}
\usepackage[us]{optional}
\usepackage{color}
\usepackage{url}
\usepackage{graphics}
\usepackage{import}
\usepackage{stackengine}
\usepackage{scalerel}

\newcommand{\IsNat}[1]{{#1}\,\mathsf{nat}}
\newcommand{\zeronat}{\mathtt{zero}}
\newcommand{\succnat}[1]{\mathtt{succ}(#1)}

\newcommand{\IsEven}[1]{{#1}\,\mathsf{even}}
\newcommand{\IsOdd}[1]{{#1}\,\mathsf{odd}}

\allowdisplaybreaks[1]       %mildly permissible to break up displayed equations

\begin{document}

\title{Strong Normalization as Transfinite Induction on Reduction}
\author{Robert Harper}
\date{\today}

\maketitle{}

\section{Introduction}

A formal type system is inductively defined by a collection of rules for deriving
judgments of the form $\Gamma\vdash M:A$ and $\Gamma\vdash M\equiv M':A$ expressing, respectively, that a term
$M$ is structurally well-typed at type $A$, relative to typing assumptions for variables
given by the context $\Gamma$, and that well-typed terms $M$ and $M'$ of type $A$ are
\emph{convertible}, or \emph{definitionally equivalent}, at type $A$, relative to context
$\Gamma$.

The question arises, is conversion decidable for well-typed terms?  By Scott's Theorem
conversion between untyped terms is undecidable, so the only hope for proving decidability
is to take advantage of typing.  Many such methods have been developed.
\emph{Reduction methods} work by checking whether or not two terms of the same
type can be reduced to a common term by applying simplifications.
\emph{Normalization methods} demand that the common term be itself irreducible,
which is to say a common \emph{normal form}.

The use of reduction and normalization-based methods for deciding conversion may be
justified by the principle of \emph{transfinite induction on reduction}.  Writing $M\to N$
for reduction, a property $\mathcal{P}$ of terms is said to be \emph{$\to$-inductive} iff
\begin{displaymath}
  \forall M.(\forall N.M\to N\supset\mathcal{P}(N))\supset\mathcal{P}(M).
\end{displaymath}
The principle of transfinite $\to$-induction states that any $\to$-inductive property holds of
every term:\footnote{Thinking of $M\to N$ as analogous to $n<m$ for natural numbers, this
  is exactly the statement of ``strong induction,'' which is derivable from mathematical
  induction.}
\begin{center}
  if $\mathcal{P}$ is $\to$-inductive, then $\forall M.\mathcal{P}(M)$.
\end{center}
The validity of transfinite $\to$-induction may be taken as the definition of the
\emph{well-foundedness} of the reduction relation, which is often defined as the
non-existence of infinite reduction sequences, a property called \emph{strong
  normalization} for reduction.  Indeed normalization (the existence of normal forms) and
strong normalization may both be proved by transfinite $\to$-induction once its validity has
been established.  Moreover, it may be used to prove that two terms are convertible iff
they may be reduced to a common (possibly normal) form, providing a decision procedure for
definitional equivalence.

The purpose of this note is to provide a direct proof of the validity of transfinite
induction on $\beta$-reduction for well-typed terms for what a class of properties called
reduction properties.

\section{Validity of Transfinite $\beta$-Reduction for Well-Typed Terms}

is this worth writing down???  experts will say they know all that.  there's nothing
interesting about the proof.  the students probably will not appreciate it.


\section{Conclusion}

\bibliographystyle{plainnat}
\bibliography{notes}

\end{document}

%%% Local Variables:
%%% mode: latex
%%% TeX-master: t
%%% fill-column: 90
%%% auto-fill-mode: t
%%% End:
