\documentclass{article}
\usepackage{chtt-notes}

\scribes{Ryan Kavanaugh, Jon Sterling}
\week{12}
% The following command will let you cross-reference labels
% in the files week1.tex, week2.tex, \ldots, week\@week.tex,
% where if l is a label in ``weekN.tex'', then you can access
% the label using \cref{WN:l}.
\doXRs

% General remark: Using \cref{label} will fill in the appropriate
% environment name. For example,
% ``\begin{lemma}\label{lem:foo} ...\end{lemma} By \cref{lem:foo}''
% will produce ``Lemma 15 ... By lemma~15''


\newcommand\Interval{\mathbb{I}}
\newcommand\IIndOp{\Interval\textsf{-ind}}
\newcommand\IInd[5]{\IIndOp\Squares*{#1}\Parens*{#2;#3;#4}\Parens*{#5}}
\newcommand\Seg{\mathsf{seg}}
\newcommand\Transport[3]{\mathsf{tr}\Squares*{#1}\Parens*{#2;#3}}

\newcommand\Circle{\mathbb{C}}
\newcommand\Base{\mathsf{base}}
\newcommand\Loop{\mathsf{loop}}
\newcommand\CIndOp{\Circle\textsf{-ind}}
\newcommand\CInd[4]{\CIndOp\Squares{#1}\Parens*{#2;#3}\Parens{#4}}
\newcommand\DimSubst[3]{#3\langle{#1}/{#2}\rangle}

\DeclarePairedDelimiter\Parens{\lparen}{\rparen}
\DeclarePairedDelimiter\Squares{[}{]}

\usetikzlibrary{arrows}

\begin{document}
\maketitle

\section{Last Time}

Last time, we introduced ``higher'' inductive definitions,
simultaneously defining the types $A$, $\Id{A}{-}{-}$,
$\Id{\Id{A}{-}{-}}{-}{-}$, etc.\ by their generators.

We also introduced the abstract interval $\Interval$ with endpoints
$0,1:\Interval$ and path constructor $\Seg:\Id{\Interval}{0}{1}$. To
construct a map out of the interval into a type $A$ is to ``draw a
line'' in the type $A$: you specify two points, and a path in $A$ that
connects them. In fact, a map $\Interval^n\to{}A$ can be thought of as
a defining an $n$-cube in $A$.

\section{Elimination for the interval}

The elimination rule (induction principle) for the interval is written
as follows, writing $\DIdeq{i.C}{\Seg}{M_0}{M_1}$ for the homogeneous
identification $\Ideq{\subst{1}{i}{C}}{\Transport{i.C}{\Seg}{M_0}}{M_1}$:
%
\begin{mathpar}
  \inferrule{
    \hasTF{\Gamma,u:\Interval}{C}
    \\
    \hasEF{\Gamma}{M}{\Interval}
    \\
    \hasEF{\Gamma}{M_0}{\subst{0}{i}{C}}
    \\
    \hasEF{\Gamma}{M_1}{\subst{1}{i}{C}}
    \\
    \hasEF{\Gamma}{M_\Seg}{
      \DIdeq{i.C}{\Seg}{M_0}{M_1}
    }
  }{
    \hasEF{\Gamma}{
      \IInd{i.C}{M_0}{M_1}{M_\Seg}{M}
    }{
      \subst{M}{i}{C}
    }
  }
\end{mathpar}

For well-typed instances of the elimination form, we impose the
following beta rules:
\begin{mathpar}
  \inferrule{}{
    \hasEEF{
      \IInd{i.C}{M_0}{M_1}{M_\Seg}{0}
    }{M_0}{
      \subst{0}{i}{C}
    }
  }
  \and
  \inferrule{}{
    \hasEEF{
      \IInd{i.C}{M_0}{M_1}{M_\Seg}{1}
    }{M_1}{
      \subst{1}{i}{C}
    }
  }
\end{mathpar}

We also have a ``propositional beta rule'' for the $\Seg$ constructor;
in this particular formal system, we are not free to impose a
definitional equality, which is a bit strange:
\[
  \Ideq{
    \Parens*{\DIdeq{i.C}{\Seg}{M_0}{M_1}}
  }{
    \hapd{}{
      \IInd{i.C}{M_0}{M_1}{M_\Seg}{-}
    }\Parens*{\Seg}
  }{
    M_\Seg
  }
\]

\section{The Circle $\Circle$ aka $\mathbb{S}^1$}

The circle $\Circle$ is given by the following generators, induction
principle, and equations:
%
\begin{mathpar}
  \inferrule{}{
    \hasEF{\Gamma}{\Base}{\Circle}
  }
  \and
  \inferrule{}{
    \hasEF{\Gamma}{\Loop}{
      \Ideq{\Circle}{\Base}{\Base}
    }
  }
  \and
  \inferrule{
    \hasTF{\Gamma,c:\Circle}{C}
    \\
    \hasEF{\Gamma}{M_\Base}{
      \subst{\Base}{c}{C}
    }
    \\
    \hasEF{\Gamma}{M_\Loop}{
      \DIdeq{c.C}{\Loop}{M_\Base}{M_\Base}
    }
  }{
    \hasEF{\Gamma}{
      \CInd{c.C}{M_\Base}{M_\Loop}{M}
    }{
      \subst{M}{c}{C}
    }
  }
  \\
  \hasEEF{
    \CInd{c}{C}{M_\Base}{M_\Loop}{\Base}
  }{M_\Base}{
    \subst{\Base}{c}{C}
  }
  \\
  \Ideq{
    \Parens*{
      \DIdeq{c.C}{\Loop}{M_\Base}{M_\Base}
    }
  }{
    \hapd{}{
      \CInd{c.C}{M_\Base}{M_\Loop}{-}
    }\Parens*{\Loop}
  }{M_\Loop}
\end{mathpar}

One thing worth pointing out is that it would have been possible (but
not correct) to replace the third premise of the elimination rule with
$\hasEF{\Gamma}{M_\Loop}{\Ideq{\subst{\Base}{c}{C}}{M_\Base}{M_\Base}}$. This
would have been type-correct, but it does not capture the right case
in the induction principle, which should be to exhibit a loop in $C$
which lies over the generating $\Loop$ in $\Circle$.

\subsection{Loop space}

There is the (based) loop space
$\Omega\Parens*{A;a_0}\triangleq\Ideq{A}{a_0}{a_0}$ for any point
$a:A$. Here, the groupoid laws become \emph{group} laws, because the
endpoints are the same. To ensure that this is actually a group, one
can take its ``zero-truncation''
$\parallel\Omega\Parens*{A;a_0}\parallel_0$, in which all the higher
structure has been squashed down to reflexivities. This truncation of
the loop space is called the ``fundamental group'' of a type.

Using the univalence axiom, it can be shown that the loop space of the
circle $\Circle$ with base point $\Base$ can be identified with the
group of integers under addition. Because the integers are already
0-truncated, we have also shown that the fundamental group of the
circle is the integers~\citep{Licata:13}.


\section{Judgmental methodology}

To study the degree to which HoTT has computational meaning (i.e.\
develop a way to phrase homotopy type theory which does have
computational meaning), we return to Martin-L\"of's version of the
notion of a \emph{judgment}, which forms the epistemic foundation of
modern type theory~\citep{MartinLof:87,MartinLof:94,MartinLof:96}.

The guiding idea is that judgments come first: structure which appears
in the connectives and type constructors is developed using analogous
structure at the level of judgments.

Gentzen's key contribution was his stress on \emph{entailment}
(hypothetical judgment) as being prior to implication; compare Hilbert
systems which have \emph{only} implication. The contradiction between
entailment and implication has a parallel in the old battle between
lambda calculus and combinators.

Entailment $J_0\vdash J_1$ expresses that the fact $J_1$ can be
concluded once the fact $J_0$ is assumed, but it does not yet express
\emph{generality}. The \emph{generic judgment} is the structure from
which the universal quantification connective is built, in the same
way as the hypothetical judgment lies underneath the implication
connective; using the notation of \citet{MartinLof:notes:87}:
\[
  \inferrule{
    \vert_x\ \Parens*{\trueF{A}}
  }{
    \trueF{\forall x.A}
  }
\]

Constructivism (which we take here as the principle that the evidence
that makes a proposition true is a mathematical object) suggests a
consolidation of the hypothetical and the generic judgments into a
single form, named ``hypothetico-general judgment'' by Martin-L\"of:
\[
  \framebox{$\hasEF{x_1:A_1,\ldots,x_n:A_n}{M}{A}$}
\]

This notation emphasizes the centrality of variables in type
theory. We must reiterate that type theory does not fix a particular
interpretation of variables, but is instead compatible with multiple
explanations of variables, each of which can be deployed in different
settings with different aims. We will call out two possible
interpretations of variables in particular:

\medskip
\begin{center}
  \begin{tabular}{ll}
    \textbf{formal} & \textbf{semantic}
    \\
    indeterminates / generic values & placeholders
    \\
    ``derivability'' & ``admissibility'' (careful!)
    \\
    $\hasEF{\Gamma}{M}{A}$ & $\hasEC{\Gamma}{M}{A}$
  \end{tabular}
\end{center}

Both forms of judgment above share the \emph{structural properties}
and present a consequence relation. However, the variables on the
formal side range over open elements (elements that may themselves
have free variables), whereas the variables on the semantic side range
over only closed elements. The semantic consequence relation and
account of variables as placeholders is essentially the mathematical
notion of a variable, whereas the formal consequence relation and
account of variables as indeterminates is proof-theoretic in nature,
making terms analogous to polynomials.

The defeasible structural principles which define the structure of
contexts (weakening/projections, contraction/diagonals,
exchange/symmetries) are designed to be admissible each consequence
relation, as are the non-defeasible ones (identity and
composition). The negotiation of the structural maps will appear again
later in the course in the context of cubical type theory.

\begin{remark}
  In the first version of Church's lambda calculus, you were required
  to use every variable that is introduced. Later this idea of working
  without weakening was codified into what some call ``relevant
  logic'' and others call ``strict logic''.
\end{remark}



\section{Judgmental structure of identifications}

To give a type-theoretic account of paths/identifications, we must
conceive them as an intrinsic concept in type theory, emanating from
the judgmental base. We are not free to simply ``throw in''
identifications without explaining them at the judgmental level, since
this would disrupt many important properties of type theory, including
its computational content.

There are multiple possible ways to give a judgmental account of
higher-dimensional structure, each of which is based on a different
kind of shape; the type-theoretic community has mostly settled on a
\emph{cubical} account of higher dimensional structure.\footnote{There
  are several different possible flavors of cubical structure; it is a
  matter of current scientific inquiry to determine the benefits and
  drawbacks of each version of cubical structure.}

In HoTT (which can be thought of as a ``vulgar globular'' account of
higher structure), higher structure is captured by iterating the
identification type. Instead, we want to give these dimension levels
names:

\begin{enumerate}
\item points
\item lines
\item squares
\item cubes
\item[\ldots] (and beyond)
\end{enumerate}

Accordingly, we extend the judgments of type theory to indicate
whether they are constructing a point in a type, or a line, or a cube,
or a higher cube, etc. For now, we might write
$\hasEEC{\Gamma}{M}{M'}{A}\mathrel{@}n$ to mean that $M$ and $M'$ are
equal ``cubes'' of type $A$ of dimension $n$. When $n=0$, this is the
familiar member equality judgment, but when $n\geq 1$, this form of
judgment asserts the unfamiliar equation between higher-dimensional
elements of a higher-dimensional type.

The way that we make sense of this is to say that an ``$n$-dimensional
type'' is a \emph{family} of types that is indexed by the generic
$n$-cube, i.e.\ the $n$th power of the interval. To make this precise,
we replace this $n$ parameter with a \emph{context} $\Psi$ (of length
$n$) of dimension variables, which are allowed to occur in
$\Gamma,M,M',A$.

When we assert the judgment $\hasEC{M}{A}\ [x]$, we are saying that
$A$ is a \emph{line of types} in the $x$ direction, and that $M$ is a
\emph{line of elements} of $A$. To get the left endpoint of the type
$A$ we write $\DimSubst{0}{x}{M}$, and we have
$\hasEC{\DimSubst{0}{x}{M}}{\DimSubst{0}{x}{A}}\ [\cdot]$. One of the
big difficulties of higher dimensional type theory is to develop a
theory of coercions and compositions which let us move an element from
the $0$ fiber to the $1$ fiber, or even the $0$ fiber to the $x$
fiber! We will study this further in future lectures.

It is important to note that the variables $\Psi=x,y,\ldots$ range
over an abstract notion of interval, rather than a metric one. In
particular, we do not have any notion of distance; you are either at
the endpoints $\DimSubst{0}{x}{A}$, $\DimSubst{1}{x}{A}$ or even in
``middle'' $\DimSubst{y}{x}{A}$, but there is no sense in which you
can be ``two-thirds of the way'', etc.

\nocite{HoTTBook:13}
\bibliographystyle{plainnat}
\bibliography{ctt}

\end{document}
