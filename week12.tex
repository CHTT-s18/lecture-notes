\documentclass{article}
\usepackage{chtt-notes}

\scribes{Ryan Kavanaugh, Jon Sterling}
\week{12}
% The following command will let you cross-reference labels
% in the files week1.tex, week2.tex, \ldots, week\@week.tex,
% where if l is a label in ``weekN.tex'', then you can access
% the label using \cref{WN:l}.
\doXRs

% General remark: Using \cref{label} will fill in the appropriate
% environment name. For example,
% ``\begin{lemma}\label{lem:foo} ...\end{lemma} By \cref{lem:foo}''
% will produce ``Lemma 15 ... By lemma~15''


\newcommand\Interval{\mathbb{I}}
\newcommand\IIndOp{\Interval\textsf{-ind}}
\newcommand\IInd[5]{\IIndOp\Squares*{#1}\Parens*{#2;#3;#4}\Parens*{#5}}
\newcommand\Seg{\mathsf{seg}}
\newcommand\Transport[3]{\mathsf{tr}\Squares*{#1}\Parens*{#2;#3}}

\newcommand\Circle{\mathbb{C}}
\newcommand\Base{\mathsf{base}}
\newcommand\Loop{\mathsf{loop}}
\newcommand\CIndOp{\Circle\textsf{-ind}}
\newcommand\CInd[4]{\CIndOp\Squares{#1}\Parens*{#2;#3}\Parens{#4}}

\DeclarePairedDelimiter\Parens{\lparen}{\rparen}
\DeclarePairedDelimiter\Squares{[}{]}

\usetikzlibrary{arrows}

\begin{document}
\maketitle

\section{Last Time}

Last time, we introudced ``higher'' inductive definitions,
simultaneously defining the types $A$, $\Id{A}{-}{-}$,
$\Id{\Id{A}{-}{-}}{-}{-}$, etc.\ by their generators.

We also introduced the abstract interval $\Interval$ with endpoints
$0,1:\Interval$ and path constructor $\Seg:\Id{\Interval}{0}{1}$. To
construct a map out of the interval into a type $A$ is to ``draw a
line'' in the type $A$: you specify two points, and a path in $A$ that
connects them. In fact, a map $\Interval^n\to{}A$ can be thought of as
a defining an $n$-cube in $A$.

\section{Elimination for the interval}

The elimination rule (induction principle) for the interval is written
as follows, writing $\DIdeq{i.C}{\Seg}{M_0}{M_1}$ for the homogeneous
identification $\Ideq{\subst{1}{i}{C}}{\Transport{i.C}{\Seg}{M_0}}{M_1}$:
%
\begin{mathpar}
  \inferrule{
    \hasTF{\Gamma,u:\Interval}{C}
    \\
    \hasEF{\Gamma}{M}{\Interval}
    \\
    \hasEF{\Gamma}{M_0}{\subst{0}{i}{C}}
    \\
    \hasEF{\Gamma}{M_1}{\subst{1}{i}{C}}
    \\
    \hasEF{\Gamma}{M_\Seg}{
      \DIdeq{i.C}{\Seg}{M_0}{M_1}
    }
  }{
    \hasEF{\Gamma}{
      \IInd{i.C}{M_0}{M_1}{M_\Seg}{M}
    }{
      \subst{M}{i}{C}
    }
  }
\end{mathpar}

For well-typed instances of the elimination form, we impose the
following beta rules:
\begin{mathpar}
  \inferrule{}{
    \hasEEF{
      \IInd{i.C}{M_0}{M_1}{M_\Seg}{0}
    }{M_0}{
      \subst{0}{i}{C}
    }
  }
  \and
  \inferrule{}{
    \hasEEF{
      \IInd{i.C}{M_0}{M_1}{M_\Seg}{1}
    }{M_1}{
      \subst{1}{i}{C}
    }
  }
\end{mathpar}

We also have a ``propositional beta rule'' for the $\Seg$ constructor;
in this particular formal system, we are not free to impose a
definitional equality, which is a bit strange:
\[
  \Ideq{
    \Parens*{\DIdeq{i.C}{\Seg}{M_0}{M_1}}
  }{
    \hapd{}{
      \IInd{i.C}{M_0}{M_1}{M_\Seg}{-}
    }\Parens*{\Seg}
  }{
    M_\Seg
  }
\]

\section{The Circle $\Circle$ aka $\mathbb{S}^1$}

The circle $\Circle$ is given by the following generators, induction
principle, and equations:
%
\begin{mathpar}
  \inferrule{}{
    \hasEF{\Gamma}{\Base}{\Circle}
  }
  \and
  \inferrule{}{
    \hasEF{\Gamma}{\Loop}{
      \Ideq{\Circle}{\Base}{\Base}
    }
  }
  \and
  \inferrule{
    \hasTF{\Gamma,c:\Circle}{C}
    \\
    \hasEF{\Gamma}{M_\Base}{
      \subst{\Base}{c}{C}
    }
    \\
    \hasEF{\Gamma}{M_\Loop}{
      \DIdeq{c.C}{\Loop}{M_\Base}{M_\Base}
    }
  }{
    \hasEF{\Gamma}{
      \CInd{c.C}{M_\Base}{M_\Loop}{M}
    }{
      \subst{M}{c}{C}
    }
  }
  \\
  \hasEEF{
    \CInd{c}{C}{M_\Base}{M_\Loop}{\Base}
  }{M_\Base}{
    \subst{\Base}{c}{C}
  }
  \\
  \Ideq{
    \Parens*{
      \DIdeq{c.C}{\Loop}{M_\Base}{M_\Base}
    }
  }{
    \hapd{}{
      \CInd{c.C}{M_\Base}{M_\Loop}{-}
    }\Parens*{\Loop}
  }{M_\Loop}
\end{mathpar}

One thing worth pointing out is that it would have been possible (but
not correct) to replace the third premise of the elimination rule with
$\hasEF{\Gamma}{M_\Loop}{\Ideq{\subst{\Base}{c}{C}}{M_\Base}{M_\Base}}$. This
would have been type-correct, but it does not capture the right case
in the induction principle, which should be to exhibit a loop in $C$
which lies over the generating $\Loop$ in $\Circle$.

\subsection{Loop space}

There is the (based) loop space
$\Omega\Parens*{A;a_0}\triangleq\Ideq{A}{a_0}{a_0}$ for any point
$a:A$. Here, the groupoid laws become \emph{group} laws, because the
endpoints are the same. To ensure that this is actually a group, one
can take its ``zero-truncation''
$\parallel\Omega\Parens*{A;a_0}\parallel_0$, in which all the higher
structure has been squashed down to reflexivities. This truncation of
the loop space is called the ``fundamental group'' of a type.

Using the univalence axiom, it can be shown that the loop space of the
circle $\Circle$ with base point $\Base$ can be identified with the
group of integers under addition. Because the integers are already
0-truncated, we have also shown that the fundamental group of the
circle is the integers~\citep{Licata:13}.


\section{What is the computational meaning of ``HoTT''?}

To study the degree to which HoTT has computational meaning (i.e.\
develop a way to phrase homotopy type theory which does have
computational meaning), we return to Martin-L\"of's version of the
notion of a \emph{judgment}, which forms the epistemic foundation of
modern type theory~\citep{MartinLof:87,MartinLof:94,MartinLof:96}.

The guiding idea is that judgments come first: structure which appears
in the connectives and type constructors is developed using analogous
structure at the level of judgments.

Gentzen's key contribution was his stress on \emph{entailment}
(hypothetical judgment) as being prior to implication; compare Hilbert
systems which have \emph{only} implication. The contradiction between
entailment and implication has a parallel in the old battle between
lambda calculus and combinators.

Entailment $J_0\vdash J_1$ expresses that the fact $J_1$ can be
concluded once the fact $J_0$ is assumed, but it does not yet express
\emph{generality}. The \emph{generic judgment} is the structure from
which the universal quantification connective is built, in the same
way as the hypothetical judgment lies underneath the implication
connective; using the notation of \citet{MartinLof:notes:87}:
\[
  \inferrule{
    \vert_x\ \Parens*{\trueF{A}}
  }{
    \trueF{\forall x.A}
  }
\]

Constructivism (which we take here as the principle that the evidence
of a proposition's truth is a mathematical object) suggests a
consolidation of the hypothetical and the generic judgments into a
single form, named ``hypothetico-general judgment'' by Martin-L\"of:
\[
  \framebox{$\hasEF{x_1:A_1,\ldots,x_n:A_n}{M}{A}$}
\]

This notation emphasizes the centrality of variables in type
theory. We must reiterate type theory does not fix a particular
interpretation of variables, but is instead compatible with multiple
explanations of variables, each of which can be deployed in different
settings with different aims. We will call out two possible
interpretations of variables in particular:

\medskip
\begin{center}
  \begin{tabular}{ll}
    \textbf{formal} & \textbf{semantic}
    \\
    indeterminates / generic values & placeholders
    \\
    ``derivability'' & ``admissibility'' (careful!)
    \\
    $\hasEF{\Gamma}{M}{A}$ & $\hasEC{\Gamma}{M}{A}$
  \end{tabular}
\end{center}

Both forms of judgment above share the \emph{structural properties}
and present a consequence relation. However, the variables on the
formal side range over open elements (elements that may themselves
have free variables), whereas the variables on the semantic side range
over only closed elements. The semantic consequence relation and
account of variables as placeholders is essentially the mathematical
notion of a variable, whereas the formal consequence relation and
account of variables as indeterminates is proof-theoretic in nature,
making terms analogous to polynomials.




\nocite{HoTTBook:13}
\bibliographystyle{plainnat}
\bibliography{ctt}

\end{document}
