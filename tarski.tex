\documentclass[11pt,twoside]{article}
\usepackage[authoryear,semicolon]{natbib}
\usepackage[T1]{fontenc}
\usepackage[utf8]{inputenc}
\usepackage{textgreek}
\usepackage{fullpage}
\usepackage[color=yellow]{todonotes}
\setlength{\marginparwidth}{1.25in}
\usepackage{xifthen}
\usepackage{amsmath,amssymb,amsthm,mathtools,stmaryrd}
\usepackage{proof,mathpartir}
\usepackage{colonequals}
\usepackage{code,verbatim}
\usepackage{comment}
\usepackage{textcomp}
\usepackage[us]{optional}
\usepackage{color}
\usepackage{url}
\usepackage{graphics}
\usepackage{import}
\usepackage{stackengine}
\usepackage{scalerel}

\allowdisplaybreaks[1]       %mildly permissible to break up displayed equations

\begin{document}
\title{Tarski's Theorem on Power Sets}
\author{Robert Harper}
\date{\today}

\maketitle{}

\section{Introduction}

Tarski's theorem states that a monotone function on a complete lattice has a
complete lattice of fixed points, in particular a least and greatest. A useful
class of special cases are powerset lattices ordered by inclusion.

\section{Powerset Lattices}

Let $X$ be a set, not necessarily empty, and let $\wp{X}$ be the set of all
subsets of $X$. The set $\wp{X}$ forms a complete lattice under set inclusion,
with meets given by intersection and joins given by union. That is, if
$\mathcal{X}\subset\wp{X}$, then $\bigcap{\mathcal{X}}$ is its meet (greatest lower bound) and
$\bigcup{\mathcal{X}}$ is its join (least upper bound). The least element is the join
of the empty set, namely $\emptyset$, and the greatest element is the meet of the empty
set, namely $X$.

A function $F:\wp{X}\to\wp{X}$ is \emph{monotone} if it preserves inclusion: if
$A\subseteq B\subseteq X$, then $F(A)\subseteq F(B)\subseteq X$. For monotone
$F$ on $\wp{X}$, a \emph{pre-fixed point} of $F$ is a set $A\subseteq X$ such that
$F(A)\subseteq A$, and a \emph{post-fixed point} of $F$ is a set $A\subseteq X$ such that
$X\subseteq F(X)$. A pre-fixed point of $F$ is also said to be \emph{$F$-closed} and a
post-fixed point of $F$ is said to be \emph{$F$-consistent}. Viewing the lattice
as a (skinny) )category, a monotone function $F$ on it is a \emph{functor}, a
pre-fixed point of it is an \emph{$F$-algebra} and a post-fixed point of it is
an \emph{$F$-coalgebra}. Thus, a \emph{least} pre-fixed point of $F$ is an
\emph{initial $F$-algebra} and a \emph{greatest} post-fixed point of $F$ is a
\emph{final $F$-coalgebra}.

If $L\subseteq X$ is a least pre-fixed point of a monotone $F$, then $L$ is a
\emph{fixed point} of $F$, which is to say $F(L)=L$. Given that
$F(L)\subseteq L$, it is enough to show that $L\subseteq F(L)$. Now with $L$ being the meet of
all pre-fixed points of $F$, it suffices to show that $F(L)$ is a pre-fixed
point of $F$. But by monotonicity of $F$ we have
$F(F(L))\subseteq F(L)\subseteq L$. (This is a special case of the theorem in category
theory stating that an initial algebra for a functor is an isomorphism.)

By duality if $M\subseteq X$ is the greatest post-fixed point of $F$, then $M$ is also a
fixed point of $F$. (Categorially, a final co-algebra for a functor is an
isomorphism.)

Every monotone $F:\wp{X}\to\wp{X}$ has a least pre-fixed point and greatest post-fixed
point, given by the equations
\begin{align*}
  \mu(F) & = \bigcap\{\,A\subseteq X\,\mid\, F(A)\subseteq A\,\} \\
  \nu(F) & = \bigcup\{\,A\subseteq X\,\mid\,A\subseteq F(A)\,\}
\end{align*}
By the previous remarks these are, respectively, the least and greatest fixed
points of $F$.

It is evident that $\mu(F)$ is contained in all pre-fixed points of $F$, being
their intersection.  It remains to show that $F(\mu(F))\subseteq \mu(F)$.  Being the
greatest lower bound of all pre-fixed points, it is enough to show that if
$F(A)\subseteq A$, then $F(\mu(F))\subseteq A$.  If $F(A)\subseteq A$, then $\mu(F)\subseteq A$ by definition, and
so by monotonicity $F(\mu(F))\subseteq F(A)\subseteq A$, and we are done.

Similarly, $\nu(F)$ contains all post-fixed points of $F$, it being their union,
and, arguing dually to the preceding, $\nu(F)$ is itself a post-fixed point of
$F$.  It is thus the greatest post-fixed point, and hence a fixed point.

It is also possible to show that the meet and join of any set of fixed points of
a monotone function is itself a fixed point, but this fact seems not to be as
useful as the construction of a least and greatest.

\end{document}

%%% Local Variables:
%%% mode: latex
%%% TeX-master: t
%%% fill-column: 90
%%% auto-fill-mode: t
%%% End:
