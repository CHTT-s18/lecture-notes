\documentclass[11pt,twoside]{article}
\usepackage[authoryear,semicolon]{natbib}
\usepackage[T1]{fontenc}
\usepackage[utf8]{inputenc}
\usepackage{textgreek}
\usepackage{fullpage}
\usepackage[color=yellow]{todonotes}
\setlength{\marginparwidth}{1.25in}
\usepackage{xifthen}
\usepackage{amsmath,amssymb,amsthm,mathtools,stmaryrd}
\usepackage{proof,mathpartir}
\usepackage{colonequals}
\usepackage{code,verbatim}
\usepackage{comment}
\usepackage{textcomp}
\usepackage[us]{optional}
\usepackage{color}
\usepackage{url}
\usepackage{graphics}
\usepackage{import}
\usepackage{stackengine}
\usepackage{scalerel}

\newcommand{\IsNat}[1]{{#1}\,\mathsf{nat}}
\newcommand{\zeronat}{\mathtt{zero}}
\newcommand{\succnat}[1]{\mathtt{succ}(#1)}

\newcommand{\IsEven}[1]{{#1}\,\mathsf{even}}
\newcommand{\IsOdd}[1]{{#1}\,\mathsf{odd}}

\allowdisplaybreaks[1]       %mildly permissible to break up displayed equations

\begin{document}

\title{Tarski's Theorem on Power Sets}
\author{Robert Harper}
\date{\today}

\maketitle{}

\section{Introduction}

Tarski's theorem states that a monotone function on a complete lattice has a complete
lattice of fixed points, in particular a least and greatest. A useful class of special
cases are powerset lattices ordered by inclusion.

\section{Powerset Lattices}

Let $X$ be a set, not necessarily empty, and let $\wp{X}$ be the set of all subsets of
$X$. The set $\wp{X}$ forms a complete lattice under set inclusion, with meets given by
intersection and joins given by union. That is, if $\mathcal{X}\subset\wp{X}$, then
$\bigcap{\mathcal{X}}$ is its meet (greatest lower bound) and $\bigcup{\mathcal{X}}$ is
its join (least upper bound). The least element is the join of the empty set, namely
$\emptyset$, and the greatest element is the meet of the empty set, namely $X$.

A function $F:\wp{X}\to\wp{X}$ is \emph{monotone} if it preserves inclusion: if
$A\subseteq B\subseteq X$, then $F(A)\subseteq F(B)\subseteq X$. For monotone $F$ on
$\wp{X}$, a \emph{pre-fixed point} of $F$ is a set $A\subseteq X$ such that $F(A)\subseteq
A$, and a \emph{post-fixed point} of $F$ is a set $A\subseteq X$ such that $X\subseteq
F(X)$. A pre-fixed point of $F$ is also said to be \emph{$F$-closed} and a post-fixed
point of $F$ is said to be \emph{$F$-consistent}. A \emph{least pre-fixed point} of a
monotone $F$ is the \emph{smallest} (with respect to containment) $F$-closed set, and a
\emph{greatest post-fixed point} of $F$ is the \emph{largest} (with respect to
containment) $F$-consistent set.

Viewing the lattice as a (skinny) category, a monotone function $F$ on it is a
\emph{functor}, a pre-fixed point of it is an \emph{$F$-algebra} and a post-fixed point of
it is an \emph{$F$-coalgebra}. Thus, a \emph{least} pre-fixed point of $F$ is an
\emph{initial $F$-algebra} and a \emph{greatest} post-fixed point of $F$ is a \emph{final
$F$-coalgebra}.

If $L\subseteq X$ is a least pre-fixed point of a monotone $F$, then $L$ is a \emph{fixed
point} of $F$, which is to say $F(L)=L$. Given that $F(L)\subseteq L$, it is enough to
show that $L\subseteq F(L)$. Now with $L$ being the meet of all pre-fixed points of $F$,
it suffices to show that $F(L)$ is a pre-fixed point of $F$. But by monotonicity of $F$ we
have $F(F(L))\subseteq F(L)\subseteq L$. (Categorially, an initial algebra for a functor
is an isomorphism.)

By duality if $M\subseteq X$ is a greatest post-fixed point of $F$, then $M$ is also a
fixed point of $F$. (Categorially, a final co-algebra for a functor is an isomorphism.)

Every monotone $F:\wp{X}\to\wp{X}$ has a (unique) least pre-fixed point and (unique)
greatest post-fixed point, given by the equations \begin{align*} \mu(F) & =
\bigcap\{\,A\subseteq X\,\mid\, F(A)\subseteq A\,\} \\ \nu(F) & = \bigcup\{\,A\subseteq
X\,\mid\,A\subseteq F(A)\,\} \end{align*} By the previous remarks these are, respectively,
the least and greatest fixed points of $F$.

It is evident that $\mu(F)$ is contained in all pre-fixed points of $F$, being their
intersection. It remains to show that $F(\mu(F))\subseteq \mu(F)$. Being the greatest
lower bound of all pre-fixed points, it is enough to show that if $F(A)\subseteq A$, then
$F(\mu(F))\subseteq A$. If $F(A)\subseteq A$, then $\mu(F)\subseteq A$ by definition, and
so by monotonicity $F(\mu(F))\subseteq F(A)\subseteq A$, and we are done.

Similarly, $\nu(F)$ contains all post-fixed points of $F$, it being their union, and,
arguing dually to the preceding, $\nu(F)$ is itself a post-fixed point of $F$. It is thus
the greatest post-fixed point, and hence a fixed point.

The least fixed point of a monotone $F$ on $\wp{X}$ affords the following \emph{induction
  principle}: to show that $\mu(F)\subseteq A$, it suffices to show that
$F(A)\subseteq A$, which is to say that $A$ is $F$-closed. Similarly, the greatest post-fixed
point, $\nu(F)$, of $F$ affords the following \emph{coinduction principle}: to show that
$A\subseteq \nu(F)$, it suffices to show that $A\subseteq F(A)$, which is to say that $A$ is $F$-consistent.

Re-phrased in terms of predicates and implication, the least fixed point of a monotone $F$
is the \emph{strongest} property $A$ of elements of $X$ such that if $x\in F(A)$, then
$x\in A$. Thus, to show that $x\in\mu(F)$ implies $x\in A$, it is enough to show that if
$x\in F(A)$, then $x\in A$.  Dually, the greatest fixed point of a monotone $F$ is the
\emph{weakest} property $A$ of elements of $X$ such that if $x\in A$, then $x\in F(A)$.

As a case in point there are two proofs that $\mu(F)\subseteq\nu(F)$, one using the minimality of
$\mu(F)$, the other using the maximality of $\nu(F)$.  Because $\mu(F)$ is the least pre-fixed
point of $F$, it is itself $F$-closed, $F(\mu(F))\subseteq\mu(F)$, and because $\nu(F)$ is the greatest
post-fixed point of $F$, it is itself $F$-consistent.  Thus, to show the containment it
suffices to show either that $\nu(F)$ is $F$-closed, $F(\nu(F))\subseteq\nu(F)$, or that
$\mu(F)$ is $F$-consistent, $\mu(F)\subseteq F(\mu(F))$.  But these are exactly the converse
containments required to show that $\mu(F)$ and $\nu(F)$ are fixed points of $F$!

\smallskip

It is also possible to show that the meet and join of any set of fixed points of a
monotone function is itself a fixed point, but this seems not to be as useful as the
construction of a least and greatest.

\section{Assertions and Rules}

A typical application of the fixed point constructions is to justify the definition of one
or more \emph{assertions}, or \emph{judgment forms}, by a collection of \emph{rules}. The
idea is that the rules constitute an inductive definition of the mentioned assertions.  For example,
the following rules define the judgment $\IsNat{n}$, stating that $n$ is a natural number:
\begin{mathpar}
  \inferrule[zero]
  {\strut}
  {\IsNat{\zeronat}}

  \inferrule[succ]
  {\IsNat{n}}
  {\IsNat{\succnat{n}}}
\end{mathpar}
Similarly, the even and odd numbers may be simultaneously defined by the following rules:
\begin{mathpar}
  \inferrule[zero-even]
  {\strut}
  {\IsEven{\zeronat}}

  \inferrule[succ-odd]
  {\IsEven{n}}
  {\IsOdd{\succnat{n}}}

  \inferrule[succ-even]
  {\IsOdd{n}}
  {\IsEven{\succnat{n}}}
\end{mathpar}
In both cases the subjects of the assertions are abstract binding trees in the sense
of~\citet{pfpl2}, among which are those used above.

The forms of assertion may be thought of as labels that distinguish one from another.  Thus, the
underlying set $X$ of the inductive definition is the collection of \emph{assertions} consisting of
an abt together with a label drawn from the set of such forms.  Each rule $r$ of the form
\begin{displaymath}
  \inferrule
  {j_{1} \dots j_{n}}
  {j}
\end{displaymath}
in which the $j_{i}$'s and $j$ are assertions (elements of $X$), defines a monotone function
$F_{r}:\wp{X}\to\wp{X}$ that applies rule $r$ to a given set $A\subseteq X$ of assertions:
\begin{displaymath}
  F_{r}(A)=\{\, j\in X \,\mid\, j_{1},\dots,j_{n}\in A\,\}.
\end{displaymath}
A set $R$ of rules induces a monotone function that collectively closes up under each rule in the
set:
\begin{displaymath}
  F_{R}(A) = \bigcup_{r\in R} F_{r}(A).
\end{displaymath}
The judgments defined by the set of rules $R$ are precisely those in $\mu(F_{R})$.  The principle of
\emph{rule induction} is simply the induction principle associated with the least fixed point of
$F_{R}$.

\smallskip

What about the greatest fixed point $\nu(F_{R})$?  As remarked earlier, $\nu(F_{R})$ contains
$\mu(F_{R})$, but is the containment strict?  The answer depends on the choice of subjects
for the assertions in $X$.  For an object to be in the least fixed point of a rule set
means that it must be forced to be so by applying rules.  But to be in the greatest fixed
point means only that if an assertion is in it, and if it arises as the conclusion of some
rule in $R$, then the premises of that rule must also be present.  This condition allows
for circular reasoning in the presence of circular syntactic objects.  For example,
suppose that $\Omega$ is the infinite stack of successors
$\succnat(\succnat(\succnat(\dots)))$, which is to say that $\Omega=\succnat{\Omega}$, it is its own
successor.\footnote{If you are wondering how this could come about, just consider that the
  syntactic objects are graphs, rather than trees, with cycles allowed.}  Considering the
greatest fixed point interpretation of the rules defining $\IsNat{-}$, it is the case that
$\IsNat{\Omega}$, because $\Omega=\succnat{\Omega}$ satisfies the requirement that if
$\IsNat{\Omega}$, then $\IsNat{\Omega}$, which is tautologous.



\bibliographystyle{plainnat}
\bibliography{notes}

\end{document}

%%% Local Variables:
%%% mode: latex
%%% TeX-master: t
%%% fill-column: 90
%%% auto-fill-mode: t
%%% End:
