\documentclass[11pt,twoside]{article}
\usepackage[authoryear,semicolon]{natbib}
\usepackage[T1]{fontenc}
\usepackage[utf8]{inputenc}
\usepackage{textgreek}
\usepackage{fullpage}
\usepackage[color=yellow]{todonotes}
\setlength{\marginparwidth}{1.25in}
\usepackage{xifthen}
\usepackage{amsmath,amssymb,amsthm,mathtools,stmaryrd}
\usepackage{proof,mathpartir}
\usepackage{colonequals}
\usepackage{code,verbatim}
\usepackage{comment}
\usepackage{textcomp}
\usepackage[us]{optional}
\usepackage{color}
\usepackage{url}
\usepackage{graphics}
\usepackage{import}
\usepackage{stackengine}
\usepackage{scalerel}

\newcommand{\IsNat}[1]{{#1}\,\mathsf{nat}}
\newcommand{\zeronat}{\mathtt{zero}}
\newcommand{\succnat}[1]{\mathtt{succ}(#1)}

\newcommand{\IsEven}[1]{{#1}\,\mathsf{even}}
\newcommand{\IsOdd}[1]{{#1}\,\mathsf{odd}}

\newcommand{\eqdef}{\mathrel{\triangleq}}

\allowdisplaybreaks[1]       %mildly permissible to break up displayed equations

\begin{document}

\title{Tarski's Theorem on Power Sets}
\author{Robert Harper}
\date{\today}

\maketitle{}

\section{Introduction}

Tarski's theorem states that a monotone function on a complete lattice has a complete
lattice of fixed points, in particular a least and greatest. A useful class of special
cases are powerset lattices ordered by inclusion.

\section{Powerset Lattices}

Let $X$ be a set, not necessarily non-empty, and let $\wp{X}$ be the set of all subsets of
$X$. The set $\wp{X}$ forms a complete lattice under set inclusion, with meets given by
intersection and joins given by union. That is, if $\mathcal{X}\subset\wp{X}$, then
$\bigcap{\mathcal{X}}$ is its meet (greatest lower bound) and $\bigcup{\mathcal{X}}$ is
its join (least upper bound). The least element is the join of the empty set, namely
$\emptyset$, and the greatest element is the meet of the empty set, namely $X$.

A function $F:\wp{X}\to\wp{X}$ is \emph{monotone} if it preserves inclusion: if
$A\subseteq B\subseteq X$, then $F(A)\subseteq F(B)\subseteq X$. For monotone $F$ on
$\wp{X}$, a \emph{pre-fixed point} of $F$ is a set $A\subseteq X$ such that
$F(A)\subseteq A$, and a \emph{post-fixed point} of $F$ is a set $A\subseteq X$ such that
$X\subseteq F(X)$. A pre-fixed point of $F$ is also said to be \emph{$F$-closed} and a post-fixed
point of $F$ is said to be \emph{$F$-consistent}. A \emph{least pre-fixed point} of a
monotone $F$ is the \emph{smallest} (with respect to containment) $F$-closed set, and a
\emph{greatest post-fixed point} of $F$ is the \emph{largest} (with respect to
containment) $F$-consistent set.  Viewing the lattice as a (skinny) category, a monotone
function $F$ on it is a \emph{functor}, a pre-fixed point of it is an \emph{$F$-algebra}
and a post-fixed point of it is an \emph{$F$-coalgebra}. Thus, a \emph{least} pre-fixed
point of $F$ is an \emph{initial $F$-algebra} and a \emph{greatest} post-fixed point of
$F$ is a \emph{final $F$-coalgebra}.

Every monotone $F:\wp{X}\to\wp{X}$ has a (unique) least pre-fixed point and (unique)
greatest post-fixed point, given by the equations
\begin{align*}
  \mu(F) & = \bigcap\{\,A\subseteq X\,\mid\, F(A)\subseteq A\,\} \\
  \nu(F) & = \bigcup\{\,A\subseteq X\,\mid\,A\subseteq F(A)\,\}
\end{align*}
It is evident that $\mu(F)$ is contained in all pre-fixed points of $F$, it being their
intersection.  In fact $\mu(F)$ is itself a pre-fixed point of $F$,
$F(\mu(F))\subseteq \mu(F)$, and it is thereby the least pre-fixed point.  To see this, it is enough
to show that if $F(A)\subseteq A$, then $F(\mu(F))\subseteq A$.  But if $F(A)\subseteq A$, then
$\mu(F)\subseteq A$ by definition of $\mu(F)$ as the greatest lower bound of all pre-fixed point.
Then, by monotonicity, $F(\mu(F))\subseteq F(A)\subseteq A$, as required.  Then again by monotonicity
$F(F(\mu(F)))\subseteq F(\mu(F))$, which is to say that $F(\mu(F))$ is a pre-fixed point of
$F$, and therefore $\mu(F)\subseteq F(\mu(F))$.  That is, $\mu(F)$ is a fixed point of
$F$, and, because any fixed point is a pre-fixed point, it is the least such.  Dually,
$\nu(F)$ contains all post-fixed points of $F$, it being their union, and, arguing dually to
the preceding, $\nu(F)$ is itself a post-fixed point of $F$.  It is thus the greatest
post-fixed point, and hence a fixed point.  (Categorially, this is Lambek's Lemma, which
states that the initial $F$-algebra and final $F$-coalgebra are isomorphisms.)

\smallskip

The least fixed point of a monotone $F$ on $\wp{X}$ affords the following \emph{induction
  principle}: to show that $\mu(F)\subseteq A$, it suffices to show that
$F(A)\subseteq A$, which is to say that $A$ is $F$-closed.  Similarly, the greatest post-fixed
point, $\nu(F)$, of $F$ affords the following \emph{coinduction principle}: to show that
$A\subseteq \nu(F)$, it suffices to show that $A\subseteq F(A)$, which is to say that $A$ is
$F$-consistent.  Re-phrased in terms of predicates and implication, the least fixed point
of a monotone $F$ is the \emph{strongest} property $A$ of elements of $X$ such that if
$x\in F(A)$, then $x\in A$. Thus, to show that $x\in\mu(F)$ implies $x\in A$, it is enough to show
that if $x\in F(A)$, then $x\in A$.  Dually, the greatest fixed point of a monotone $F$ is the
\emph{weakest} property $A$ of elements of $X$ such that if $x\in A$, then $x\in F(A)$.

As a case in point there are two proofs that $\mu(F)\subseteq\nu(F)$, one using the minimality of
$\mu(F)$, the other using the maximality of $\nu(F)$.  Because $\mu(F)$ is the least pre-fixed
point of $F$, it is itself $F$-closed, $F(\mu(F))\subseteq\mu(F)$, and because $\nu(F)$ is the greatest
post-fixed point of $F$, it is itself $F$-consistent.  Thus, to show the containment it
suffices to show either that $\nu(F)$ is $F$-closed, $F(\nu(F))\subseteq\nu(F)$, or that
$\mu(F)$ is $F$-consistent, $\mu(F)\subseteq F(\mu(F))$.  But these are exactly the converse
containments that were obtained earlier to show that $\mu(F)$ and $\nu(F)$ are fixed points of
$F$.

\smallskip

Yet another perspective on the least pre-fixed point and greatest post-fixed point of a
monotone $F$ is provided by the following visualization of inductive and coinductive
proofs.  To show that every element of $\mu(F)$ is also in some set $A$ representing a
property of interest, it is enough to show that $\mu(F)\cap A$ is closed under $F$.  For if it
is, then $\mu(F)$ is contained in this intersection, and hence in $A$.  The intersection
$\mu(F)\cap A$ is \emph{a priori} smaller than $\mu(F)$, consisting only of those elements of
$\mu(F)$ that are ``good enough'' to have property $A$.  But if the intersection is not so
restrictive as to not be closed under $F$, then in fact the intersection is no restriction
at all, it being that $\mu(F)\cap A=\mu(F)$.  Dually, to show that some collection
$A\subseteq X$ of elements is contained in $\nu(F)$, it is enough to boldly assert that they are by
forming $\nu(F)\cup A$, which is \emph{a priori} larger than $\nu(F)$.  But if the union is
consistent with $F$, then the assertion of $A$ is indefeasible, and hence
$\nu(F)\cup A\subseteq\nu(F)$.  The union is not, in fact, larger than $\nu(F)$ after all---the elements of
$A$ were present from the get-go.

\smallskip

It is also possible to show that the meet and join of any set of fixed points of a
monotone function is itself a fixed point, but this seems not to be as useful as the
construction of a least and greatest.

\smallskip

It sometimes arises that two sets (properties) are to be \emph{simultaneously} inductively
defined because the definition of each depends on the other.  This situation can be
expressed by considering two monotone operators $F, G : \wp{X}\times\wp{X}\to\wp{X}$ in the sense that
if $A\subseteq A'$ and $B\subseteq B'$, then $F(A,B)\subseteq F(A',B)$ and
$G(A,B)\subseteq G(A,B')$.  The operator $(F,G)(A,B)\eqdef (F(A,B),G(A,B))$ is therefore monotone
with respect to the componentwise ordering, $(A,B)\subseteq (A',B')$ iff $A\subseteq A'$ and
$B\subseteq B'$.  By arguments analogous to those given above\footnote{This is where the (mild)
  generalization of Tarski's Theorem to complete lattices is helpful.} it has a least
fixed point, $\mu(F, G)$, a pair $(A_{0},B_{0})$ of subsets of $X$ such that
$F(A_{0},B_{0})=A_{0}$ and $G(A_{0},B_{0})=B_{0}$, each ``cross-referencing'' the other as
intended.

It is also possible to manage the interdependencies by an iterated process of taking least
fixed points of monotone operators on $\wp{X}$.  This reduction of simultaneous to iterated
least fixed points is known as \emph{Bekic's Lemma}.\footnote{The formulation given here
  is adapted from~\citet{davey-priestley} Chapter 8, exercises 8.30 and 8.31.}  Given $F$
and $G$ as above, define their curried forms by $F_{B}(A)\eqdef{} F(A,B)$, which fixes the
$B$ argument of $F$, and $G^{A}(B)=G(A,B)$, which fixes the $A$ argument of $G$.  These
are both monotone, and hence admit least fixed points,
$\mu(F_{B})=F_{B}(\mu(F_{B}))=F(\mu(F_{B}),B)$, which solves for $A$ parametrically in
$B$, and $\mu(G^{A})=G^{A}(\mu(G^{A}))=G(A,\mu(G^{A}))$, which solves for $B$ parametrically in
$A$.  The maps $A\mapsto F(A,\mu(G^{A}))$ and $B\mapsto G(\mu(F_{B}),B)$ are monotone, and hence have
least fixed points.  Bekic's Lemma states that the simultaneous and iterated fixed points
agree:
$$\mu(F,G) = (\mu(A\mapsto F(A,\mu(A\mapsto G(A,B)))),\mu(B\mapsto G(\mu(A\mapsto F(A,B)),B))).$$

The proof of Bekic's Lemma is not difficult, but it is easy to get lost in the morass of
fixed points.  As suggested in \textit{loc. cit.} it is clearer to separate the pattern of
diagonalization, and then instantiate it to the present situation.  Suppose that
$H:\wp{X}\times\wp{X}\to\wp{X}$ is monotone in each argument.  Define
$\Delta_{H}(A)=H(A,A)$, the diagonalization of $H$.  Let $L_{H}(A) \eqdef{} \mu(H^{A})$ and
$R_{H}(B) \eqdef{} \mu(H_{B})$.  Note that $L_{H}(A)=H(A,L_{H}(A))$ and
$R_{H}(B) = H(R_{H}(B),B)$.  The Diagonal Lemma states that
$\mu(\Delta_{H})=\mu(L_{H})=\mu(R_{H})$.  For the first of these equations, calculate:
\begin{align*}
  \mu(L_{H}) & = L_{H}(\mu(L_{H})) \\
           & = H(\mu(L_{H}),L_{H}(\mu(L_{H}))) \\
           & = H(\mu(L_{H}),\mu(L_{H})) \\
           & = \Delta_{H}(\mu(L_{H})).
\end{align*}
But then $\mu(\Delta_{H})\subseteq\mu(L_{H})$, it being the least pre-fixed point of
$\Delta_{H}$.  Conversely, to show that $\mu(L_{H})\subseteq\mu(\Delta_{H})$, it suffices to show that
$L_{H}(\mu(\Delta_{H}))\subseteq\mu(\Delta_{H})$.  But since $L_{H}(\mu(\Delta_{H}))=\mu(H^{\mu(\Delta_{H})})$, it is enough to
show that $H^{\mu(\Delta_{H})}(\mu(\Delta_{H}))\subseteq\mu(\Delta_{H})$, as follows:
\begin{align*}
  H^{\mu(\Delta_{H})}(\mu(\Delta_{H}))
  & = H(\mu(\Delta_{H}),\mu(\Delta_{H})) \\
  & = \Delta_{H}(\mu(\Delta_{H})) \\
  & = \mu(\Delta_{H}).
\end{align*}
A similar calculation shows that $\mu(R_{H})=\mu(\Delta_{H})$, which completes the proof.

To prove Bekic's Lemma, simply apply the Diagonal Lemma to $\Delta_{(F,G)}$, and note that
$\mu(L_{F})$ and $\mu(R_{G})$ are equal to the desired iterated forms.

\section{Assertions and Rules}

A typical application of the fixed point constructions is to justify the definition of one
or more \emph{assertions}, or \emph{formal judgments}, by a collection of
\emph{rules}. The idea is that the rules constitute an inductive definition of the
mentioned assertions.  For example, the following rules define the judgment $\IsNat{n}$,
stating that $n$ is a natural number:
\begin{mathpar}
  \inferrule[zero]
  {\strut}
  {\IsNat{\zeronat}}

  \inferrule[succ]
  {\IsNat{n}}
  {\IsNat{\succnat{n}}}
\end{mathpar}
Similarly, the even and odd numbers may be simultaneously defined by the following rules:
\begin{mathpar}
  \inferrule[zero-even]
  {\strut}
  {\IsEven{\zeronat}}

  \inferrule[succ-odd]
  {\IsEven{n}}
  {\IsOdd{\succnat{n}}}

  \inferrule[succ-even]
  {\IsOdd{n}}
  {\IsEven{\succnat{n}}}
\end{mathpar}
In both cases the subjects of the assertions are abstract binding trees in the sense
of~\citet{pfpl2}, among which are those used above.

The forms of assertion may be thought of as labels that distinguish one from another.  Thus, the
underlying set $X$ of the inductive definition is the collection of \emph{assertions} consisting of
an abt together with a label drawn from the set of such forms.  Each rule $r$ is of the form
\begin{displaymath}
  \inferrule
  {j_{1} \dots j_{n}}
  {j}
\end{displaymath}
wherein the $j_{i}$'s and $j$ are assertions (elements of $X$).  Each such rule $r$
determines a monotone function $F_{r}:\wp{X}\to\wp{X}$ that applies rule $r$ to a given set
$A\subseteq X$ of assertions:
\begin{displaymath}
  F_{r}(A)=\{\, j\in X \,\mid\, j_{1},\dots,j_{n}\in A\,\}.
\end{displaymath}
A set $R$ of rules induces a monotone function that collectively closes up under each rule in the
set:
\begin{displaymath}
  F_{R}(A) = \bigcup_{r\in R} F_{r}(A).
\end{displaymath}
The assertions defined by the set of rules $R$ are precisely those in $\mu(F_{R})$.  The
principle of \emph{rule induction} is simply the induction principle associated with the
least fixed point of $F_{R}$.

\smallskip

What about the greatest fixed point $\nu(F_{R})$?  As remarked earlier, $\nu(F_{R})$ contains
$\mu(F_{R})$, but is the containment strict?  The answer depends on the choice of subjects
for the assertions in $X$.  For an object to be in the least fixed point of a rule set
means that it must be forced to be so by applying rules.  But to be in the greatest fixed
point means only that if an assertion is in it, and if it arises as the conclusion of some
rule in $R$, then the premises of that rule must also be present.  This condition allows
for circular reasoning in the presence of self-referential syntactic objects.  For
example, suppose that $\omega$ is the infinite stack of successors
$\succnat{\succnat{\succnat{\dots}}}$, which is to say that $\omega=\succnat{\omega}$, it is its own
successor.\footnote{If you are wondering how this could come about, just consider that the
  syntactic objects are graphs, rather than trees, with cycles allowed.}  Considering the
greatest fixed point interpretation of the rules defining $\IsNat{-}$, it is the case that
$\IsNat{\omega}$, because $\omega=\succnat{\omega}$ satisfies the requirement that if
$\IsNat{\omega}$, then $\IsNat{\omega}$, which is tautologous.

\smallskip

Finally, a word about ``side conditions'' on rules is in order\footnote{\citet{avron91:scr} calls these
  ``impurities.''}.  Often rules are not presented fully schematically, but with so-called
side conditions that constrain their applicability.  For example, in the typing rule for
$\lambda$-abstractions it is usual to include the requirement that $x$ is not already declared
in $\Gamma$:
\begin{displaymath}
  \inferrule
  {\Gamma,x{:}A\vdash M:B \\ x\notin\Gamma}
  {\Gamma\vdash\lambda x{:}A.M:A\to B}
\end{displaymath}
When terms are identified up to $\alpha$-equivalence, the restriction can always be met by
choosing an appropriate representative, because $\Gamma$ declares only finitely many of the
infinitely many choices for the bound variable.  In that case the condition may be omitted
from the rule, it being understood implicitly.

Whereas in the preceding case the restrictions on the applicability of the rule are
benign, it is entirely possible, even common, to abuse the privilege to the point of
absurdity.  For example, it is entirely meaningless to state as a premise of a rule the
\emph{negation} of the judgment being defined, as if to say ``this rule is applicable only
if the stated assertion is not derivable.''  Doing this, or things tantamount to it, ruins
the inductive character of the intended definition, precisely because the associated
operator is no longer monotone!  And without that the rules cannot be said to define
anything at all.  For a particularly blatant case, consider the supposed inductive
definition of an assertion $j$ by the rule
\begin{displaymath}
  \inferrule
  {\neg j}
  {j}
\end{displaymath}
Were there a fixed point of the associated (non-monotone) operator, the assertion would be
derivable iff it is not, so obviously it must not have one.  More subtle examples are
observed in the wild, so it is wise to be careful out there!

\nocite{davey-priestley}

\bibliographystyle{plainnat}
\bibliography{notes}

\end{document}

%%% Local Variables:
%%% mode: latex
%%% TeX-master: t
%%% fill-column: 90
%%% auto-fill-mode: t
%%% End:
