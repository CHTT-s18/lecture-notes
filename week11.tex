\documentclass{article}
\usepackage{chtt-notes}
\scribes{Di Wang}
\week{11}
% The following command will let you cross-reference labels
% in the files week1.tex, week2.tex, \ldots, week\@week.tex,
% where if l is a label in ``weekN.tex'', then you can access
% the label using \cref{WN:l}.
\doXRs

% General remark: Using \cref{label} will fill in the appropriate
% environment name. For example,
% ``\begin{lemma}\label{lem:foo} ...\end{lemma} By \cref{lem:foo}''
% will produce ``Lemma 15 ... By lemma~15''

\newcommand{\calU}{\mathcal{U}}
\newcommand{\bbI}{\mathbb{I}}
\newcommand{\Irecop}{\mathsf{Irec}}
\newcommand{\Irec}[4]{\Irecop(#1;#2;#3)(#4)}

\begin{document}
\maketitle

\section{Setting the Scene}

Last week, we further explored some properties of the semantics of 
computational dependent type theory, such as functionality, P.E.R valued, 
symmetry, and transitivity.
Then we revisited Intensional Type Theory (ITT) and claimed there is no 
notion of function extensionality, and we even can not prove $0 \neq 1$ in ITT.
We enriched ITT by introducing \emph{universes}, i.e., types of types, and 
studied higher identification $\mathsf{Id}_\calU$ in a universe $\calU$.
We explored the \emph{groupoid} structure of identification.
Intuitively, for two types $A,B : \calU$, $\Id{\calU}{A}{B}$ is inhabited if 
there is an ``isomorphism'' $f : A \to B$ between $A$ and $B$.
The question remains how to express the idea in the type theory.

This week, we are going to study Homotopy Type Theory (HoTT), which is an 
extension of ITT with the following two ideas:
\begin{enumerate}
	\item \textbf{Univalence:} types are identified up to \emph{equivalence}, 
	which is 
	defined based on the idea of isomorphism.
	\item \textbf{Higher inductive types:} types can specify both \emph{points} 
	and 
	\emph{paths}, with respect to identification.
\end{enumerate}

\section{Univalence}

The question we are going to address in this section is the following:
\begin{quotation}
	How do we characterize $\Id{\calU}{A}{B}$ for types $A,B:\calU$ in ITT 
	enriched with universes?
\end{quotation}

Before proceeding, we review some notions of ITT:
\begin{table}[h]
	\centering
	\begin{tabular}{c@{\hspace{3em}}l}
		$\Id{A}{M}{N}, \Ideq{A}{M}{N}$ & identity type, identification type, path 
		type \\
		$\Idelim{a,b,c.C}{a.Q}{P}$ & elimination form of the identity type, path 
		induction \\
		$\refl{A}{M}$ & introduction form of the identity type \\
		$\transport{a.B}(P)$ & transport
	\end{tabular}
\end{table}

Intuitively, two types $A$ and $B$ are equal iff they are isomorphic up to 
identification.
First, we observe that any function $f : A \to B$ preserves identification.

\begin{lemma}
	If $f : A \to B$ and $P : \Ideq{A}{M}{M'}$, then there is an operation 
	$\hap{f}{P} : \Ideq{B}{f(M)}{f(M')}$, such that $\hap{f}{\refl{A}{M}} 
	\equiv \refl{B}{f(M)}$.
\end{lemma}
\begin{proof}
	We give a definition of $\hap{f}{P}$ by path induction on $P$:
	\[
	\hap{f}{P} \eqdef  \Idelim{a,b,-. \Ideq{B}{f(a)}{f(b)}}{a.\refl{B}{f(a)}}{P}.
	\]
\end{proof}

Then we want to extend this result to pi-types $f : 
\hpitype{a}{A}{B}$.
At the first attempt, we might write down the following formulation.

\begin{claim}
	If $f : \hpitype{a}{A}{B}$ and $P : \Ideq{A}{M}{M'}$, then there is an 
	operation $\hapd{f}{P} : \Ideq{B}{f(M)}{f(M')}$.
\end{claim}

However, this statement will not typecheck---$B$ could depend on $a$, thus 
$f(M)$ and 
$f(M')$ could have different types $\subst{M}{a}{B}$ and 
$\subst{M'}{a}{B}$, 
respectively.
In other words, we need to come up with a notion of \emph{heterogeneous 
paths} between $f(M)$ and $f(M')$.


Recall that \emph{transport} has the following property:
If $a : A \vdash B~\mathsf{type}$ and $P : \Ideq{A}{M}{M'}$, then 
$\transport{a.B}(P) : \subst{M}{a}{B} \to \subst{M'}{a}{B}$, as well as 
$\transport{a.B}(\refl{A}{M})(Q) \equiv Q$ for any $Q : \subst{M}{a}{B}$.
Then we can define heterogeneous paths between $f(M)$ and $f(M')$ as:
\[
\DIdeq{a.B}{P}{f(M)}{f(M')} \eqdef  \Ideq{ \subst{M'}{a}{B}  
}{\transport{a.B}(P)(f(M))}{f(M')}.
\]

\begin{lemma}
	If $f : \hpitype{a}{A}{B}$ and $P: \Ideq{A}{M}{M'}$, then there is an 
	operation $\hapd{f}{P} : \DIdeq{a.B}{P}{f(M)}{f(M')}$.
\end{lemma}
\begin{proof}
	We give a definition of $\hapd{f}{P}$ by path induction on $P$:
	\[
	\hapd{f}{P} \eqdef \Idelim{a,b,p. 
	 \DIdeq{a.B}{p}{f(a)}{f(b)} }{a. \refl{B}{f(M)} 
	}{P}.
	\]
\end{proof}

Then we need to reason about when a function $f : A \to B$ is a bijection up to 
identification.
Intuitively, $f$ is a bijection if for any $b : B$, $f^{-1}(b)$ contains exactly 
one element.
To proceed, we give a formal account of $f^{-1}(b)$.

\begin{definition}[Fibers]
	For any $b : B$, its fiber is defined as $f^{-1}(b) \eqdef \hsigtype{a}{A}{ 
	\Ideq{B}{f(a)}{b} }$.
\end{definition}

Rather than containing a single element, we want to express the property as 
containing a \emph{center} such that every element has a path to the center.

\begin{definition}[Contractibility]
	A type $A$ is contractible iff there exists a center $c : A$ such that every 
	element $a : A$ is identical to $c$.
	Formally, the following type expresses contractibility:
	\[
	\mathsf{isContractible}(A) \eqdef  \hsigtype{c}{A}{ \hpitype{a}{A}{ 
			\Ideq{A}{a}{c} } }.
	\]
	A contractible type is also called a \emph{singleton}.
\end{definition}

Now we are able to express that $f : A \to B$ is a bijection up to identification.

\begin{definition}[Equivalences]
	$f : A \to B$ is an equivalence between types $A$ and $B$, iff for any $b : B$, 
	its fiber is contractible.
	Formally, the following type expresses this property:
	\[
	\mathsf{isEquiv}(f) \eqdef  
	\hpitype{b}{B}{\mathsf{isContractible}(f^{-1}(b))}.
	\]
	Further, we say that two types $A,B : \calU$ are \emph{equivalent}, iff there 
	is an equivalence $f : A \to B$.
	Formally, the following type characterizes this property:
	\[
	A \simeq B \eqdef \hsigtype{f}{A \to B}{\mathsf{isEquiv}(f)}.
	\]
\end{definition}

Recall that our intuition is that two types $A$ and $B$ are equal iff they are 
isomorphic up to identification.
We might come up with the following claim:

\begin{proposition}[Pre-univalence]
	$\Ideq{\calU}{A}{B}$ has inhabitants if and only if $A \simeq B$ does.
\end{proposition}

We can strengthen this idea by claiming the two types $\Ideq{\calU}{A}{B}$ 
and $A \simeq 
B$ are indeed equivalent.
In fact, we are able to define a map from $\Ideq{\calU}{A}{B}$ to $A 
\simeq B$.

\begin{lemma}
	There is a function $\mathsf{idtoequiv}_{A,B} : (\Ideq{\calU}{A}{B}) \to (A 
	\simeq 
	B)$.
\end{lemma}
\begin{proof}
	For any path $p : \Ideq{\calU}{A}{B}$, we define 
	$\mathsf{idtoequiv}_{A,B}(p)$ 
	by path induction on $p$:
	\[
	\mathsf{idtoequiv}_{A,B} \eqdef \lambda p : \Ideq{\calU}{A}{B}. 
	\Idelim{a,b,-. a 
	\simeq b}{a.  \langle \mathsf{id}_a,  
	\lambda x : a. \mathsf{pf\_contr}  \rangle 
	}{p}
	\]
	where $a : \calU, x : a \vdash \mathsf{pf\_contr} : 
	\mathsf{isContractible}((\mathsf{id}_a)^{-1}(x))$.
	Let $T \eqdef (\mathsf{id}_a)^{-1}(x)$.
	$\mathsf{pf\_contr}$ is defined as
	\[
	\mathsf{pf\_contr} \eqdef \langle  \langle x, 
	\refl{a}{x} \rangle , 
	\lambda y : 
	T.  \refl{T}{ \langle y.\mathsf{1}, y.\mathsf{2}  \rangle   } \rangle.
	\]
\end{proof}

However, we could not \emph{prove} that $\mathsf{idtoequiv}_{A,B}$ is an 
equivalence.
Instead, we introduce an \emph{axiom} to ensure it is an equivalence.

\begin{axiom}[Univalence]
	$\mathsf{idtoequiv}_{A,B}$ is an equivalence.
	For $A,B : \calU$ and $E : A \simeq B$, the axiom gives us a term 
	$\mathsf{ua}(A,B,E) : \Ideq{\calU}{A}{B}$.
\end{axiom}

\begin{remark}
	The introduction of univalence axiom breaks computationality.
	Suppose we are evaluating $\Idelim{a,b,p.C}{a.Q}{\mathsf{ua}(A,B,E)}$.
	The computation is stuck because $\mathsf{ua}(A,B,E)$ is given by an axiom 
	and thus irreducible.
	The insight is that we should define entailment \emph{before} implication, 
	i.e., first express identity judgmentally, and then internalize the identity into 
	the language.
	We will explore this idea in future lectures.
\end{remark}

We then turn to show that ITT extended with univalence has function 
extensionality.

\begin{definition}[Total path spaces]
	The total path space in a type $A$ is characterized by the type 
	$\mathsf{Paths}(A) \eqdef 
	\hsigtype{a}{A}{\hsigtype{b}{A}{\Ideq{A}{a}{b}}}$.
	In addition, $\hsigtype{b}{A}{\Ideq{A}{a}{b}}$ is called the \emph{based 
	path space} at $a$.
	Based path spaces are contractible by definition.
\end{definition}

\begin{lemma}
	$A$ is equivalent to $\mathsf{Paths}(A)$, i.e., $A \simeq 
	\mathsf{Paths}(A)$ has inhabitants.
\end{lemma}
\begin{proof}
	We prove this by showing there are a map $f$ from $A$ to 
	$\mathsf{Paths}(A)$ and a map $g$ from $\mathsf{Paths}(A)$ to $A$, as 
	well as $f$ and $g$ are mutually inverse.
	\[
	\begin{array}{rcl}
	f & \eqdef & \lambda a :A. \langle a, a, \refl{A}{a} \rangle \\
	g & \eqdef & \lambda t : \mathsf{Paths}(A). t.\mathsf{1}.
	\end{array}
	\]
	It is straightforward to show that $\Ideq{A}{g(f(a))}{a}$ for all $a : A$.
	We need to show that $\Ideq{\mathsf{Paths}(A)}{f(g(t))}{t}$ for all $t : 
	\mathsf{Paths}(A)$, i.e., $\Ideq{\mathsf{Paths}(A)}{\langle t.\mathsf{1}, 
	t.\mathsf{1}, \refl{A}{t.\mathsf{1}} \rangle }{t}$.
	This is proved by the fact that $\langle t.\mathsf{1}, 
	\refl{A}{t.\mathsf{1}} \rangle$ belongs to the based path space at 
	$t.\mathsf{1}$, which is contractible.
\end{proof}

By univalence axiom we obtain the following identification.

\begin{corollary}
	$A$ is identical to $\mathsf{Paths}(A)$, i.e., 
	$\Ideq{\calU}{A}{\mathsf{Paths}(A)}$ has inhabitants.
\end{corollary}

\begin{definition}[Homotopies]
	The homotopy between types $A$ and $B$ is characterized by the 
	type $\mathsf{Htpy}(A,B) \eqdef \hsigtype{f,g}{A \to B}{A \sim B}$, where 
	$A \sim B \eqdef \hpitype{a}{A}{\Ideq{B}{f(a)}{g(a)}}$.
\end{definition}

\begin{lemma}
	$A \to \mathsf{Paths}(B)$ is equivalent to $\mathsf{Htpy}(A,B)$, i.e., $(A 
	\to \mathsf{Paths}(B)) \simeq \mathsf{Htpy}(A,B)$ has inhabitants.
\end{lemma}
\begin{proof}
	We prove this by showing there are a map $f$ from $A \to 
	\mathsf{Paths}(B)$ to $\mathsf{Htpy}(A,B)$ and a map $g$ from 
	$\mathsf{Htpy}(A,B)$ to $A \to \mathsf{Paths}(B)$, as well as $f$ and $g$ 
	are mutually inverse.
	\[
	\begin{array}{rcl}
	f & \eqdef & \lambda F : A \to \mathsf{Paths}(B). \langle \lambda a : A. 
	F(a).\mathsf{1}, \lambda a : A. F(a).\mathsf{2}, \lambda a:A. 
	F(a).\mathsf{3} \rangle \\
	g & \eqdef & \lambda H : \mathsf{Htpy}(A,B). \lambda a : A. \langle 
	H.\mathsf{1}(a), H.\mathsf{2}(a), H.\mathsf{3}(a) \rangle.
	\end{array}
	\] 
	It is straightforward to show they are mutually inverse.
\end{proof}

By univalence axiom we obtain the following identification.

\begin{corollary}
	$A \to \mathsf{Paths}(B)$ is identical to $\mathsf{Htpy}(A,B)$, i.e., 
	$\Ideq{\calU}{A \to \mathsf{Paths}(B)}{\mathsf{Htpy}(A,B)}$ has 
	inhabitants.
\end{corollary}

\begin{lemma}
	$\mathsf{Paths}(A \to B)$ is identical to $\mathsf{Htpy}(A,B)$, i.e., 
	$\Ideq{\calU}{\mathsf{Paths}(A \to B)}{\mathsf{Htpy}(A,B)}$ has 
	inhabitants.
\end{lemma}
\begin{proof}
	By transitivity of identification and lemma above we can derive the following:
	\[
	\begin{array}{cl}
	& \mathsf{Paths}(A \to B) \\
	=_{\calU} & A \to B \\
	=_{\calU} & A \to \mathsf{Paths}(B) \\
	=_{\calU} & \mathsf{Htpy}(A,B).
	\end{array}
	\]
\end{proof}

\begin{corollary}
	Univalence entails function extensionality.
\end{corollary}
\begin{proof}
	Function extensionality states that identical functions map the same 
	arguments to identical results.
	Identical functions in $A \to B$ are captured by $\mathsf{Paths}(A\to B) = 
	\hsigtype{f,g}{A\to B}{ \Ideq{A \to B}{f}{g} }$.
	Functions that map the same arguments to identical results are captured  by 
	$\mathsf{Htpy}(A,B) = \hsigtype{f,g}{A \to B}{ \hpitype{a}{A}{ 
	\Ideq{A}{f(a)}{g(a)} } }$.
	Because $\mathsf{Paths}(A \to B)$ is identical to $\mathsf{Htpy}(A,B)$, we 
	conclude function extensionality in ITT extended with univalence.
\end{proof}

\section{Higher Inductive Types}

\renewcommand{\natrecop}{\mathsf{natrec}}

Inductive types are essentially characterized by a ``mapping out'' property.
Recall the formalization of natural numbers.
We define two introduction forms (for zero and successors, respectively), and 
one elimination form (i.e., natural recursion).
Then we claim the natural number type is the least thing that satisfies the 
specification.
\begin{mathpar}
	\inferrule[\rn{I-zero}]{ }{ \Gamma \vdash \natz : \mathsf{Nat} }
	\and
	\inferrule[\rn{I-succ}]{ \Gamma \vdash M : \mathsf{Nat} }{ \Gamma \vdash 
	\nats{M} : \mathsf{Nat}  }
	\and
	\inferrule[\rn{E}]{ \Gamma,n:\mathsf{Nat} \vdash A~\mathsf{type} \\ 
	\Gamma \vdash M : \mathsf{Nat} \\ \Gamma \vdash M_0 : 
	[\natz/n]A \\ \Gamma,n:\mathsf{Nat},a:A \vdash M_1 : [\nats{n}/n]A  }{ 
	\Gamma \vdash 
	\natrec{n.A}{M_0}{n.a}{M_1}{M} : 
	\subst{M}{n}{A} 
	}
\end{mathpar}

As a generalization of the idea, in defining a higher inductive type, we might 
have constructors for not only \emph{points}, but also \emph{paths} in that 
type.
In other words, an inductively defined type $A$ is somehow coupled to 
$\Id{A}{-}{-}$, $\Id{A}{-}{-}$ is somehow coupled to 
$\Id{\Id{A}{-}{-}}{*}{*}$, and so on.

As an example, we explore the \emph{interval} type, denoted by $\bbI$.
Following are the introduction forms.
\begin{mathpar}
	\inferrule{ }{ \Gamma \vdash 0 : \bbI }
	\and
	\inferrule{ }{ \Gamma \vdash 1 : \bbI }
	\and
	\inferrule{ }{ \Gamma \vdash \mathsf{seg} : \Id{\bbI}{0}{1} } 
\end{mathpar}

\begin{fact}
	$\bbI$ is contractible.
\end{fact}

The elimination form is trickier than introduction forms---we should keep in 
mind that the type is the least thing that satisfies our specification.
At the first attempt, we assume the result type of the elimination does not 
depend on the value to be eliminated.
\begin{mathpar}
	\inferrule{ \Gamma \vdash A ~\mathsf{type} \\ \Gamma \vdash M : \bbI \\ 
	\Gamma \vdash M_0 : A \\ \Gamma \vdash M_1 : A \\ \Gamma \vdash 
	M_{\mathsf{seg}} : \Id{A}{M_0}{M_1} }{ \Gamma \vdash 
	\Irec{M_0}{M_1}{M_{\mathsf{seg}}}{M} : A }
\end{mathpar}

The following properties should hold:
\[
\begin{array}{l}
\Irec{M_0}{M_1}{M_{\mathsf{seg}}}{0} \equiv M_0 \\
\Irec{M_0}{M_1}{M_{\mathsf{seg}}}{1} \equiv M_1 \\
\Ideq{\Id{A}{M_0}{M_1}}{ \hap{ 
\Irec{M_0}{M_1}{M_{\mathsf{seg}}}{\cdot} 
}{M_{\mathsf{seg}}} }{ M_{\mathsf{seg}}}
\end{array}
\]

Then we consider the general case where $A$ could depend on the value of 
$M$.
Similar to previous discussion on $\mathsf{ap}$ and $\mathsf{apd}$, we 
make use of transport to establish heterogeneous paths.
\begin{mathpar}
	\inferrule{ \Gamma, i : \bbI \vdash A~\mathsf{type} \\ \Gamma \vdash M : 
	\bbI \\ \Gamma \vdash M_0 : \subst{0}{i}{A} \\ \Gamma \vdash M_1 : 
	\subst{1}{i}{A} \\ \Gamma \vdash \DIdeq{i.A}{\mathsf{seg}}{M_0}{M_1} 
	}{ \Gamma \vdash \Irec{M_0}{M_1}{M_\mathsf{seg}}{M} : 
	\subst{M}{i}{A} }
\end{mathpar}

\begin{lemma}
	$\mathsf{Paths}(A)$ is identical to $\bbI \to A$.
\end{lemma}
\begin{proof}
	By univalence axiom, it suffices to show that $\mathsf{Paths}(A)$ is 
	equivalent to $\bbI \to A$.
	We then show there are a map $f$ from $\mathsf{Paths}(A)$ to $\bbI \to A$ 
	and a map $g$ from $\bbI \to A$ to $\mathsf{Paths}(A)$, as well as $f$ and 
	$g$ are mutually inverse.
	\[
	\begin{array}{rcl}
	f & \eqdef & \lambda t : \mathsf{Paths}(A). \lambda i : \bbI. 
	\Irec{t.\mathsf{1}}{t.\mathsf{2}}{t.\mathsf{3}}{i} \\
	g & \eqdef & \lambda h : \bbI \to A. \langle h(0), h(1), 
	\hap{h}{\mathsf{seg}} \rangle.
	\end{array}
	\]
	It is straightforward to show they are mutually inverse.
\end{proof}

\section{Outlook}

We will finish exploration of HoTT by investigating another example of higher 
inductive types, the \emph{circle} type, which contains a base (as a 
point) and a loop (as a path).
Then we will redevelop the idea of \emph{dimensionality} directly and turn to 
discuss cubical type theory, which takes care of computationality and 
univalence becomes definable in the framework.

\bibliographystyle{plainnat}
\bibliography{ctt}

\end{document}
