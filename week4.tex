\documentclass{article}
\usepackage{chtt-notes}
\scribes{Yue Niu and Charles}
\week{4}
\doXRs

\begin{document}

\maketitle
\section{Girard's Method}

Recall hereditary termination we discovered through several trials: 
\[
\htermFF{A}{M}{\delta}{\eta}{\Theta}
\]
Read as $M$ is hereditarily terminating at $A$, where $delta$ is a closing type variable substitution and $\eta$ is 
a candidate assingment, all with respect to $\Theta$. Since it's usually clear from context, we drop $\Theta$ unless it's
required. 

To make precise the argument of type candidates, we need to make precise what they mean: 

\emph{Type candidates} are any collection $\mathcal{C}$ of \emph{closed, erased} terms such that:
\begin{enumerate}
\item $\mathcal{C}$ is closed under head expansion, and possibly 
\item $T \in \mathcal{C}$ and $T(M) \implies \bterm{M}$
\end{enumerate}

Before presenting the FTLR for system F, we will need the following lemma: 
\begin{lemma}[Compositionality]\label{lem:comp}
$\htermF{[A/X]B}{M}{\delta}{\eta}$ iff $\htermF{B}{M}{\delta[X \mapsto A]}{\eta[X \mapsto \htermF{A}{-}{\delta}{\eta}]}$
\end{lemma}

\begin{proof}
Induction on $B$.
\begin{itemize}
\setlength\itemsep{1em}
\item $B = X$\\
For the forward direction, suppose $\htermF{[A/X]X}{M}{\delta}{\eta}$. We need to show that  
$\htermF{X}{M}{\delta[X \mapsto A]}{\eta[X \mapsto \htermF{A}{-}{\delta}{\eta}]}$. By definition, it sufficese to show
$(\eta[X \mapsto \htermF{A}{-}{\delta}{\eta}])(X)(M)$, or that $\htermF{A}{M}{\delta}{\eta}$, which follows from the 
assumption.\\

For the backward direction, suppose $\htermF{X}{M}{\delta[X \mapsto A]}{\eta[X \mapsto \htermF{A}{-}{\delta}{\eta}]}$. 
We need to show that $\htermF{[A/X]X}{M}{\delta}{\eta}$, or $\htermF{A}{M}{\delta}{\eta}$, which follows by unfolding
the assumption. 

\item $B = X'$ where $X \ne X'$\\ 
For the forward direction, suppose $\htermF{[A/X]X'}{M}{\delta}{\eta}$. We need to show that  
$\htermF{X'}{M}{\delta[X \mapsto A]}{\eta[X \mapsto \htermF{A}{-}{\delta}{\eta}]}$. By definition, it sufficese to show
$(\eta[X \mapsto \htermF{A}{-}{\delta}{\eta}])(X')(M)$. Since $ X \ne X'$, 
it suffices to show $\eta(X')(M)$, which holds by assumption. \\

For the backward direction, suppose $\htermF{X'}{M}{\delta[X \mapsto A]}{\eta[X \mapsto \htermF{A}{-}{\delta}{\eta}]}$. 
We need to show that $\htermF{[A/X]X'}{M}{\delta}{\eta}$, or $\htermF{X'}{M}{\delta}{\eta}$. By definition, it suffices
to show that $\eta(X')(M)$. By assumption, we know $(\eta[X \mapsto \htermF{A}{-}{\delta}{\eta}])(X')(M)$ holds. Since
$X \ne X'$, $(\eta[X \mapsto \htermF{A}{-}{\delta}{\eta}])(X') = \eta(X')$, and we have that $\eta(X')(M)$ holds.

\item $B = \fn{A_1}{A_2}$\\
For the forward direction, suppose $\htermF{[A/X]B}{M}{\delta}{\eta}$. We need to show that  
$\htermF{B}{M}{\delta[X \mapsto A]}{\eta[X \mapsto \htermF{A}{-}{\delta}{\eta}]}$. 
Suppose $\htermF{A_1}{N}{\delta[X \mapsto A]}{\eta[X \mapsto \htermF{A}{-}{\delta}{\eta}]}$. It suffices to show
then $\htermF{A_2}{\ap{M}{N}}{\delta[X \mapsto A]}{\eta[X \mapsto \htermF{A}{-}{\delta}{\eta}]}$. 
By IH, we have $\htermF{[A/X]XA_1}{N}{\delta}{\eta}$. Further, from assumption we get
$\htermF{\fn{[A/X]A_1}{[A/X]A_2}}{M}{\delta}{\eta}$. By the definition of hereditary termination, we get 
$\htermF{[A/X]A_2}{\ap{M}{N}}{\delta}{\eta}$, and the result follows from the IH.

For the backward direction, suppose 
$\htermF{\fn{A_1}{A_2}}{M}{\delta[X \mapsto A]}{\eta[X \mapsto \htermF{A}{-}{\delta}{\eta}]}$. 
We need to show that $\htermF{[A/X]\fn{A_1}{A_2}}{M}{\delta}{\eta}$, 
or $\htermF{\fn{[A/X]A_1}{[A/X]A_2}}{M}{\delta}{\eta}$. Suppose $\htermF{[A/X]A_1}{N}{\delta}{\eta}$. 
It suffices to show $\htermF{[A/X]A_2}{\ap{M}{N}}{\delta}{\eta}$. By IH, we have 
$\htermF{A_1}{N}{\delta[X \mapsto A]}{\eta[X \mapsto \htermF{A}{-}{\delta}{\eta}]}$.
Along with the assumption, we get $\htermF{A_2}{\ap{M}{N}}{\delta[X \mapsto A]}{\eta[X \mapsto \htermF{A}{-}{\delta}{\eta}]}$, and the result follows from IH.

\item $B = \all{Z}{B'}$\\
For the forward direction, suppose $\htermF{[A/X]\all{Z}{B'}}{M}{\delta}{\eta}$. We need to show that  
$\htermF{\all{Z}{B'}}{M}{\delta[X \mapsto A]}{\eta[X \mapsto \htermF{A}{-}{\delta}{\eta}]}$. Let $C$ be a closed type, 
and $T$ a type candidate. Letting $\delta' = \delta[X \mapsto A]$ and 
$\eta' = \eta[X \mapsto \htermF{A}{-}{\delta}{\eta}]$, it sufficese to show
$\htermF{B'}{\ap{\erase{M}}{\downarrow}}{\delta'[Z \mapsto C]}{\eta'[Z \mapsto T]}$. By the assumption (and 
capture-avoiding substitution), we have $\htermF{\all{Z}{[A/X]B'}}{M}{\delta}{\eta}$. 
Now instantiate the RHS definition with $C$ and $T$, obtaining 
$\htermF{[A/X]B'}{\ap{\erase{M}}{\downarrow}}{\delta[Z \mapsto C]}{\eta[Z \mapsto T]}$. 
Let $\delta'' = \delta[Z \mapsto C]$ and $\eta'' = \eta[Z \mapsto T]$. By IH, we have 
$\htermF{B'}{\ap{\erase{M}}{\downarrow}}{\delta''[X \mapsto A]}{\eta''[X \mapsto \htermF{A}{-}{\delta''}{\eta''}]}$, 
which is what was needed. \footnote{ is the interpretation of $Z$ irrelevant when considering $A$?}\\

For the backward direction, suppose 
$\htermF{\all{Z}{B'}}{M}{\delta[X \mapsto A]}{\eta[X \mapsto \htermF{A}{-}{\delta}{\eta}]}$.
We need to show that $\htermF{[A/X]\all{Z}{B'}}{M}{\delta}{\eta}$, or $\htermF{\all{Z}{[A/X]B'}}{M}{\delta}{\eta}$. 
Let $C$ be a closed type, and $T$ a type candidate. By definition, it suffices
to show $\htermF{[A/X]B'}{\ap{\erase{M}}{\downarrow}}{\delta[Z \mapsto C]}{\eta[Z \mapsto T]}$. 
Instantiating the assumption with $C$ and $T$, we have $\htermF{B'}{\ap{\erase{M}}{\downarrow}}{\delta[X \mapsto A][Z \mapsto C]}{\eta[X \mapsto \htermF{A}{-}{\delta}{\eta}][Z \mapsto T]}$. 
By IH, we have $\htermF{[A/X]B'}{\ap{\erase{M}}{\downarrow}}{\delta[Z \mapsto C]}{\eta[Z \mapsto T]}$. \footnote{There needs to be a weakening of the interpretation in order to apply the IH.}
\end{itemize}
\end{proof}

\begin{theorem}[FTLR for System F]
If $\hasEF{\Theta;\Omega}{M}{A}$, and 
\begin{enumerate}
\item $\delta : \Theta$ is a closing type substitution 
\item $\eta : \Theta$ is a candidate assignment 
\item $\gamma : \cdot \to \Gamma$ is a closig term substitution
\item $\htermF{\Gamma}{\gamma}{\delta}{\eta}$
\end{enumerate}
then $\htermF{A}{\hat\gamma(\erase{M})}{\delta}{\eta}$.
\end{theorem}

\begin{proof}
Induction on typing.
\begin{itemize}
\setlength\itemsep{1em}
\item $\inferrule{
  \hasEF{\Theta, X; \Gamma}{M}{B}
  }{
  \hasEF{\Theta; \Gamma}{\tlam{X}{M}}{\all{X}{B}}
  }$\\
Fix $\delta : \Theta, \eta : \Theta$ and $\htermF{\Gamma}{\gamma}{\delta}{\eta}$. We need to show that 
$\htermFF{\all{X}{B}}{\hat\gamma(\erase{\tlam{X}{M}})}{\delta}{\eta}{\Theta}$, or that
$\htermFF{\all{X}{B}}{\alam{\hat\gamma(\erase{M})}}{\delta}{\eta}{\Theta}$.
Let $C$ be a closed type, $T$ a type candidate, 
$\delta' = \delta[X \mapsto C]$, $\eta' = \eta[X \mapsto T]$, and $\Theta' = \Theta,X$. 
We need to show that $\htermFF{B}{\ap{\alam{\hat\gamma(\erase{M})}}{\downarrow}}{\delta'}{\eta'}{\Theta'}$. 
By head expansion, it suffices to show
that $\htermFF{B}{\hat\gamma(\erase{M})}{\delta'}{\eta'}{\Theta'}$, which follows from IH. 

\item $\inferrule{
  \hasEF{\Theta;\Gamma}{M}{\all{X}{B}}\\
  \hasTF{\Theta}{C}
  }{
  \hasEF{\Theta;\Gamma}{\tap{M}{C}}{[C/X]B}
  }$\\
Fix $\delta : \Theta, \eta : \Theta$ and $\htermF{\Gamma}{\gamma}{\delta}{\eta}$. We need to show that 
$\htermF{[C/X]B}{\hat\gamma(\erase{\tap{M}{C}})}{\delta}{\eta}$, or that
$\htermF{[C/X]B}{\ap{\hat\gamma(\erase{M})}{\downarrow}}{\delta}{\eta}$. By compositionality, it suffices to show that
$\htermF{B}{\ap{\hat\gamma(\erase{M})}{\downarrow}}{\delta[X \mapsto C]}{\eta[X \mapsto \htermF{A}{-}{\delta}{\eta}]}$.
By IH, we have $\htermF{\all{X}{B}}{\hat\gamma(\erase{M})}{\delta}{\eta}$. Instantiating the RHS with $A$ and 
$\htermF{A}{-}{\delta}{\eta}$ (which by Lemma \ref{stc} is a type candidate), we get
$\htermF{B}{\ap{\hat\gamma(\erase{M})}{\downarrow}}{\delta[X \mapsto C]}{\eta[X \mapsto \htermF{A}{-}{\delta}{\eta}]}$.
\end{itemize}
\end{proof}

\begin{corollary}
$\htermF{\ans}{M}{\emptyset}{\emptyset} \implies \bterm{M}$.
\end{corollary}

In order to extend this to all types, we need to stipulate termination for all type candidates and formulate hereditary
termination positively. 

\section{Equality}

The plan from now on:
\begin{enumerate}
\item Equality: formal vs semantic
\item Parametricity / data abstraction by ``binarizing'' Girard's method
\item Propositions as types and dependent types
\item CTT / HDTT
\end{enumerate}

Recall definitional (structural) equality (equivalence) in formal type theory, originating from Gentzen's inversion 
principles. D.E. can be charaterized as follows:
\begin{enumerate}
\item RST (is a equivalence relation)
\item Compatible with the term formers (is a congruence)
\item ``Calculates'' via beta-reduction (but is not directed)
\end{enumerate}

D.E. is a very strong ``equality'', and doesn't support hypothetical reasoning. For instance, if we defined plus in 
G{\"o}del's T (by recursion on one of the arguments), 
we will not be able to prove $\hasEEF{x : \nat, y : \nat}{x + y}{y + x}{\nat}$, even though for all numerals
$\bar n, \bar m$, we \emph{can} derive $\hasEEF{}{\bar n + \bar m}{\bar m + \bar n}{\nat}$.
\bibliographystyle{plainnat}
\bibliography{ctt}

\end{document}
